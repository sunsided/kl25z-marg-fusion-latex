\addto{\captionsngerman}{
  \renewcommand{\acronymname}{Abkürzungen}
}

\addto{\captionsngerman}{
  \renewcommand{\glossaryname}{Glossar}
}

\newglossaryentry{Cortex}{name = {Cortex}, description={32bit-Mikrocontrollerarchitektur von ARM, Inc}}
\newglossaryentry{cortex-m0}{name = {Cortex-M0}, description={Mikroprozessor der M0-Serie der ARM \gls{Cortex}-Architektur}}
\newglossaryentry{cortex-m0+}{name = {Cortex-M0+}, description={Mikroprozessor der M0+-Serie der ARM \gls{Cortex}-Architektur}}
\newglossaryentry{CoreSight}{name = {CoreSight}, description={Debug-Schnittstelle in \gls{Cortex}-\glspl{mcu}}}

\newglossaryentry{Kinetis}{name = {Kinetis}, description={32bit-Mikrocontrollertyp von Freescale Semiconductor}}
\newglossaryentry{kl25z}{name = {KL25Z}, description={Mikrocontroller der \gls{Kinetis}-L-Serie von Freescale auf Basis des ARM Cortex-M0+}}
\newglossaryentry{frdm-kl25z}{name = {FRDM-KL25Z}, description={Freescale Freedom Development Board mit KL25Z-\gls{mcu}}}

\newglossaryentry{Gimbal Lock}{name = {Gimbal Lock}, description={Kardanische Blockade eines Systemes bei der Verwendung von Euler'schen Winkeln, die bei ungünstiger Kombination von Rotationen zum Verlust eines Freiheitsgrades führt}}

\newglossaryentry{Kalman-Filter}{name = {Kalman-Filter}, description={Rekursives, lineares Filter zur Schätzung stochastischer Systemparameter, dessen Entwicklung auf Rudolf Emil Kálmán zurückgeht}}
\newglossaryentry{OpenSDA}{name = {OpenSDA}, description={Proprietäre, erweiterbare Programmierschnittstelle von Freescale}}
\newglossaryentry{OpenOCD}{name = {OpenOCD}, description={Quelloffene Programmierschnittstelle}}
\newglossaryentry{JTAG}{name = {JTAG}, description={Schnittstelle für Tests von Controllerschnittstellen und Programmierung}}
\newglossaryentry{elf}{name = {ELF}, description={Executable and Linking Format, eine ausführbare Datei}}
\newglossaryentry{mbed}{name = {mbed}, description={Plattform für die Entwicklung auf Cortex-M-MCUs}}
\newglossaryentry{Thumb2}{name = {Thumb2}, description={Befehlssatz von \glslink{Cortex}{ARM Cortex-M}-Prozessoren}}

\newglossaryentry{Quaternion}{name = {Quaternion}, description={Hamilton-Zahl $\mathbb{H}$ im 4-dimensionalen komplexer Raum, die zur singularitätsfreien Beschreibung von Orientierungen verwendet werden kann}}

\newglossaryentry{hard iron}{name = {Hard-Iron-Effekt}, description={Lineare Verzerrung im Magnetfeld durch Einwirkung "`harter"' ferromagnetischer Metalle, vgl. Soft-Iron-Effekt}}
\newglossaryentry{soft iron}{name = {Soft-Iron-Effekt}, description={Nichtlineare Verzerrung im Magnetfeld durch Einwirkung "`weicher"' ferromagnetischer Metalle, vgl. Hard-Iron-Effekt}}

\newglossaryentry{systick}{name = {SysTick}, description={System Tick Interrupt: Konfigurierbarer Timerinterrupt des Cortex-Cores zur Zeitmessung, welcher die Realisierungen von Echtzeitanwendungen unterstützen soll.}}

\newglossaryentry{bit banding}{name = {Bit-Banding}, description={Mapping von Speicheradresse auf einzelne Bits einer anderen Adresse}}

\newglossaryentry{tilt compensation}{name = {Tilt Compensation}, description={Korrektur der dreidimensionalen Neigung des Magnetometer-Messvektors mittels zusätzlicher Ausrichtungsdaten zur Bestimmung der magnetischen Nordrichtung in der zweidimensionalen Ebene}}


\newglossaryentry{Komplementaerfilter}{name = {Komplementärfilter}, description={Wichtungsfilter zwischen Beschleunigungs- und Drehratensensor zur Korrektur der spezifischen Drift- und Rauschmerkmale}}

\newglossaryentry{triad}{name = {TRIAD}, description={Algorithmus zur Orientierungsdeterminierung aus drei Vektorbeobachtungen}}

\newglossaryentry{q16}{name = {Q16}, description={32bit-Zahlenformat für Festkommazahlen mit jeweils 16 bit für Vor"- und Nachkommaanteil}}

\newglossaryentry{mpug}{name = {MPU}, description={Motion Processing Unit, eine durch Kopplung externer Sensoren erweiterbare Sensorserie von Invensense zur Verarbeitung und Fusion von Inertialdaten.}}

\newglossaryentry{i2cg}{name = {I\textsuperscript{2}C}, description={(auch IIC) Inter-Integrated Circuit. Ein von Philips entworfenes Bussystem für Inter-Chip-Kommunikation}}

\newglossaryentry{imug}{name = {IMU}, description={Inertial Measurement Unit, dt. Inertialmesseinheit; Ein Sensorsystem bestehend aus mehreren kombinierten Inertialsensoren}}

\newacronym{i2c}{I\textsuperscript{2}C}{Inter-Integrated Circuit\glsadd{i2cg}}
\newacronym{mpu}{MPU}{Motion Processing Unit\glsadd{mpug}}

\newacronym{spi}{SPI}{Serial Peripheral Interface}

\newacronym{dmp}{DMP}{Digital Motion Processor}
\newacronym{mac}{MAC}{Multiply and Accumulate}

%\newacronym{i2c}{I2C}{Inter-Integrated Circuit}

\newacronym[
	\glsshortpluralkey={MEMS},
	\glslongpluralkey={mikroelektromechanische Systeme}
]{mems}{MEMS}{mikroelektromechanisches System}

\newacronym{marg}{MARG}{Magnetic, Angular Rate and Gravitational}

\newacronym[
	\glsshortpluralkey={IMUs},
	\glslongpluralkey={Inertial Measurement Units}
]{imu}{IMU}{Inertial Measurement Unit\glsadd{imug}}

\newacronym{hci}{HCI}{Human-Computer Interaction}
\newacronym{hid}{HID}{Human Interface Device}
\newacronym{cmsis}{CMSIS}{Cortex Microcontroller Software Interface Standard}
\newacronym{cmsis-dap}{CMSIS-DAP}{\gls{cmsis} Debug Access Port}
\newacronym[
	\glsshortpluralkey={MCUs},
	\glslongpluralkey={Microcontroller Units}
]{mcu}{MCU}{Microcontroller Unit}
\newacronym{ocd}{OCD}{On-Chip-Debugger}
\newacronym{msd}{MSD}{Mass Storage Device}
\newacronym{usb}{USB}{Universal Serial Bus}
\newacronym{gdb}{GDB}{GNU Debugger}
\newacronym{dcm}{DCM}{Direction Cosine Matrix}
\newacronym{swd}{SWD}{Serial Wire Debug}
\newacronym{request}{REQUEST}{Recursive Quaternion Estimator}
\newacronym{quest}{QUEST}{Quaternion Estimator}

\newacronym{uart}{UART}{Universal Asynchronous Receiver Transmitter}

\newacronym[
	\glsshortpluralkey={GPIOs},
	\glslongpluralkey={General-Purpose I/Os}
]{gpio}{GPIO}{General-Purpose I/O}


\newacronym{soh}{SOH}{Start of Header}
\newacronym{eot}{EOT}{End of Transmission}
\newacronym{esc}{ESC}{Escape}

\newacronym{p2pp}{P2PP}{Point-to-Point Protocol}

\newacronym{wnu}{WNU}{West-North-Up}
\newacronym{enu}{ENU}{East-North-Up}
\newacronym{ecef}{ECEF}{Earth-Centered, Earth-Fixed}
\newacronym{fpu}{FPU}{Floating-Point Unit}

\newacronym{bme}{BME}{Bit Manipulation Engine}
\newacronym{bfi}{\texttt{BFI}}{Bit Field Insert}
\newacronym{lac}{\texttt{LAC}}{Load-and-Clear}
\newacronym{las}{\texttt{LAS}}{Load-and-Set}
\newacronym{ubfx}{\texttt{UBFX}}{Usigned Bit Field Extract}
\newacronym{pll}{PLL}{Phase-Locked Loop}