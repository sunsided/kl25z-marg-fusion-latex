%\setpartpreamble[Position][Breite]{Text}
\setpartpreamble[u, r][6cm]{Dies ist ein Beispiel für eine Part-Präambel.}
\part[]{Entwicklungsumgebung}
\chapter{Entwicklungsumgebung}

\section{Freescale Kinetis KL25Z}

Auf jedem Freescale Freedom-Entwicklungsboard wie dem \gls{frdm-kl25z} ist  außerdem ein \gls{ocd} verbaut, welcher zusätzlich zum Debugging die 
Fähigkeit besitzt, die \gls{mcu} zu programmieren. Freescale nennt diese Einheit \gls{OpenSDA} (ganz im Sinne der "`Freedom"'-Benamung der Boards), 
die zwar entgegen ihres Namens proprietärer Natur ist, jedoch die zusätzliche Beschaffung eines externen Programmers erspart.

\section{Freescale OpenSDA}

Der \gls{OpenSDA}-Chip auf dem  kommuniziert dabei mit der im \gls{Cortex}-Kern integrierten \gls{CoreSight}-Einheit über das \gls{JTAG}-Protokoll.
Der Chip selbst kann dabei mit einer alternativen Firmware geflasht werden (vgl. Abb. \ref{fig:opensda-block}), welche eine spezialisierte 
(d.h. treiberspezifische) Ansteuerung der \gls{CoreSight}-Einheit ermöglicht.

\begin{figure}[htbp]
		\centering
		\includegraphics{./images/opensda-block-diagram.png}
    \caption[OpenSDA-Blockdiagramm]{OpenSDA-Blockdiagramm}
		Quelle: \texttt{http://mcuoneclipse.com/2012/09/20/opensda-on-the-freedom-kl25z-board/}
    \label{fig:opensda-block}
\end{figure}

Die Standard-Firmware (PEmicro OpenSDA MSD) repräsentiert sich dabei nach dem Verbinden mit dem USB-Host als \gls{msd} und ist in der Lage, die \gls{mcu}
durch einfaches Drag \& Drop einer \gls{elf}-Datei programmiert werden. Debugging-Funktionalität bietet dieses Interface jedoch nicht, wodurch der
erste Schritt der Inbetriebnahme in den meisten Fällen darin bestehen wird, eine alternative Firmware (z.B. PEmicro OpenSDA DEBUG) zu flashen. Hierdurch
verliert man zwar die Funktionalität des USB-\gls{msd} und ist somit auf eine Programmersoftware angewiesen, diese ist jedoch z.B. in Form der Toolchain
der Freescale CodeWarrior-IDE (siehe Abschnitt~\ref{subsec:codewarrior} auf Seite~\pageref{subsec:codewarrior}) gegeben.

Eine Gegenüberstellung der alternativen SDA-Firmwares ohne Anspruch auf Vollständigkeit kann Tabelle \ref{tab:opensda-options} entnommen werden.

\subsection{OpenSDA unter Windows 8.1}
\label{subsec:opensda_windows8.1}

Beachtet werden muss, dass die \gls{msd}-Schnittstelle der \gls{OpenSDA}-Firmware unter Windows 8.1 nicht korrekt funktioniert! Dies ist der strengeren Handhabung
der USB Device-Deskriptoren unter der neueren Windows-Version geschuldet, wodurch die Antwort der OpenSDA-Firmware abgelehnt wird, was zu abbrechenden
Dateitransfers und damit fehlerhaft geflashter Firmware führt.

Um den \gls{msd}-Modus zum Programmieren zu verwenden oder die Firmware selbst umzuflashen muss demnach wahlweise Windows 7 oder niedriger oder ein Linux-basiertes System
verwendet werden, um das Laufwerk zu mounten. Die Verwendung einer virtuellen Maschine zum umprogrammieren der \gls{OpenSDA}-Firmware wurde versucht,
verlief jedoch aufgrund der Handhabung des USB-redirects erfolglos.

Eine Beschreibung des Vorgehens zur umprogrammierung des \gls{OpenSDA}-Chips unter Linux ist in Anhang~\ref{chap:opensda_linux} gegeben.

\begin{table}
\begin{tabular}{lllll} 
\toprule
Gegenüberstellung von OpenSDA-Firmwares\\  
\midrule 
Name & Art & Breakpoints & UART & MSD \\ 
\midrule 
P\&E OpenSDA MSD & proprietär & (kein Debugging) & ja & ja\\
P\&E OpenSDA & proprietär & hardware & ja & nein \\
CMSIS-DAP & open source & hardware & nein & nein \\
Segger J-Link & proprietär & unbegrenzt & nein & nein \\ 
\bottomrule
\end{tabular}
\caption{Gegenüberstellung von OpenSDA-Firmwares}
\label{tab:opensda-options}
\end{table}

\subsection{CMSIS-DAP}

Eine alternative (und quelloffene) Firmware für den \gls{OpenSDA}-Chip ist die \gls{cmsis-dap}-Firmware.

--> Verwendung mit \gls{OpenOCD}

\begin{figure}[htbp]
		\centering
		\includegraphics[width=0.8\textwidth]{./images/cmsis_dap_interface.png}
    \caption[CMSIS-DAP]{CMSIS-DAP}
		Quelle: \texttt{http://nimblemachines.com/cmsis-dap/}
    \label{fig:cmsis-dap}
\end{figure}


\subsection{Segger J-Link}

\lipsum[2]

\section{Integrierte Entwicklungsumgebungen}

\subsection{Freescale CodeWarrior}
\label{subsec:codewarrior}

\lipsum[3]

\subsection{SysProgs VisualGDB}

\lipsum[4]