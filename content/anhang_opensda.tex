\chapter{Umprogrammieren der OpenSDA-Firmware unter Linux}
\label{chap:opensda_linux}

\lstdefinestyle{lolbash}{
    language={bash}, 
		basicstyle=\ttfamily\tiny,
		frame=single,
    moredelim=**[is][\slshape]{`}{`},
    moredelim=**[is][\color{orange}]{°}{°},
}

Wie in Abschnitt~\ref{subsec:opensda_windows8.1} erwähnt, kann das \gls{frdm-kl25z} unter Windows 8.1 nicht ordnungsgemäß
in Betrieb genommen werden, sobald die Firmware einen \gls{msd}-Modus verwendet. Dies betrifft sowohl den Programmer
der \gls{mcu}, als auch den Programmer des \gls{OpenSDA}-Chips selbst.

Im Folgenden findet sich eine Beschreibung des Flashvorganges unter Linux, nachdem das Board auch unter Windows 8.1
wieder verwendet werden kann. Dieser Vorgang sollte auch dann zum Erfolg führen, wenn das Board unter Windows 8.1
bereits fehlerhaft (d.h. mit Abbruch des Dateitransfers) geflasht wurde.

Zuerst wird das Board bei gedrückter Reset-Taste angeschlossen, wodurch es im Bootloader-Modus startet. Die Ausgabe 
von \texttt{lsusb} sollte das Gerät mit der ID \texttt{2504:0200} auflisten.

\begin{lstlisting}[style=lolbash]
user@host:~$ lsusb
Bus 002 Device 001: ID 1d6b:0002 Linux Foundation 2.0 root hub
Bus 007 Device 001: ID 1d6b:0001 Linux Foundation 1.1 root hub
Bus 006 Device 001: ID 1d6b:0001 Linux Foundation 1.1 root hub
°Bus 005 Device 008: ID 2504:0200°
Bus 005 Device 001: ID 1d6b:0001 Linux Foundation 1.1 root hub
Bus 001 Device 001: ID 1d6b:0002 Linux Foundation 2.0 root hub
Bus 004 Device 001: ID 1d6b:0001 Linux Foundation 1.1 root hub
Bus 003 Device 002: ID 0483:2016 STMicroelectronics Fingerprint Reader
Bus 003 Device 003: ID 0a5c:2110 Broadcom Corp. BCM2045B (BDC-2) [Bluetooth Controller]
Bus 003 Device 001: ID 1d6b:0001 Linux Foundation 1.1 root hub
\end{lstlisting}

Mittels \texttt{dmesg} wird nun ermittelt, als welches Device das USB-Gerät angemeldet wurde.

\begin{lstlisting}[style=lolbash]
user@host:~$ dmesg
[ 2178.532100] usb 5-2: new full-speed USB device number 10 using uhci_hcd
[ 2178.703183] usb 5-2: New USB device found, idVendor=2504, idProduct=0200
[ 2178.703193] usb 5-2: New USB device strings: Mfr=1, Product=2, SerialNumber=3
°[ 2178.703201] usb 5-2: Product: OpenSDA MSD APP°
°[ 2178.703207] usb 5-2: Manufacturer: FREESCALE SEMICONDUCTOR INC.°
°[ 2178.703213] usb 5-2: SerialNumber: 0123456789ABCDEF°
°[ 2178.706301] usb-storage 5-2:1.0: USB Mass Storage device detected°
[ 2178.706467] scsi11 : usb-storage 5-2:1.0
°[ 2179.709257] scsi 11:0:0:0: Direct-Access     FSL      FSL/PEMICRO MSD  0001 PQ: 0 ANSI: 4°
[ 2179.709869] sd 11:0:0:0: Attached scsi generic sg2 type 0
°[ 2179.718217] sd 11:0:0:0: [sdb] 1983999 512-byte logical blocks: (1.01 GB/968 MiB)°
[ 2179.721525] sd 11:0:0:0: [sdb] Write Protect is off
[ 2179.721531] sd 11:0:0:0: [sdb] Mode Sense: 00 00 00 00
[ 2179.724171] sd 11:0:0:0: [sdb] Asking for cache data failed
[ 2179.724183] sd 11:0:0:0: [sdb] Assuming drive cache: write through
[ 2179.742182] sd 11:0:0:0: [sdb] Asking for cache data failed
[ 2179.742187] sd 11:0:0:0: [sdb] Assuming drive cache: write through
[ 2179.766202]  sdb:
[ 2179.783187] sd 11:0:0:0: [sdb] Asking for cache data failed
[ 2179.783193] sd 11:0:0:0: [sdb] Assuming drive cache: write through
[ 2179.783197] sd 11:0:0:0: [sdb] Attached SCSI removable disk
\end{lstlisting}

Man kann erkennen, dass das Laufwerk als \texttt{/dev/sdb} angemeldet wurde.
Es kann nun mittels \texttt{mount -t vfat} gemountet und der Erfolg überprüft werden.

\begin{lstlisting}[style=lolbash]
user@host:~$ sudo mount -t vfat /dev/sdb /mnt
user@host:~$ ls -lisa /mnt
insgesamt 84
  1 16 drwxr-xr-x  2 root root 16384 Jan  1  1970 .
  2  4 drwxr-xr-x 23 root root  4096 Nov 21 02:12 ..
102 16 -r-xr-xr-x  1 root root   512 Aug  8  2012 FSL_WEB.HTM
100 16 -r-xr-xr-x  1 root root    68 Aug  8  2012 LASTSTAT.TXT
101 16 -r-xr-xr-x  1 root root  1536 Aug  8  2012 SDA_INFO.HTM
103 16 -r-xr-xr-x  1 root root   512 Aug  8  2012 TOOLS.HTM
\end{lstlisting}

Es ist zu beachten, dass der Mountvorgang durchaus mehrere Minuten dauern kann.

Die Datei \texttt{LASTSTAT.TXT} beinhaltet den letzten Status der Firmware, womit man
den erfolgreichen Verlauf des Mountvorganges überprüfen kann.

\begin{lstlisting}[style=lolbash]
user@host:~$ cat /mnt/LASTSTAT.TXT
°Ready.°
\end{lstlisting}

Die neue Firmware kann nun durch einen Kopierbefehl an das Gerät gesendet werden; Im Falle
der \emph{Segger J-Link}-Variante könnte dies etwa wie folgt aussehen:

\begin{lstlisting}[style=lolbash]
user@host:~$ sudo cp JLink_OpenSDA.sda /mnt
user@host:~$ cat /mnt/LASTSTAT.TXT
°Completed.°
\end{lstlisting}

Ein \texttt{Completed.} signalisiert den erfolgreichen Flash-Vorgang.

Nach dem Auswerfen des Laufwerkes mittels \texttt{umount}

\begin{lstlisting}[style=lolbash]
user@ahost:~$ sudo umount /mnt
\end{lstlisting}

und einem Neustart des Boards im regulären Modus (d.h. ohne gedrückten Reset-Taster) meldet 
sich das Board mit der neuen Firmware an:

\begin{lstlisting}[style=lolbash]
user@host:~$ lsusb
Bus 002 Device 001: ID 1d6b:0002 Linux Foundation 2.0 root hub
Bus 007 Device 001: ID 1d6b:0001 Linux Foundation 1.1 root hub
Bus 006 Device 001: ID 1d6b:0001 Linux Foundation 1.1 root hub
°Bus 005 Device 011: ID 1366:0101 SEGGER J-Link ARM°
Bus 005 Device 001: ID 1d6b:0001 Linux Foundation 1.1 root hub
Bus 001 Device 001: ID 1d6b:0002 Linux Foundation 2.0 root hub
Bus 004 Device 001: ID 1d6b:0001 Linux Foundation 1.1 root hub
Bus 003 Device 002: ID 0483:2016 STMicroelectronics Fingerprint Reader
Bus 003 Device 003: ID 0a5c:2110 Broadcom Corp. BCM2045B (BDC-2) [Bluetooth Controller]
Bus 003 Device 001: ID 1d6b:0001 Linux Foundation 1.1 root hub
\end{lstlisting}

Analog kann die Ausgabe mittels \texttt{dmesg} überprüft werden.

\begin{lstlisting}[style=lolbash]
user@host:~$ dmesg
[ 2637.380139] usb 5-2: USB disconnect, device number 10
[ 2638.312070] usb 5-2: new full-speed USB device number 11 using uhci_hcd
[ 2638.479164] usb 5-2: New USB device found, idVendor=1366, idProduct=0101
[ 2638.479176] usb 5-2: New USB device strings: Mfr=1, Product=2, SerialNumber=3
°[ 2638.479183] usb 5-2: Product: J-Link°
°[ 2638.479189] usb 5-2: Manufacturer: SEGGER°
°[ 2638.479195] usb 5-2: SerialNumber: 000621000000°
\end{lstlisting}

Wird stattdessen die \emph{P\&E Microcomputer Debug OpenSDA}-Variante geflasht, lauten die letzten
Ausgaben sinngemäß

\begin{lstlisting}[style=lolbash]
user@host:~$ lsusb
Bus 002 Device 001: ID 1d6b:0002 Linux Foundation 2.0 root hub
Bus 007 Device 001: ID 1d6b:0001 Linux Foundation 1.1 root hub
Bus 006 Device 001: ID 1d6b:0001 Linux Foundation 1.1 root hub
°Bus 005 Device 013: ID 1357:0089 P&E Microcomputer Systems°
Bus 005 Device 001: ID 1d6b:0001 Linux Foundation 1.1 root hub
Bus 001 Device 001: ID 1d6b:0002 Linux Foundation 2.0 root hub
Bus 004 Device 001: ID 1d6b:0001 Linux Foundation 1.1 root hub
Bus 003 Device 002: ID 0483:2016 STMicroelectronics Fingerprint Reader
Bus 003 Device 003: ID 0a5c:2110 Broadcom Corp. BCM2045B (BDC-2) [Bluetooth Controller]
Bus 003 Device 001: ID 1d6b:0001 Linux Foundation 1.1 root hub
\end{lstlisting}

und 

\begin{lstlisting}[style=lolbash]
sunside@aquitaine:~$ dmesg | tail
[ 3015.141218] sd 12:0:0:0: [sdb] Attached SCSI removable disk
[ 3346.412156] usb 5-2: USB disconnect, device number 12
[ 3347.768119] usb 5-2: new full-speed USB device number 13 using uhci_hcd
[ 3347.950185] usb 5-2: New USB device found, idVendor=1357, idProduct=0089
[ 3347.950195] usb 5-2: New USB device strings: Mfr=1, Product=3, SerialNumber=5
°[ 3347.950202] usb 5-2: Product: OpenSDA Hardware°
°[ 3347.950208] usb 5-2: Manufacturer: P&E Microcomputer Systems Inc.°
°[ 3347.950214] usb 5-2: SerialNumber: SDADBB27E5D°
[ 3347.953292] cdc_acm 5-2:1.0: This device cannot do calls on its own. It is not a modem.
[ 3347.953332] cdc_acm 5-2:1.0: ttyACM0: USB ACM device
\end{lstlisting}