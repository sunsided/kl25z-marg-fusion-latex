\section{Saleae Logic}
\label{sec:logan}

Zur �berpr�fung der externen Interrupt-Signale und der Buskommunikation wurde der 8-Kanal Logic Analyzer \emph{logic} der Firma Saleae verwendet (Abbildung~\ref{fig:logan-bild}). 
Dieser ist als digitales Speicheroszilloskop angelegt und in der Lage, bei Abtastraten von bis zu $24\mega\hertz$ bis zu 10 Milliarden Samples aufzunehmen. Triggerungen
auf Logiklevel, sowie Flanken ist dabei m�glich.

\begin{figure}[htbp]
		\centering
		\includegraphics[width=0.6\textwidth]{./images/saleae-logic.jpg}
    \caption[Saleae logic]{Saleae logic\\Quelle: \url{http://play-zone.ch}}
    \label{fig:logan-bild}
\end{figure}

Vorteilhaft ist die f�r alle g�ngigen Betriebssysteme verf�gbare Software, welche Analyzer f�r �bliche Protokolle (inklusive \gls{i2c}, \gls{spi}, \gls{can} und UART mit auto-bauding) bietet
und einen Export in das \gls{csv}-Format, sowie das simulator�bliche \gls{vcd}-Format erm�glicht.

Es ist anzumerken, dass die Hardware nahezu baugleich mit 8051-basierten Analyzern wie dem (teureren) \emph{USBee SX} ist und dass viele weitaus kosteng�nstigere Kopien der Hardware
existieren. Anleitungen hierzu finden sich im Netz\footnote{Etwa unter \url{http://www.jwandrews.co.uk/2011/12/saleae-logic-analyser-clone-teardown-and-reprogramming/}.}.



