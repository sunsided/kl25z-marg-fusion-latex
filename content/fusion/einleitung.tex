\section{Einleitung}

Das Problem der Orientierungserkennung besteht darin, dass jeder der üblicherweise verwendeten Sensoren nur einen Aspekt der Orientierungsfrage beantworten kann. So liefern
Accelerometer (neben der Beschleunigung) nur eine Information über die Neigung des Sensors gegenüber der horizontalen ($X\mathrm{-}Y\mathrm{-}$)Ebene, Magnetometer (neben
der Magnetfeldstärke) im besten Fall nur eine Abweichung von der Nordrichtung und Gyrosensoren nur eine Veränderung der Rotationsgeschwindigkeit um den Mittelpunkt des Sensors;
Einen echten "`Orientierungssensor"' gibt es nicht. Um die Orientierung eines Systemes dennoch zu bestimmen, werden die Informationen verschiedener Sensoren kombiniert -- fusioniert --
um so ein Gesamtbild zu erhalten.

Dass dabei Informationen redundant vorliegen, ist ein erwünschter Nebeneffekt zur Erhöhung der Verlässlichkeit: So kann die Orientierung etwa aus Accelerometer und Magnetometer
gewonnen werden, jedoch auch durch Integration der Winkelveränderungen aus dem Gyrosensor; Magnetometer und Accelerometer teilen sich jeweils eine Achse, etc.

Spricht man über Orientierungen, ist es naheliegend, diese in Form von \textsc{Euler}'schen Winkeln zu beschreiben, da diese eine intuitive Beschreibung der Lage ermöglichen. 
Wieso dies problembehaftet ist und welche Methoden man (alternativ und ergänzend) verwendet, soll im folgenden kurz umrissen werden.

\subsection{Der "`Gimbal Lock"' -- Singularität \textsc{Euler}'scher Winkel}

Allen Fusionsalgorithmen gemein ist die Problematik, dass die Verwendung \textsc{Euler}'scher Winkel --- trotz oder eben gerade wegen ihrer Einfachheit --- zu 
nicht auflösbaren Singularitäten führt. Dieses Phänomen wird \gls{Gimbal Lock} genannt, nach dem englischen Namens des kardanischen Kreisel (siehe Abbildung~\ref{fig:kardanischer_kreisel}),
bei dem dieser Effekt deutlich wird.

\begin{figure}[htbp]
	\centering
	\includegraphics[width=0.3\textwidth]{./images/kardanischer_kreisel.png}
	\caption[Kardanischer Kreisel]{Darstellung eines kardanischen Kreisels.\\Quelle: Prof. Dr. G. Flügge: Experimentalphysik I: Mechanik, RHTW Aachen}
	\label{fig:kardanischer_kreisel}
\end{figure}


Abhängig von der Wahl des Rotationsmodells etwa als Euler\footnote{Rotationsfolgen $Z\text{--}X\text{--}Z$, $X\text{--}Y\text{--}X$, $Y\text{--}Z\text{--}Y$, 
$Z\text{--}Y\text{--}Z$, $X\text{--}Z\text{--}X$ und $Y\text{--}X\text{--}Y$}- oder der in der Technik (vgl. DIN 9300, DIN 70000) üblicheren 
Tait-Bryan-Folge\footnote{Rotationsfolgen $X\text{--}Y\text{--}Z$, $Y\text{--}Z\text{--}X$, $Z\text{--}X\text{--}Y$, 
$X\text{--}Z\text{--}Y$, $Z\text{--}Y\text{--}X$ und $Y\text{--}X\text{--}Z$}, existieren hierbei unterschiedliche Rotationskonstellationen
mit nicht eindeutig bestimmbaren Winkeln. Dies ist im Falle des kardanischen Kreisels immer dann der Fall, wenn sich zwei direkt nebeneinander liegende Ringe  
durch geeignete Wahl der Rotationen in derselben Ebene befinden, wodurch eine beliebige Rotation des äußeren der beiden Ringe auch immer eine --- d.h. dieselbe! --- Rotation des inneren Ringes
bewirkt, was dem Verlust eines Freiheitsgrades des Systems entspricht (vgl. Abbildung~\ref{fig:gimbal-lock-full}). In solchen Fällen ist es oft nur noch möglich, einen einzelnen Winkel 
(etwa eine Neigung $\theta$) exakt anzugeben, wobei für die verbleibenden Winkel lediglich eine Bedingung der Art $\left|\phi + \psi \right| \leq 180^\circ$ angegeben werden kann.

Die Auswirkungen dieses Effektes in der Praxis sind in Abbildung~\ref{fig:euler_extrakt} deutlich ersichtlich. Hierbei wurde das System
über eine Zeit von ca. 40 Sekunden zweimal um jede seiner Achsen rotiert, wobei bei vorausgesetztem $Z\text{--}Y\text{--}X$-System\footnote{"`yaw-pitch-roll"'} 
ein Nickwinkel (Pitch, Elevation) von $\theta = \pm 90^\circ$ zu einer Singularität auf den Achsen $\phi$ (Yaw, Azimuth) und $\psi$ (Roll) führte.

\subsection{Ansätze zur Sensorfusion}

Verschiedene Ansätze zur Orientierungserkennung unter Vermeidung \textsc{Euler}'scher Winkel werden daher in der Literatur behandelt. 
\cite{madgwick_quat}, \cite{indirect_quaternion_kalman} und \cite{vecops} formulieren \gls{Quaternion}-basierte \gls{marg}-Fusionsalgorithmen.
In der Quadrocopter-Bastlerszene großer Beliebtheit erfreut sich aufgrund seiner geringen Komplexität der von \cite{dcmdraft} vorgeschlagene 
Fusionsalgorithmus auf Basis der Winkelkosinusmatrix (\gls{dcm}) in Kombination mit einem \gls{Komplementaerfilter}, wie er von \cite{mahony_comp_eucl}, 
\cite{mahony_coupled} und \cite{mahony_compl} beschrieben wird. Das \gls{Komplementaerfilter} wird als sich im Frequenzband komplementierende Zusammenstellung von Hoch- und 
Tiefpassfiltern implementiert, welche -- im vorliegenden Szenario -- die Vorteile der hohen Genauigkeit des Gyrosensors (ohne dessen Drift) mit jenen der hohen Dynamik des 
Accelerometers (ohne dessen Streuung) kombinieren, wobei jedoch die Signal- und Prozessrauscheigenschaften nur statisch betrachtet werden können.
\cite{comp_comp_kal} stellt jene Filter dem \gls{Kalman-Filter} gegenüber.

Untersuchungen finden sich weiterhin zu den sog. Erweiterten \glslink{Kalman-Filter}{Kalman-Filtern} (z.B. \citealp{yadlin09} und \citealp{indirect_quaternion_kalman}), 
sowie den Unscented \glslink{Kalman-Filter}{Kalman-Filtern}\footnote{Einer Variation des erweiterten \glslink{Kalman-Filter}{Kalman-Filters} unter Zuhilfenahme der 
Unscented-Transformation, wodurch die Notwendigkeit der Linearisierung des Systems vermieden wird} (etwa \citealp{unscented}). Ferner existieren adaptive 
Varianten des \glslink{Kalman-Filter}{Kalman-Filters}, wie etwa von \cite{liwang12} untersucht.
Diese Arten von Filtern versuchen die nichtlinearen Zusammenhänge des Systems --- allerdings auf Kosten der Komplexität des Filters --- direkt abzubilden. 

\cite{orientation_dcm} vergleichen eine \gls{Quaternion}-Methode in einem erweiterten \gls{Kalman-Filter}, sowie die (bekanntermaßen problembehaftete) direkte Bestimmung der \textsc{Euler}'schen Winkel
mit einem Entwurf zur indirekten Schätzung der \gls{dcm} in einem regulären \gls{Kalman-Filter}. 
Er zeigt auf, dass die Verwendung des regulären Filters zur indirekten Schätzung trotz der Nichtlinearitäten keine inhärenten Nachteile mit sich bringt, 
sondern im Gegenteil mit der \gls{Quaternion}-Methode konkurrieren kann, weswegen dieser Algorithmus im Rahmen des Projektes umgesetzt wurde.
\cite{Tsang} und \cite{Filieri} beschreiben in diesem Zusammenhang ein Fehlermodell zur Schätzung der Beschleunigungs- und Integrationsunsicherheiten, welches unterstützend auch 
zur direkten Schätzung der Sensordrift angewendet werden kann. Dieser Aspekt wurde in MATLAB (erfolgreich) untersucht, jedoch nicht in der Firmware implementiert.

\begin{figure}[htbp]
	\centering
	\includegraphics[width=\textwidth]{./images/matlab/rollpitchyaw45-2.png}
	\caption[Extraktion der \textsc{Euler}'schen Winkel]{Extraktion der \textsc{Euler}'schen Winkel. Deutlich zu erkennen sind die Singularitäten bei Sekunden 13, 17 und 29 und 32.}
	\label{fig:euler_extrakt}
\end{figure}
