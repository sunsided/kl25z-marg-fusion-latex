\setchapterpreamble[u]{%
\dictum[Luhmann]{Die Klassiker sind Klassiker, weil sie Klassiker sind \dots}}
\chapter{Fusionsalgorithmus}

Verschiedene Ansätze zur Orientierungserkennung werden in der Literatur beschrieben. 
\cite{madgwick_quat} formuliert einen \gls{Quaternion}-basierten \gls{marg}-Fusionsalgorithmus auf Basis eines Gradientenabstiegsverfahrens.
In der Quadrocopter-Bastlerszene großer Beliebtheit erfreut sich aufgrund seiner geringen Komplexität der von \cite{dcmdraft} vorgeschlagene 
Fusionsalgorithmus auf Basis der Winkelkosinusmatrix (\gls{dcm}) in Kombination mit einem \gls{Komplementaerfilter}, wie er von \cite{mahony_comp_eucl}, 
\cite{mahony_coupled} und \cite{mahony_compl} beschrieben wird.

Viele Untersuchungen wurden zu erweiterten \glslink{Kalman-Filter}{Kalman-Filtern}, sowie unscented \glslink{Kalman-Filter}{Kalman-Filtern} vorgenommen,
welche die nichtlinearen Zusammenhänge der Sensoren direkt abzubilden versuchen. \todo{Quellen für diesen Unfug}

\cite{Tsang} beschreibt einen Filter auf Basis des regulären \gls{Kalman-Filter} zur direkten Schätzung der \gls{dcm} anstelle des sonst üblichen Verfahrens
zur Schätzung der \textsc{Euler}'schen Winkel. \todo{Dies wegen Singularitäten}

\cite{Filieri} und, außerdem . Dort werden \glspl{mems} erwähnt. So ein \gls{mems} ist toll.

\begin{figure}[htbp]
	\centering
	\includegraphics[width=\textwidth]{./images/matlab/rollpitchyaw45-2.png}
	\caption[Extraktion der \textsc{Euler}'schen Winkel]{Extraktion der \textsc{Euler}'schen Winkel. Deutlich zu erkennen sind die Singularitäten bei Sekunden 13, 17 und 29 und 32.}
\end{figure}
