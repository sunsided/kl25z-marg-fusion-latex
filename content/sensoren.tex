\chapter{Sensoren}

Alle in diesem Projekt verwendeten Sensoren verfügen über eine \gls{i2c}-Schnittstelle, über welche die Kommunikation abgewickelt wurde. Einige Module, wie die MPU6050 (siehe Abschnitt~\ref{sec:mpu6050})
verfügen darüberhinaus über Interrupt-Leitungen, welche das Vorliegen neuer Messwerte signalisieren.

\section{MMA8451Q - Accelerometer}

Der Freescale MMA8451Q ist der auf dem Freedom-Board verbaute Beschleunigungssensor. Er bietet eine Auflösung von X bei einer maximalen Samplingrate von Y. 

\subsection{Treiber}

\section{MPU6050 - Accelerometer und Gyrosensor}
\label{sec:mpu6050}

Die Invensense MPU6050 ist eine \gls{imu}, die einen Beschleunigungs- und einen Drehratensensor vereinigt. Diese \gls{mpu} umfasst sowohl einen Beschleunigungs-, als auch einen Drehratensensor und bietet die Möglichkeit, über einen als slave betriebenen, zusätzlichen \gls{i2c}-Bus weitere Sensoren zu verarbeiten, wodurch Aufwand vom Hauptcontroller abfällt. Der \gls{mpu} zugeschaltet ist ein sog. \gls{dmp}, welcher die internen Daten, sowie die über den slave-Bus bezogenen externen Daten fusionieren kann. Während die Ergebnisregister des \gls{dmp} frei zugänglich sind, sind die Inhalte dieser Felder nicht dokumentiert. Der Zugriff auf sie erfolgt stattdessen --- sofern erwünscht --- über eine proprietäre MotionsApps-Firmware\footnote{\url{http://www.invensense.com/developers/forum/viewtopic.php?f=3&t=142}}.

\subsection{Treiber}

\subsection{Lessons Learned}

Bei der Kommunikation mit der MPU6050 trat das Problem auf, dass bei aktiviertem Interrupt-Signal und reduzierter Samplingrate nach einem Kaltstart keinerlei nennenswerte Verzögerung (d.h. nur wenige Millisekunden \todo{Verifizieren}) zwischen aufienander folgenden Interruptsignalen festzustellen war. 
Dies führte zu der Problematik, dass die durch den Interrupt-Handler freigeschaltete Routine zum Beziehen der Sensordaten über \gls{i2c} und die anschließende Verarbeitung direkt durch einen erneuten Interrupt unterbrochen wurde. Da die serielle Ausgabe der Werte nur mit deutlich geringerer Taktung laufen kann, führte dies zu einem Engpass am vorgeschalteten Ringpuffer, wodurch die Verarbeitung deutlich inperformant wurde.

Wurde das System jedoch durch einen Reset (d.h. durch einen Warmstart) neu initialisiert, verlief die Kommunikation wie gewünscht. Durch einen weiteren Kaltstart konnte der Zustand erneut herbeigeführt werden.

\todo{Bild mit Freifeuer vom Logan}

Dieses Problem war erst dadurch zu beheben, dass zu Beginn der Konfiguration der \gls{imu} die interne Takteinheit deaktiviert wurde, um sie dann im Zuge der folgenden Konfiguration erneut auf den gewünschten Betriebsmodus zu stellen.

\todo{Bild ohne Freifeuer vom Logan}

\section{HMC5883L - Magnetometer}

Der HMC5883L von Honeywell liefert bei einer Auflösung von 12 Bit (inkl. Vorzeichen) bei einem Wertebereich von $\pm$ 8 Gauss, wodurch er für den Einsatz in Präsenz starker lokaler Magnetfelder geeignet ist.

\subsection{Treiber}

\subsection{Lessons Learned}

Bei der Auswertung der Sensordaten des HMC5883L traten zwei kleine bis mittlere Komplikationen auf. Eines dieser Probleme lag in der Ansteuerung des Sensors, 
das andere dagegen in der Interpretation der Messwerte begründet.

\subsubsection{Reihenfolge der Messwerte}

Während in sämtlichen anderen Sensoren die Messwerte in üblicher $X$,$Y$,$Z$-Reihenfolge vorlagen, speichert der HMC5883L die Messwerte in der Reihenfolge $X$,$Z$,$Y$. Dies ist aus dem Datenblatt (\citealp{hmc5883l}) nur
dann ersichtlich, wenn die im Datenblatt vorliegende --- sehr kurze --- Registerliste (S. 11) sehr genau beachtet wird. Im Gegensatz zu der dort aufgeführten Reihenfolge, verweisen sämtlichen
anderen Stellen des Datenblattes (insbesondere im Abschnitt Data Output Registers, S. 15) auf die reguläre Anordnung.

\subsubsection{Interpretation der Messergebnisse}

\begin{figure}[htbp]
		\centering
		\begin{subfigure}[b]{0.5\textwidth}
			\centering
			\includegraphics[width=\textwidth]{./images/earth_magnetic_field_poles_shutterstock.jpg}
			\caption{Populistische Darstellung \\ Quelle: Shutterstock}
		\end{subfigure}%
		~
		\begin{subfigure}[b]{0.5\textwidth}
			\centering
			\includegraphics[width=\textwidth]{./images/Dipole_field_wikibooks.jpg}
			\caption{Vereinfachte realistische Darstellung \\ Quelle: Wikibooks}
		\end{subfigure}%
		\caption[Darstellungen des Erdmagnetfeldes]{Darstellungen des Erdmagnetfeldes.\\Links: Die Feldlinien treffen sich an den Polen.\\Rechts: Die Feldlinien durchdringen den Erdmantel.}
\end{figure}