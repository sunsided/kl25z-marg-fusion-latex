\chapter{Sensoren}

Alle in diesem Projekt verwendeten Sensoren verfügen über eine \gls{i2c}-Schnittstelle, über welche die Kommunikation abgewickelt wurde. Einige Module, wie die MPU6050 (siehe Abschnitt~\ref{sec:mpu6050})
verfügen darüberhinaus über Interrupt-Leitungen, welche das Vorliegen neuer Messwerte signalisieren.

\section{MMA8451Q - Accelerometer}

Der Freescale MMA8451Q ist der auf dem Freedom-Board verbaute Beschleunigungssensor. Er bietet eine Auflösung von X bei einer maximalen Samplingrate von Y. 

\subsection{Treiber}

\section{MPU6050 - Accelerometer und Gyrosensor}
\label{sec:mpu6050}

Der Invensense MPU6050 ist eine \gls{imu}, die einen Beschleunigungs- und einen Drehratensensor vereinigt.

\subsection{Treiber}

\section{HMC5883L - Magnetometer}

Der HMC5883L von Honeywell liefert bei einer Auflösung von 12 Bit (inkl. Vorzeichen) bei einem Wertebereich von $\pm$ 8 Gauss, wodurch er für den Einsatz in Präsenz starker lokaler Magnetfelder geeignet ist.

\subsection{Treiber}

\subsection{Lessons Learned}

Bei der Auswertung der Sensordaten des HMC5883L traten zwei kleine bis mittlere Komplikationen auf. Eines dieser Probleme lag in der Ansteuerung des Sensors, 
das andere dagegen in der Interpretation der Messwerte begründet.

\subsubsection{Reihenfolge der Messwerte}

Während in sämtlichen anderen Sensoren die Messwerte in üblicher $X$,$Y$,$Z$-Reihenfolge vorlagen, speichert der HMC5883L die Messwerte in der Reihenfolge $X$,$Z$,$Y$. Dies ist aus dem Datenblatt nur
dann ersichtlich, wenn die im Datenblatt auf Seite 11 vorliegende --- sehr kurze --- Registerliste (Form no. 900405, Revision E) genau beachtet wird. Das Problem entstand hierbei daraus, dass in sämtlichen
anderen Stellen des Datenblattes (insbesondere im Absatz Data Output Registers, Seite 15) die reguläre Reihenfolge beschrieben wurde.

