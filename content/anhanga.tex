\chapter{Schluss}

Für den Schluss ist zu überlegen, wie man den Anhang formatiert haben möchte: Das KOMA-Script kennt den Befehl \verb#\backmatter#. Hierdurch wird die Nummerierung der Gliederungseinheiten im Text und im Inhaltsverzeichnis unterdrückt. Erwartet man die übliche Beschriftung mit "`Anhang A"' bzw. "`A."' so verwendet man den Befehl \verb#\appendix# und verzichte auf \verb#\backmatter# oder setze es zu einem späteren Punkt ein.

Viel Spaß! Für Rückfragen, die diese Vorlage betreffen, stehe ich Ihnen gern in der Mailingliste von TXC zur Verfügung. Ansonsten sind die Dokumente \texttt{lshort}, \texttt{l2tabu}, die \texttt{FAQ der Newsgroup de.text.tex} und natürlich der \texttt{scrguide} immer sehr hilfreich.
