\setchapterpreamble[u]{%
\dictum[Luhmann]{Die Klassiker sind Klassiker, weil sie Klassiker sind \dots}}
\chapter{Diktum}

Übrigens wird ohne weitere Angaben ein Drittel der aktuellen Textbreite verwendet. Wie fast alles bei der Verwendung von \LaTeX , kann dies natürlich angepasst werden. Wie das geht und auch alles andere zur Verwendung von Präambeln steht im scrguide 3.\,6.\,2.

Exemplarischer Verweis auf ein Buch mit dieser Aussage, vgl. \cite{Filieri}, außerdem \cite{Tsang}. Dort werden \glspl{mems} erwähnt. So ein \gls{mems} ist toll.