\addto{\captionsngerman}{
  \renewcommand{\acronymname}{Abkürzungen}
}

\addto{\captionsngerman}{
  \renewcommand{\glossaryname}{Glossar}
}

\newglossaryentry{Cortex}{name = {Cortex}, description={32bit-Mikrocontrollerarchitektur von ARM, Inc}}
\newglossaryentry{cortex-m0}{name = {Cortex-M0}, description={\gls{mcu} der M0-Serie der ARM \gls{Cortex}-Architektur}}
\newglossaryentry{cortex-m0+}{name = {Cortex-M0+}, description={\gls{mcu} der M0+-Serie der ARM \gls{Cortex}-Architektur}}
\newglossaryentry{CoreSight}{name = {CoreSight}, description={Debug-Schnittstelle in \gls{Cortex}-\glspl{mcu}}}

\newglossaryentry{Kinetis}{name = {Kinetis}, description={32bit-Mikrocontrollertyp von Freescale Semiconductor}}
\newglossaryentry{kl25z}{name = {KL25Z}, description={Mikrocontroller der \gls{Kinetis}-L-Serie von Freescale auf Basis des ARM Cortex-M0+}}
\newglossaryentry{frdm-kl25z}{name = {FRDM-KL25Z}, description={Freescale Freedom Development Board mit KL25Z-\gls{mcu}}}

\newglossaryentry{Gimbal Lock}{name = {Gimbal Lock}, description={Kardanische Blockade eines Systemes bei der Verwendung von Euler'schen Winkeln, die bei ungünstiger Kombination von Rotationen zum Verlust eines Freiheitsgrades führt}}

\newglossaryentry{Kalman-Filter}{name = {Kalman-Filter}, description={Rekursives, lineares Filter zur Schätzung stochastischer Systemparameter, dessen Entwicklung auf Rudolf Emil Kálmán zurückgeht}}
\newglossaryentry{OpenSDA}{name = {OpenSDA}, description={Proprietäre, erweiterbare Programmierschnittstelle von Freescale}}
\newglossaryentry{OpenOCD}{name = {OpenOCD}, description={Quelloffene Programmierschnittstelle}}
\newglossaryentry{JTAG}{name = {JTAG}, description={Schnittstelle für Tests von Controllerschnittstellen und Programmierung}}
\newglossaryentry{elf}{name = {ELF}, description={Executable and Linking Format, eine ausführbare Datei}}
\newglossaryentry{mbed}{name = {mbed}, description={Plattform für die Entwicklung auf Cortex-M-MCUs}}
\newglossaryentry{i2c}{name = {I2C}, description={(auch IIC) Inter-Integrated Circuit. Ein von Philips entworfenes Bussystem für Inter-Chip-Kommunikation.}}

% http://tex.stackexchange.com/questions/8946/how-to-combine-acronym-and-glossary
%\newglossaryentry{i2cg}{
%	name=\glslink{i2c}{Application Programming Interface (\gls{i2c})},
%	description={Application Programming Interface Desc}
%}

% http://tex.stackexchange.com/questions/8946/how-to-combine-acronym-and-glossary
%\newglossaryentry{i2c}{
%	type=\acronymtype,
%	name=I2C,
%	first=Inter-Integrated Circuit (I2C),
%	see=[Glossary:]{\gls{i2cg}}, 
%	description=\glslink{i2cg}{Inter-Integrated Circuit}
%}

%\newacronym{i2c}{I2C}{Inter-Integrated Circuit \glsadd{i2cg}}
%\newacronym{i2c}{I2C}{Inter-Integrated Circuit}

\newacronym[
	\glsshortpluralkey={MEMS},
	\glslongpluralkey={mikroelektromechanische Systeme}
]{mems}{MEMS}{mikroelektromechanisches System}

\newacronym{marg}{MARG}{Magnetic, Angular Rate and Gravitational}

\newacronym[
	\glsshortpluralkey={IMUs},
	\glslongpluralkey={Inertial Measurement Units}
]{imu}{IMU}{Inertial Measurement Unit}

\newacronym{hci}{HCI}{Human-Computer Interaction}
\newacronym{hid}{HID}{Human Interface Device}
\newacronym{cmsis}{CMSIS}{Cortex Microcontroller Software Interface Standard}
\newacronym{cmsis-dap}{CMSIS-DAP}{\gls{cmsis} Debug Access Port}
\newacronym[
	\glsshortpluralkey={MCUs},
	\glslongpluralkey={Microcontroller Units}
]{mcu}{MCU}{Microcontroller Unit}
\newacronym{ocd}{OCD}{On-Chip-Debugger}
\newacronym{msd}{MSD}{Mass Storage Device}
\newacronym{usb}{USB}{Universal Serial Bus}
\newacronym{gdb}{GDB}{GNU Debugger}
\newacronym{swd}{SWD}{Serial Wire Debug}

\newacronym[
	\glsshortpluralkey={GPIOs},
	\glslongpluralkey={General-Purpose I/Os}
]{gpio}{GPIO}{General-Purpose I/O}
