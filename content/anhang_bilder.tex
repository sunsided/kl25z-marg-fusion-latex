\chapter{Weitere Abbildungen}
\label{chap:bilder}

\begin{figure}[htbp]
	\centering
	\includegraphics[width=0.8\textwidth]{./images/fused-csharp-video.jpg}
	\caption[Fusionierte Ausgabe mittels OpenGL]{Fusionierte Ausgabe mittels OpenGL mit aktiviertem Modus \emph{44} "`Quaternion und Roll-Pitch-Yaw"' (vgl. Seite~\pageref{subsubsec:quat-and-rpy}).}
	\label{fig:fused-csharp}
\end{figure}

\begin{figure}[htbp]
		\centering
			\begin{subfigure}[b]{\textwidth}
			\centering
			\includegraphics[width=0.8\textwidth]{./images/logan-full-sample.png}
			\caption{Übersicht der Kommunikation. Zu erkennen bei $1466$, $1479$ und $1495$ ms ist die timergesteuerte Abfrage an den HMC5883L. Deutlich ersichtlich bei $1467$ ms auch die adaptive Länge des Filter-Korrekturschrittes.}
			\label{fig:logan-full}
		\end{subfigure}

	\begin{subfigure}[b]{\textwidth}
			\centering
			\includegraphics[width=0.8\textwidth]{./images/logan-irq-scl-sda-tx.png}
			\caption{Detail der Kommunikation. Rot und orange: \gls{i2c}-Transfer zur MPU6050. Der Lesezugriff auf Register \texttt{0x3A} löscht den IRQ (Data Ready). Gelb: Serielle Übertragung im Modus \emph{44} (\texttt{0x2C} --- vgl. Seite~\pageref{subsubsec:quat-and-rpy})}
			\label{fig:logan-detail}
	\end{subfigure}
	
	\caption[Mitschnitt der Kommunikation durch den Logic Analyzer]{Mitschnitt der Kommunikation durch den Logic Analyzer.\\Lila (7): Data Ready-Signal von der MPU6050 (low-aktiv). Rot (2) und orange (3): \gls{i2c}-Transfer zu MPU6050 und HMC5883L. Schwarz (0) und braun (1): Prädiktions- und Korrekturphase des Filters. Gelb (4): UART-Sendetransfer mit 115200 baud.}
\end{figure}

\begin{figure}[htbp]
		\centering
		\includegraphics[width=0.8\textwidth]{./images/bit_banding.png}
    \caption[Bit-Banding]{Bit-Banding im \gls{Cortex}-M3\\Quelle: ARM}
    \label{fig:bit-banding}
\end{figure}

\begin{figure}[htbp]
	\centering
	\includegraphics[width=0.8\textwidth]{./images/gimbal-lock.jpg}
	\caption[Kardanischer Kreisel mit Gimbal Lock]{Kardanischer Kreisel mit \gls{Gimbal Lock} (rechts): Rotation ist nur noch in zwei Richtungen möglich, ein Freiheitsgrad ist verloren.\\Quelle: \url{around-the-corner.typepad.com}}
	\label{fig:gimbal-lock-full}
\end{figure}

\begin{figure}[htbp]
	\centering
	\includegraphics[width=\textwidth]{./images/kl25z-clocking.png}
	\caption[KL25Z: Übersicht der Clock-Einheit]{KL25Z: Übersicht der Clock-Einheit (\citealp{kl25z_sfrm})}
	\label{fig:kl25z-clocking}
\end{figure}

\begin{figure}[htbp]
	\centering
	\includegraphics[width=\textwidth]{./images/kl25z-mcg.png}
	\caption[KL25Z: Übersicht des MCG]{KL25Z: Übersicht des \gls{mcg} (\citealp{kl25z_sfrm})}
	\label{fig:kl25z-mcg}
\end{figure}

\begin{figure}[htbp]
	\centering
	\includegraphics[width=\textwidth]{./images/kl25z-mcg-states.png}
	\caption[KL25Z: Übersicht des MCG]{KL25Z: Zustandsdiagramm des \gls{mcg} (\citealp{kl25z_sfrm})\\\gls{fei}: \gls{fll} Engaged Internal\\FEE: \gls{fll} Engaged External\\FBI: \gls{fll} Bypassed Internal\\FBE: \gls{fll} Bypassed External\\\gls{pee}: \gls{pll} Engaged External\\PBE: \gls{pll} Bypassed External\\BLPI: Bypassed Low Power Internal\\BLPE: Bypassed Low Power External}
	\label{fig:kl25z-mcg-states}
\end{figure}
