\chapter{Weitere Abbildungen}
\label{chap:bilder}

\begin{figure}[htbp]
	\centering
	\includegraphics[width=0.8\textwidth]{./images/gimbal-lock.jpg}
	\caption[Kardanischer Kreisel mit Gimbal Lock]{Kardanischer Kreisel mit \gls{Gimbal Lock} (rechts): Rotation ist nur noch in zwei Richtungen (yaw=roll und pitch) möglich, ein Freiheitsgrad ist verloren.\\Quelle: \url{around-the-corner.typepad.com}}
	\label{fig:gimbal-lock-full}
\end{figure}

\begin{figure}[htbp]
		\centering
		\begin{subfigure}[b]{0.5\textwidth}
			\centering
			\includegraphics[width=\textwidth]{./images/earth_magnetic_field_poles_shutterstock.jpg}
			\caption{Populistische Darstellung \\ Quelle: Shutterstock}
		\end{subfigure}%
		~
		\begin{subfigure}[b]{0.5\textwidth}
			\centering
			\includegraphics[width=\textwidth]{./images/Dipole_field_wikibooks.jpg}
			\caption{Vereinfachte realistische Darstellung \\ Quelle: Wikibooks}
		\end{subfigure}%
		\caption[Darstellungen des Erdmagnetfeldes]{Darstellungen des Erdmagnetfeldes.\\Links: Die Feldlinien treffen sich an den Polen.\\Rechts: Die Feldlinien durchdringen den Erdmantel.}
		\label{fig:magnetfeld_erde_hmpf}
\end{figure}

\begin{figure}[htbp]
	\centering
	\includegraphics[width=0.8\textwidth]{./images/kl26z.jpg}
	\caption[Übersicht FRDM-KL26Z]{Übersicht über das Freescale FRDM-KL26Z. Zu beachten: 18 Pins sind nicht verbunden.\\Quelle: \cite{kl26z_qsg}}
	\label{fig:kl26z}
\end{figure}

\begin{figure}[htbp]
		\centering
		\includegraphics[width=\textwidth]{./images/visualgdb_project.png}
    \caption{VisualGDB: Auswahl des Controllers}
    \label{fig:visualgdb_controller}
\end{figure}

\begin{figure}[htbp]
		\centering
		\includegraphics[width=\textwidth]{./images/visualgdb_programmer.png}
    \caption{VisualGDB: Auswahl des Programmers}
    \label{fig:visualgdb_programmer}
\end{figure}

\begin{figure}[htbp]
		\centering
		\includegraphics[width=\textwidth]{./images/visualgd_makefile.png}
    \caption{VisualGDB: Make-Optionen}
    \label{fig:visualgd_makefile}
\end{figure}

\begin{figure}[htbp]
		\centering
			\begin{subfigure}[b]{\textwidth}
			\centering
			\includegraphics[width=\textwidth]{./images/sensors/gyro_unsaturated_2000dgs.png}
			\caption{Auflösung von $200^\circ/\second$. Sensordaten wie erwartet.}
			\label{fig:gyro-unsaturated}
		\end{subfigure}

	\begin{subfigure}[b]{\textwidth}
			\centering
			\includegraphics[width=\textwidth]{./images/sensors/gyro_saturated_250dgs.png}
			\caption{Auflösung von $25^\circ/\second$. Sättigung wird erreicht, integrierter Winkel fällt zu gering aus.}
			\label{fig:gyro-saturated}
	\end{subfigure}
	
	\caption[Gyrosensor: Betrieb im Sättigungsbereich]{Gyrosensor bei Normalbetrieb und im Sättigungsbereich bei vergleichbarer Rotation. Darstellung von integriertem Winkel (blau, linke Achse) und Winkelgeschwindigkeit (grün, rechte Achse). Da die Rotationen manuell durchgeführt wurden, sind sie nicht exakt deckungsgleich.}
	\label{fig:gyro-sat-unsat}
\end{figure}

\begin{figure}[htbp]
		\centering
			\begin{subfigure}[b]{\textwidth}
			\centering
			\includegraphics[width=\textwidth]{./images/logan-full-sample.png}
			\caption{Übersicht der Kommunikation. Zu erkennen bei $1466$, $1479$ und $1495$ ms ist die timergesteuerte Abfrage an den HMC5883L. Deutlich ersichtlich bei $1467$ ms auch die adaptive Länge des Filter-Korrekturschrittes.}
			\label{fig:logan-full}
		\end{subfigure}

	\begin{subfigure}[b]{\textwidth}
			\centering
			\includegraphics[width=\textwidth]{./images/logan-irq-scl-sda-tx.png}
			\caption{Detail der Kommunikation. Rot und orange: \gls{i2c}-Transfer zur MPU6050. Der Lesezugriff auf Register \texttt{0x3A} löscht den IRQ (Data Ready). Gelb: Serielle Übertragung im Modus \emph{44} (\texttt{0x2C} --- vgl. Seite~\pageref{subsubsec:quat-and-rpy})}
			\label{fig:logan-detail}
	\end{subfigure}
	
	\caption[Mitschnitt der Kommunikation durch den Logic Analyzer]{Mitschnitt der Kommunikation durch den Logic Analyzer.\\Lila (7): Data Ready-Signal von der MPU6050 (low-aktiv). Rot (2) und orange (3): \gls{i2c}-Transfer zu MPU6050 und HMC5883L. Schwarz (0) und braun (1): Prädiktions- und Korrekturphase des Filters. Gelb (4): UART-Sendetransfer mit 115200 baud.}
\end{figure}

\begin{figure}[htbp]
		\centering
		\includegraphics[width=0.8\textwidth]{./images/bit_banding.png}
    \caption[Bit-Banding]{Bit-Banding im \gls{Cortex}-M3\\Quelle: ARM}
    \label{fig:bit-banding}
\end{figure}

\begin{figure}[htbp]
	\centering
	\includegraphics[width=\textwidth]{./images/kl25z-clocking.png}
	\caption[KL25Z: Übersicht der Clock-Einheit]{KL25Z: Übersicht der Clock-Einheit (\citealp{kl25z_sfrm})}
	\label{fig:kl25z-clocking}
\end{figure}

\begin{figure}[htbp]
	\centering
	\includegraphics[width=\textwidth]{./images/kl25z-mcg.png}
	\caption[KL25Z: Übersicht des MCG]{KL25Z: Übersicht des \gls{mcg} (\citealp{kl25z_sfrm})}
	\label{fig:kl25z-mcg}
\end{figure}

\begin{figure}[htbp]
	\centering
	\includegraphics[width=\textwidth]{./images/kl25z-mcg-states.png}
	\caption[KL25Z: Übersicht des MCG]{KL25Z: Zustandsdiagramm des \gls{mcg} (\citealp{kl25z_sfrm})\\\gls{fei}: \gls{fll} Engaged Internal\\FEE: \gls{fll} Engaged External\\FBI: \gls{fll} Bypassed Internal\\FBE: \gls{fll} Bypassed External\\\gls{pee}: \gls{pll} Engaged External\\PBE: \gls{pll} Bypassed External\\BLPI: Bypassed Low Power Internal\\BLPE: Bypassed Low Power External}
	\label{fig:kl25z-mcg-states}
\end{figure}
