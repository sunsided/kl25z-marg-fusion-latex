\chapter{Präambeln}

Durch den Befehl \verb#\setpartpreamble[Position][Breite]{Präambel}# wird zusammen mit der Angabe des Teils (part) zudem der angegebene Text gesetzt. Dies kann z.\,B. eine kurze Inhaltsangabe sein. Ein Beipiel ist unter Hauptteil zu sehen.  Die Präambel wird in eine Box gesetzt, deren Position und Breite angegeben werden kann. Unterbleibt dies, wird sie unterhalb der Überschriften im normalen Blocksatz über den gesamten Textbereich gesetzt.

Eine ähnliche Funktion findet sich auch für Kapitel (chapter). Die Anweisung lautet hier entsprechend \verb#\setchapterpreamble[Position][Breite]{Präambel}#. 

Für ein einleitendes Zitat, ein sog. Diktum bietet das KOMA-Script die Anweisung \verb#\dictum[Urheber]{Spruch}#. Sie wird in der Regel in eine \verb#\setchapterpreamble# oder \verb#\setpartpreamble# gesetzt. Ein Beispiel soll folgen: