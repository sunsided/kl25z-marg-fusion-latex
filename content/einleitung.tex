\chapter{Einleitung}

Ein häufiges Problem in der mobilen Robotik und eine zunehmende Notwendigkeit in der \gls{hci} --- etwa bei der Entwicklung 
von modernen Controllern für Unterhaltungssysteme --- ist die robuste Erkennung der Orientierung eines
beweglichen Systems, sei es einer autonomen Plattform (z.B. Roboter, Quadrocopter, ...) oder eines Eingabegerätes
(z.B. Nintendo Wiimote).

Während hochpräzise Inertialsensoren (\gls{imu}) wie Gyroskope in ihrer ursprünglichen Bauart rein mechanische Systeme sind, halten seit
einigen Jahren zunehmend \glspl{mems} Einzug in die Sensortechnik und stehen dabei in unterschiedlichen Genauigkeits-
und Kostengraden sowohl für Low-End, als auch High-End-Anwendungen zur Verfügung.

Hierbei werden grundlegend zwei Typen von Sensoren unterschieden: Inertialsensoren (\glspl{imu}), welche auf der Messung der Trägheit
des Systems beruhen --- klassische Vertreter sind der Accelerometer zur Messung von Beschleunigungen, 
sowie der Gyrosensor zur Messung von Drehraten --- als auch magnetische Sensoren, welche die Auswirkungen eines Magnetfeldes auf das 
System messen. Werden solche drei Sensoren in einem System kombiniert, spricht man von einem sogenannten \gls{marg}-Sensorsystem.

Dieses Projekt beschäftigt sich mit der Fusionierung eines \gls{marg}-Sensorsystems zur robusten Orientierungserkennung. Ziel ist
es, ein Maß für die Ausrichtung des Systemes im Raum zu ermitteln, welches

\begin{itemize}
	\item unanfällig gegenüber Messrauschen der Sensoren,
	\item stabil gegenüber externen Beschleunigungen und
	\item frei von Singulatitäten (\gls{Gimbal Lock}) ist.
\end{itemize}

Hierbei wird ein reguläres \gls{Kalman-Filter} zum Einsatz auf einem \gls{cortex-m0} entwickelt, welches adaptiv auf vorliegende
Messwerte reagiert und für den Echtzeiteinsatz geeignet ist. Die Implementierung erfolgt hierbei auf einem \gls{kl25z},
einem Mikrocontroller der Kinetis-Serie von Freescale auf Basis des ARM Cortex-M0+.

\begin{figure}[htbp]
		\centering
	\begin{subfigure}[b]{\textwidth}
		\centering
		\includegraphics[width=0.8\textwidth]{./images/board.jpg}
		\caption[FRDM-KL25Z mit externen Sensoren]{FRDM-KL25Z mit externen Sensoren: MPU6050 (blau) und HMC5883L (rot)}
		\label{fig:board}
	\end{subfigure}

	\begin{subfigure}[b]{\textwidth}
		\centering
		\includegraphics[width=0.8\textwidth]{./images/board-mit-logan.jpg}
		\caption[FRDM-KL25Z mit angeschlossenem Logic-Analyzer]{FRDM-KL25Z mit angeschlossenem Logic-Analyzer (rechts). Im Hintergrund links der verwendete UART-zu-USB-Transceiver.}
		\label{fig:board-logan}
	\end{subfigure}
	
	\caption{Prototyp auf dem Breadboard}
	\label{fig:prototype}
\end{figure}
