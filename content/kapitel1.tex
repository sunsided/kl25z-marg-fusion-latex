%\setpartpreamble[Position][Breite]{Text}
\setpartpreamble[u, r][6cm]{Dies ist ein Beispiel für eine Part-Präambel.}
\part[]{Hauptteil}
\chapter{Gliederung}

In den report- und book-Klassen steht, im Vergleich zu den article-Klassen als zusätzliche Gliederungseinheit \verb#\chapter[Kurzform]{Langform}# zur Verfügung. 

Kapitel beginnen in der Regel in Büchern auf einer ungeraden, d.\,h. rechten Seite. Will man fortlaufenden Textsatz erreichen und also den Beginn auch auf linken Seiten zulassen, verwendet man die Option \verb#openany# gleich in der Dokumenten-Präambel. Hier finden sich auch andere Optionen zur Regelung der Überschriftengröße und deren Beschriftung.