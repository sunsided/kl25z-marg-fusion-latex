%%%%%%%%%%%%%%%%%%%%%%%%%%%%%%%%%%%%%%%%%%%%%%%%%%%%%%%%%%%%%%%%%%%%%%%
%% Optionen zum Layout des Buchs                                     %%
%%%%%%%%%%%%%%%%%%%%%%%%%%%%%%%%%%%%%%%%%%%%%%%%%%%%%%%%%%%%%%%%%%%%%%%
\documentclass[
a4paper,							% alle weiteren Papierformat einstellbar
%landscape,						% Querformat
11pt,								  % Schriftgröße (12pt, 11pt (Standard))
%BCOR1cm,							% Bindekorrektur, bspw. 1 cm
%DIVcalc,							% führt die Satzspiegelberechnung neu aus
%											  s. scrguide 2.4
%oneside,							% einseitiges Layout
%twocolumn,						% zweispaltiger Satz
%openany,							% Kapitel können auch auf linken Seiten beginnen
%halfparskip*,				% Absatzformatierung s. scrguide 3.1
%headsepline,					% Trennline zum Seitenkopf	
%footsepline,					% Trennline zum Seitenfuß
%notitlepage,					% in-page-Titel, keine eigene Titelseite
bibliography=totoc,
chapterprefix,				% vor Kapitelüberschrift wird "Kapitel Nummer" gesetzt
appendixprefix,				% Anhang wird "Anhang" vor die Überschrift gesetzt 
%normalheadings,			% Überschriften etwas kleiner (smallheadings)
%leqno,								% Nummerierung von Gleichungen links
fleqn,								% Ausgabe von Gleichungen linksbündig
%draft								  % überlangen Zeilen in Ausgabe gekennzeichnet
]
{scrbook}

%\pagestyle{empty}		% keine Kopf und Fußzeile (k. Seitenzahl)
%\pagestyle{headings}	% lebender Kolumnentitel  

%% Deutsche Anpassungen %%%%%%%%%%%%%%%%%%%%%%%%%%%%%%%%%%%%%
\usepackage[utf8]{inputenc}
\usepackage[T1]{fontenc}
\usepackage[ngerman]{babel}
\usepackage{lmodern}

%\PrerenderUnicode{ü}
%\PrerenderUnicode{Ü}
%\PrerenderUnicode{ä}
%\PrerenderUnicode{Ä}
%\PrerenderUnicode{ö}
%\PrerenderUnicode{Ö}
%\PrerenderUnicode{ß}

%% Falls die automatische Worttrennung in Wörtern mit Umlauten
%% nicht funktionieren sollte oder der Text pixelig aussieht:
%% ==> Installieren Sie die cm-super Fonts (z.B. mit dem mikTeX Package Manager).
%% Eine nicht ganz vollwertige Alternative ist die Verwendung dieses Pakets:
%\usepackage{ae, aeguill}

%% Stichwortverzeichnis %%%%%%%%%%%%%%%%%%%%%%%%%%%%%%%%%%%%%
\usepackage{makeidx}

%% Packages für Grafiken & Abbildungen %%%%%%%%%%%%%%%%%%%%%%
\usepackage{graphicx}
\usepackage{caption}
\usepackage{subcaption} %%Moderne Alternative zu subfig

%% Beachten Sie:
%% Die Einbindung einer Grafik erfolgt mit \includegraphics{Dateiname}
%% bzw. über den Dialog im Einfügen-Menü.
%% 
%% Im Modus "LaTeX => PDF" können Sie u.a. folgende Grafikformate verwenden:
%%   .jpg  .png  .pdf  .mps
%% 
%% In den Modi "LaTeX => DVI", "LaTeX => PS" und "LaTeX => PS => PDF"
%% können Sie u.a. folgende Grafikformate verwenden:
%%   .eps  .ps  .bmp  .pict  .pntg


%% Mathe %%%%%%%%%%%%%%%%%%%%%%%%%%%%%%%%%%%%%
\usepackage{amsmath}
\usepackage{amsfonts}

%% Todo %%%%%%%%%%%%%%%%%%%%%%%%%%%%%%%%%%%%%
\usepackage{todonotes}

%% Farbanpassungen %%%%%%%%%%%%%%%%%%%%%%%%%%%%%%%%%%%%%%%%%%%%%%%%%%%%
\usepackage{xcolor}
\definecolor{unigruen}{rgb}{0,0.5961,0.6314}
\definecolor{unirot}{rgb}{0.373,0.0941,0.1176}

%% Seitenränder %%%%%%%%%%%%%%%%%%%%%%%%%%%%%%%%%%%%%%%%%%%%%%%%%%%%%%%
\usepackage[left=40mm, right=30mm, bottom=30mm]{geometry}

%% Kopf- und Fußzeilen %%%%%%%%%%%%%%%%%%%%%%%%%%%%%%%%%%%%%%%%%%%%%%%%
\usepackage{scrpage2}
\clearscrheadfoot %% löscht standard einstellungen
\ohead{\pagemark} %kopf rechts Seitennummer
%%\ihead{\thechapter\ \thesection\ \sectionmark} %Kopf links
\ihead{\leftmark}
\automark{section} 
\setheadsepline{0.2pt} %% horizontale linie im Kopf
\setfootsepline{0.2pt} %% horizontale linie im Fuß

%% Extended UTF-8 support  %%%%%%%%%%%%%%%%%%%%%%%%%%%%%%%%%%%%%%%%%%%%
\usepackage{ucs}

%% Tabellen  %%%%%%%%%%%%%%%%%%%%%%%%%%%%%%%%%%%%%%%%%%%%%%%%%%%%%%%%%%
\usepackage{booktabs}
%\usepackage{tabularx}

%% Protokolle etc. als Bitfeld  %%%%%%%%%%%%%%%%%%%%%%%%%%%%%%%%%%%%%%%
%% http://texdoc.net/texmf-dist/doc/latex/bytefield/bytefield.pdf
\usepackage{bytefield}

%% Bibliographiestil %%%%%%%%%%%%%%%%%%%%%%%%%%%%%%%%%%%%%%%%%%%%%%%%%%
%% ftp://ftp.tex.ac.uk/tex-archive/macros/latex/contrib/natbib/natbib.pdf
\usepackage{natbib}

%% Lorem Ipsum %%%%%%%%%%%%%%%%%%%%%%%%%%%%%%%%%%%%%%%%%%%%%%%%%%%%%%%%
%% http://ctan.org/pkg/lipsum
\usepackage{lipsum}

%% Absatz-Einstellungen %%%%%%%%%%%%%%%%%%%%%%%%%%%%%%%%%%%%%%%%%%%%%%%
\usepackage{parskip}

%% Code-Listings %%%%%%%%%%%%%%%%%%%%%%%%%%%%%%%%%%%%%%%%%%%%%%%%%%%%%%
\usepackage{listings}

%% Hyperlinks %%%%%%%%%%%%%%%%%%%%%%%%%%%%%%%%%%%%%%%%%%%%%%%%%%%%%%%%%
%% http://www.tug.org/applications/hyperref/ftp/doc/manual.pdf
\usepackage[backref,bookmarks=true,pdfpagelabels,plainpages=false]{hyperref}

\hypersetup{colorlinks=true}

%% Glossaries  %%%%%%%%%%%%%%%%%%%%%%%%%%%%%%%%%%%%%%%%%%%%%%%%%%%%%%%%%
%% http://ctan.org/pkg/glossaries

%\usepackage[nonumberlist,acronym,toc,section]{glossaries}
\usepackage[acronym]{glossaries}

%% Stichwortverzeichnis anfordern %%%%%%%%%%%%%%%%%%%%%%%%%%%%%%%%%%%%%%
\makeindex

%% Glossar anfordern %%%%%%%%%%%%%%%%%%%%%%%%%%%%%%%%%%%%%%%%%%%%%%%%%%%
\makeglossaries

% Glossar einbinden
\addto{\captionsngerman}{
  \renewcommand{\acronymname}{Abkürzungen}
}

\addto{\captionsngerman}{
  \renewcommand{\glossaryname}{Glossar}
}

\newglossaryentry{Cortex}{name = {Cortex}, description={32bit-Mikrocontrollerarchitektur von ARM, Inc}}
\newglossaryentry{cortex-m0}{name = {Cortex-M0}, description={Mikroprozessor der M0-Serie der ARM \gls{Cortex}-Architektur}}
\newglossaryentry{cortex-m0+}{name = {Cortex-M0+}, description={Mikroprozessor der M0+-Serie der ARM \gls{Cortex}-Architektur}}
\newglossaryentry{CoreSight}{name = {CoreSight}, description={Debug-Schnittstelle in \gls{Cortex}-\glspl{mcu}}}

\newglossaryentry{Kinetis}{name = {Kinetis}, description={32bit-Mikrocontrollertyp von Freescale Semiconductor}}
\newglossaryentry{kl25z}{name = {KL25Z}, description={Mikrocontroller der \gls{Kinetis}-L-Serie von Freescale auf Basis des ARM Cortex-M0+}}
\newglossaryentry{frdm-kl25z}{name = {FRDM-KL25Z}, description={Freescale Freedom Development Board mit KL25Z-\gls{mcu}}}

\newglossaryentry{Gimbal Lock}{name = {Gimbal Lock}, description={Kardanische Blockade eines Systemes bei der Verwendung von Euler'schen Winkeln, die bei ungünstiger Kombination von Rotationen zum Verlust eines Freiheitsgrades führt}}

\newglossaryentry{Kalman-Filter}{name = {Kalman-Filter}, description={Rekursives, lineares Filter zur Schätzung stochastischer Systemparameter, dessen Entwicklung auf Rudolf Emil Kálmán zurückgeht}}
\newglossaryentry{OpenSDA}{name = {OpenSDA}, description={Proprietäre, erweiterbare Programmierschnittstelle von Freescale}}
\newglossaryentry{OpenOCD}{name = {OpenOCD}, description={Quelloffene Programmierschnittstelle}}
\newglossaryentry{JTAG}{name = {JTAG}, description={Schnittstelle für Tests von Controllerschnittstellen und Programmierung}}
\newglossaryentry{elf}{name = {ELF}, description={Executable and Linking Format, eine ausführbare Datei}}
\newglossaryentry{mbed}{name = {mbed}, description={Plattform für die Entwicklung auf Cortex-M-MCUs}}
\newglossaryentry{Thumb2}{name = {Thumb2}, description={Befehlssatz von \glslink{Cortex}{ARM Cortex-M}-Prozessoren}}

\newglossaryentry{Quaternion}{name = {Quaternion}, description={Hamilton-Zahl $\mathbb{H}$ im 4-dimensionalen komplexer Raum, die zur singularitätsfreien Beschreibung von Orientierungen verwendet werden kann}}

\newglossaryentry{hard iron}{name = {Hard-Iron-Effekt}, description={Lineare Verzerrung im Magnetfeld durch Einwirkung "`harter"' ferromagnetischer Metalle, vgl. Soft-Iron-Effekt}}
\newglossaryentry{soft iron}{name = {Soft-Iron-Effekt}, description={Nichtlineare Verzerrung im Magnetfeld durch Einwirkung "`weicher"' ferromagnetischer Metalle, vgl. Hard-Iron-Effekt}}

\newglossaryentry{systick}{name = {SysTick}, description={System Tick Interrupt: Konfigurierbarer Timerinterrupt des Cortex-Cores zur Zeitmessung, welcher die Realisierungen von Echtzeitanwendungen unterstützen soll.}}

\newglossaryentry{bit banding}{name = {Bit-Banding}, description={Mapping von Speicheradresse auf einzelne Bits einer anderen Adresse}}

\newglossaryentry{tilt compensation}{name = {Tilt Compensation}, description={Korrektur der dreidimensionalen Neigung des Magnetometer-Messvektors mittels zusätzlicher Ausrichtungsdaten zur Bestimmung der magnetischen Nordrichtung in der zweidimensionalen Ebene}}


\newglossaryentry{Komplementaerfilter}{name = {Komplementärfilter}, description={Wichtungsfilter zwischen Beschleunigungs- und Drehratensensor zur Korrektur der spezifischen Drift- und Rauschmerkmale}}

\newglossaryentry{triad}{name = {TRIAD}, description={Algorithmus zur Orientierungsdeterminierung aus drei Vektorbeobachtungen}}

\newglossaryentry{q16}{name = {Q16}, description={32bit-Zahlenformat für Festkommazahlen mit jeweils 16 bit für Vor"- und Nachkommaanteil}}

\newglossaryentry{mpug}{name = {MPU}, description={Motion Processing Unit, eine durch Kopplung externer Sensoren erweiterbare Sensorserie von Invensense zur Verarbeitung und Fusion von Inertialdaten.}}

\newglossaryentry{i2cg}{name = {I\textsuperscript{2}C}, description={(auch IIC) Inter-Integrated Circuit. Ein von Philips entworfenes Bussystem für Inter-Chip-Kommunikation}}

\newglossaryentry{imug}{name = {IMU}, description={Inertial Measurement Unit, dt. Inertialmesseinheit; Ein Sensorsystem bestehend aus mehreren kombinierten Inertialsensoren}}

\newacronym{i2c}{I\textsuperscript{2}C}{Inter-Integrated Circuit\glsadd{i2cg}}
\newacronym{mpu}{MPU}{Motion Processing Unit\glsadd{mpug}}

\newacronym{spi}{SPI}{Serial Peripheral Interface}

\newacronym{dmp}{DMP}{Digital Motion Processor}
\newacronym{mac}{MAC}{Multiply and Accumulate}

%\newacronym{i2c}{I2C}{Inter-Integrated Circuit}

\newacronym[
	\glsshortpluralkey={MEMS},
	\glslongpluralkey={mikroelektromechanische Systeme}
]{mems}{MEMS}{mikroelektromechanisches System}

\newacronym{marg}{MARG}{Magnetic, Angular Rate and Gravitational}

\newacronym[
	\glsshortpluralkey={IMUs},
	\glslongpluralkey={Inertial Measurement Units}
]{imu}{IMU}{Inertial Measurement Unit\glsadd{imug}}

\newacronym{hci}{HCI}{Human-Computer Interaction}
\newacronym{hid}{HID}{Human Interface Device}
\newacronym{cmsis}{CMSIS}{Cortex Microcontroller Software Interface Standard}
\newacronym{cmsis-dap}{CMSIS-DAP}{\gls{cmsis} Debug Access Port}
\newacronym[
	\glsshortpluralkey={MCUs},
	\glslongpluralkey={Microcontroller Units}
]{mcu}{MCU}{Microcontroller Unit}
\newacronym{ocd}{OCD}{On-Chip-Debugger}
\newacronym{msd}{MSD}{Mass Storage Device}
\newacronym{usb}{USB}{Universal Serial Bus}
\newacronym{gdb}{GDB}{GNU Debugger}
\newacronym{dcm}{DCM}{Direction Cosine Matrix}
\newacronym{swd}{SWD}{Serial Wire Debug}
\newacronym{request}{REQUEST}{Recursive Quaternion Estimator}
\newacronym{quest}{QUEST}{Quaternion Estimator}

\newacronym{uart}{UART}{Universal Asynchronous Receiver Transmitter}

\newacronym[
	\glsshortpluralkey={GPIOs},
	\glslongpluralkey={General-Purpose I/Os}
]{gpio}{GPIO}{General-Purpose I/O}


\newacronym{soh}{SOH}{Start of Header}
\newacronym{eot}{EOT}{End of Transmission}
\newacronym{esc}{ESC}{Escape}

\newacronym{p2pp}{P2PP}{Point-to-Point Protocol}

\newacronym{wnu}{WNU}{West-North-Up}
\newacronym{enu}{ENU}{East-North-Up}
\newacronym{ecef}{ECEF}{Earth-Centered, Earth-Fixed}
\newacronym{fpu}{FPU}{Floating-Point Unit}

\newacronym{bme}{BME}{Bit Manipulation Engine}
\newacronym{bfi}{\texttt{BFI}}{Bit Field Insert}
\newacronym{lac}{\texttt{LAC}}{Load-and-Clear}
\newacronym{las}{\texttt{LAS}}{Load-and-Set}
\newacronym{ubfx}{\texttt{UBFX}}{Usigned Bit Field Extract}
\newacronym{pll}{PLL}{Phase-Locked Loop}

%% PDF-spezifische Einstellungen %%%%%%%%%%%%%%%%%%%%%%%%%%%%%%%%%%%%%%
\hypersetup{
						pdfpagemode=UseOutlines
						}

\makeatletter
\AtBeginDocument{
	\let\oldand\and\def\and{and }
	\hypersetup{
							pdftitle = {\@title},
							pdfauthor = {\@author},
							pdfsubject = {\@subject},
							pdfkeywords = {BHT, ARM, Cortex-M0, Kalman-Filter, MARG, IMU, Sensor Fusion, Signal Processing}
						}
	\let\and\oldand
}
\makeatother

%% Dokumenten-Titel und verwandtes %%%%%%%%%%%%%%%%%%%%%%%%%%%%%%%%%%%%%
\subject{Fortgeschrittene ARM-Microcontroller-Programmierung}
\title{Orientierungserkennung durch MARG-Sensorfusion mittels eines Kalman-Filters auf dem Freescale Kinetis KL25Z}
\author{Markus Mayer
				\and
				Julian Dombrow}
\publishers{Beuth-Hochschule für Technik, Berlin}
\date{\today}

\begin{document}

\cleardoublepage

\pagenumbering{gobble} % Seitenzahl entfernen! fix für pdflatex "identifier ... has been already used"'
\clearpage
\setlength{\voffset}{-15mm}
%\setlength{\hoffset}{10mm}
\thispagestyle{empty}

\makeatletter
\begin{titlepage}

\begin{flushright}
\includegraphics[height=2cm]{./images/Beuth_Logo_basis.pdf}
\end{flushright}

\textcolor{unigruen}{
	{\hrule height 1.5pt }
}

\begin{centering}
\vspace{2cm}
\textsc{\huge{Beuth Hochschule \\ für Technik Berlin}}\\
\vspace{2cm}
\textsc{\large{Fachbereich VI \\ Technische Informatik \\ Master of Embedded Systems}}\\
\vspace{0.6cm}
\textsc{\large{Fach: \\ Fortgeschrittene ARM-Microcontroller-Programmierung}}\\
\textsc{\large{WS 2013/14}}\\
\vspace{0.6cm}
\textsc{\large{Projekt: \\ \@title}}\\
\end{centering}

\vspace{2cm}
\begin{table}[h]
\begin{center}
\textsc{
\begin{tabular}{l l}
	\@author
\end{tabular}
}
\end{center}
\end{table}

\vfill

\begin{center}
\textsc{\large{Berlin, den \today}}
\end{center}

\end{titlepage}
\makeatother

\cleardoublepage
%% Angaben zur Standardformatierung des Titels %%%%%%%%%%%%%%%%%%%%%%%%
%\titlehead{Titelkopf}
%\thanks{Fußnote}					% entspr. \footnote im Fließtext

%% Rückseite der Titelseite %%%%%%%%%%%%%%%%%%%%%%%%%%%%%%%%%%%%%%%%%%%
\uppertitleback{Markus Mayer, B.Eng.\\Matr.-Nr. 798481\\\url{widemeadows@gmail.com}}
%\lowertitleback{Titelrückseitenfuß}

%% Widmungsseite %%%%%%%%%%%%%%%%%%%%%%%%%%%%%%%%%%%%%%%%%%%%%%%%%%%%%%
\dedication{Widmung}

\pagenumbering{Alph}
\maketitle 						% Titelei wird erzeugt
\thispagestyle{empty}

%% Erzeugung von Verzeichnissen %%%%%%%%%%%%%%%%%%%%%%%%%%%%%%%%%%%%%%%
\tableofcontents			% Inhaltsverzeichnis
\thispagestyle{empty}

\listoffigures				% Abbildungsverzeichnis
\thispagestyle{empty}

\lstlistoflistings
\thispagestyle{empty}

\listoftables				% Tabellenverzeichnis
\thispagestyle{empty}

\printglossaries		% Glossar
\thispagestyle{empty}

%\printglossary[type=symbolslist,title=Symbolverzeichnis]
%\thispagestyle{empty}

%% Der Text %%%%%%%%%%%%%%%%%%%%%%%%%%%%%%%%%%%%%%%%%%%%%%%%%%%%%%%%%%%
\frontmatter					% Vorspann (z.B. römische Seitenzahlen)
\chapter{Einleitung}

Ein häufiges Problem in der mobilen Robotik und eine zunehmende Notwendigkeit in der \gls{hci} --- etwa bei der Entwicklung 
von modernen Controllern für Unterhaltungssysteme --- ist die robuste Erkennung der Orientierung eines
beweglichen Systems, sei es einer autonomen Plattform (z.B. Roboter, Quadrocopter, ...) oder eines Eingabegerätes
(z.B. Nintendo Wiimote).

Während hochpräzise Inertialsensoren (\gls{imu}) wie Gyroskope in ihrer ursprünglichen Bauart rein mechanische Systeme sind, halten seit
einigen Jahren zunehmend \glspl{mems} Einzug in die Sensortechnik und stehen dabei in unterschiedlichen Genauigkeits-
und Kostengraden sowohl für Low-End, als auch High-End-Anwendungen zur Verfügung.

Hierbei werden grundlegend zwei Typen von Sensoren unterschieden: Inertialsensoren (\glspl{imu}), welche auf der Messung der Trägheit
des Systems beruhen --- klassische Vertreter sind der Accelerometer zur Messung von Beschleunigungen, 
sowie der Gyrosensor zur Messung von Drehraten --- als auch magnetische Sensoren, welche die Auswirkungen eines Magnetfeldes auf das 
System messen. Werden solche drei Sensoren in einem System kombiniert, spricht man von einem sogenannten \gls{marg}-Sensorsystem.

Dieses Projekt beschäftigt sich mit der Fusionierung eines \gls{marg}-Sensorsystems zur robusten Orientierungserkennung. Ziel ist
es, ein Maß für die Ausrichtung des Systemes im Raum zu ermitteln, welches

\begin{itemize}
	\item unanfällig gegenüber Messrauschen der Sensoren,
	\item stabil gegenüber externen Beschleunigungen und
	\item frei von Singulatitäten (\gls{Gimbal Lock}) ist.
\end{itemize}

Hierbei wird ein reguläres \gls{Kalman-Filter} zum Einsatz auf einem \gls{cortex-m0} entwickelt, welches adaptiv auf vorliegende
Messwerte reagiert und für den Echtzeiteinsatz geeignet ist. Die Implementierung erfolgt hierbei auf einem \gls{kl25z},
einem Mikrocontroller der Kinetis-Serie von Freescale auf Basis des ARM Cortex-M0+.

\begin{figure}[htbp]
		\centering
	\begin{subfigure}[b]{\textwidth}
		\centering
		\includegraphics[width=0.8\textwidth]{./images/board.jpg}
		\caption[FRDM-KL25Z mit externen Sensoren]{FRDM-KL25Z mit externen Sensoren: MPU6050 (blau) und HMC5883L (rot)}
		\label{fig:board}
	\end{subfigure}

	\begin{subfigure}[b]{\textwidth}
		\centering
		\includegraphics[width=0.8\textwidth]{./images/board-mit-logan.jpg}
		\caption[FRDM-KL25Z mit angeschlossenem Logic-Analyzer]{FRDM-KL25Z mit angeschlossenem Logic-Analyzer (rechts). Im Hintergrund links der verwendete UART-zu-USB-Transceiver.}
		\label{fig:board-logan}
	\end{subfigure}
	
	\caption{Prototyp auf dem Breadboard}
	\label{fig:prototype}
\end{figure}


\mainmatter						% Hauptteil
%\setpartpreamble[Position][Breite]{Text}
\setpartpreamble[u, r][6cm]{Dies ist ein Beispiel für eine Part-Präambel.}
\part[]{Entwicklungsumgebung}
\chapter{Entwicklungsumgebung}

\section{Freescale Kinetis KL25Z}

Auf jedem Freescale Freedom-Entwicklungsboard wie dem \gls{frdm-kl25z} ist  außerdem ein \gls{ocd} verbaut, welcher zusätzlich zum Debugging die 
Fähigkeit besitzt, die \gls{mcu} zu programmieren. Freescale nennt diese Einheit \gls{OpenSDA} (ganz im Sinne der "`Freedom"'-Benamung der Boards), 
die proprietärer Natur ist, jedoch die zusätzliche Beschaffung eines externen Programmers erspart.

\section{Freescale OpenSDA}

Der \gls{OpenSDA}-Chip auf dem  kommuniziert dabei mit der im \gls{Cortex}-Kern integrierten \gls{CoreSight}-Einheit über das \gls{JTAG}-Protokoll.

\begin{figure}[htbp]
		\centering
		\includegraphics{./images/opensda-block-diagram.png}
    \caption[OpenSDA-Blockdiagramm]{OpenSDA-Blockdiagramm}
		Quelle: \texttt{http://mcuoneclipse.com/2012/09/20/opensda-on-the-freedom-kl25z-board/}
    \label{fig:cmsis-dap}
\end{figure}

Eine Gegenüberstellung der alternativen SDA-Firmwares ohne Anspruch auf Vollständigkeit kann Tabelle \ref{tab:opensda-options} entnommen werden.

\begin{table}
\begin{tabular}{llll} 
\toprule
Gegenüberstellung von OpenSDA-Firmwares\\  
\midrule 
Name & Art & Breakpoints & UART \\ 
\midrule 
OpenSDA & proprietär & hardware & ja\\
CMSIS-DAP & open source & hardware & nein \\
Segger J-Link & proprietär & unbegrenzt & nein \\ 
\bottomrule
\end{tabular}
\caption{Gegenüberstellung von OpenSDA-Firmwares}
\label{tab:opensda-options}
\end{table}

\subsection{CMSIS-DAP}

Eine alternative (und quelloffene) Firmware für den \gls{OpenSDA}-Chip ist die \gls{cmsis-dap}-Firmware.

--> Verwendung mit \gls{OpenOCD}

\begin{figure}[htbp]
		\centering
		\includegraphics[width=0.8\textwidth]{./images/cmsis_dap_interface.png}
    \caption[CMSIS-DAP]{CMSIS-DAP}
		Quelle: \texttt{http://nimblemachines.com/cmsis-dap/}
    \label{fig:cmsis-dap}
\end{figure}


\subsection{Segger J-Link}

\lipsum[2]


\section{Freescale CodeWarrior}

\lipsum[3]

\section{SysProgs VisualGDB}

\lipsum[4]
\chapter{Sensoren}

Alle in diesem Projekt verwendeten Sensoren verfügen über eine \gls{i2c}-Schnittstelle, über welche die Kommunikation abgewickelt wurde. Einige Module, wie die MPU6050 (siehe Abschnitt~\ref{sec:mpu6050})
verfügen darüberhinaus über Interrupt-Leitungen, welche das Vorliegen neuer Messwerte signalisieren.

\section{MMA8451Q - Accelerometer}

Der Freescale MMA8451Q ist der auf dem Freedom-Board verbaute Beschleunigungssensor. Er bietet eine Auflösung von X bei einer maximalen Samplingrate von Y. 

\subsection{Treiber}

\section{MPU6050 - Accelerometer und Gyrosensor}
\label{sec:mpu6050}

Die Invensense MPU6050 ist eine \gls{imu}, die einen Beschleunigungs- und einen Drehratensensor vereinigt. Diese \gls{mpu} umfasst sowohl einen Beschleunigungs-, als auch einen Drehratensensor und bietet die Möglichkeit, über einen als slave betriebenen, zusätzlichen \gls{i2c}-Bus weitere Sensoren zu verarbeiten, wodurch Aufwand vom Hauptcontroller abfällt. Der \gls{mpu} zugeschaltet ist ein sog. \gls{dmp}, welcher die internen Daten, sowie die über den slave-Bus bezogenen externen Daten fusionieren kann. Während die Ergebnisregister des \gls{dmp} frei zugänglich sind, sind die Inhalte dieser Felder nicht dokumentiert. Der Zugriff auf sie erfolgt stattdessen --- sofern erwünscht --- über eine proprietäre MotionsApps-Firmware\footnote{\url{http://www.invensense.com/developers/forum/viewtopic.php?f=3&t=142}}.

\subsection{Treiber}

\subsection{Lessons Learned}

Bei der Kommunikation mit der MPU6050 trat das Problem auf, dass bei aktiviertem Interrupt-Signal und reduzierter Samplingrate nach einem Kaltstart keinerlei nennenswerte Verzögerung (d.h. nur wenige Millisekunden \todo{Verifizieren}) zwischen aufienander folgenden Interruptsignalen festzustellen war. 
Dies führte zu der Problematik, dass die durch den Interrupt-Handler freigeschaltete Routine zum Beziehen der Sensordaten über \gls{i2c} und die anschließende Verarbeitung direkt durch einen erneuten Interrupt unterbrochen wurde. Da die serielle Ausgabe der Werte nur mit deutlich geringerer Taktung laufen kann, führte dies zu einem Engpass am vorgeschalteten Ringpuffer, wodurch die Verarbeitung deutlich inperformant wurde.

Wurde das System jedoch durch einen Reset (d.h. durch einen Warmstart) neu initialisiert, verlief die Kommunikation wie gewünscht. Durch einen weiteren Kaltstart konnte der Zustand erneut herbeigeführt werden.

\todo{Bild mit Freifeuer vom Logan}

Dieses Problem war erst dadurch zu beheben, dass zu Beginn der Konfiguration der \gls{imu} die interne Takteinheit deaktiviert wurde, um sie dann im Zuge der folgenden Konfiguration erneut auf den gewünschten Betriebsmodus zu stellen.

\todo{Bild ohne Freifeuer vom Logan}

\section{HMC5883L - Magnetometer}

Der HMC5883L von Honeywell liefert bei einer Auflösung von 12 Bit (inkl. Vorzeichen) bei einem Wertebereich von $\pm$ 8 Gauss, wodurch er für den Einsatz in Präsenz starker lokaler Magnetfelder geeignet ist.

\subsection{Treiber}

\subsection{Lessons Learned}

Bei der Auswertung der Sensordaten des HMC5883L traten zwei kleine bis mittlere Komplikationen auf. Eines dieser Probleme lag in der Ansteuerung des Sensors, 
das andere dagegen in der Interpretation der Messwerte begründet.

\subsubsection{Reihenfolge der Messwerte}

Während in sämtlichen anderen Sensoren die Messwerte in üblicher $X$,$Y$,$Z$-Reihenfolge vorlagen, speichert der HMC5883L die Messwerte in der Reihenfolge $X$,$Z$,$Y$. Dies ist aus dem Datenblatt (\citealp{hmc5883l}) nur
dann ersichtlich, wenn die im Datenblatt vorliegende --- sehr kurze --- Registerliste (S. 11) sehr genau beachtet wird. Im Gegensatz zu der dort aufgeführten Reihenfolge, verweisen sämtlichen
anderen Stellen des Datenblattes (insbesondere im Abschnitt Data Output Registers, S. 15) auf die reguläre Anordnung.

\subsubsection{Interpretation der Messergebnisse}

\begin{figure}[htbp]
		\centering
		\begin{subfigure}[b]{0.5\textwidth}
			\centering
			\includegraphics[width=\textwidth]{./images/earth_magnetic_field_poles_shutterstock.jpg}
			\caption{Populistische Darstellung \\ Quelle: Shutterstock}
		\end{subfigure}%
		~
		\begin{subfigure}[b]{0.5\textwidth}
			\centering
			\includegraphics[width=\textwidth]{./images/Dipole_field_wikibooks.jpg}
			\caption{Vereinfachte realistische Darstellung \\ Quelle: Wikibooks}
		\end{subfigure}%
		\caption[Darstellungen des Erdmagnetfeldes]{Darstellungen des Erdmagnetfeldes.\\Links: Die Feldlinien treffen sich an den Polen.\\Rechts: Die Feldlinien durchdringen den Erdmantel.}
\end{figure}
\setchapterpreamble[u]{%
\dictum[Luhmann]{Die Klassiker sind Klassiker, weil sie Klassiker sind \dots}}
\chapter{Fusionsalgorithmus}

Exemplarischer Verweis auf ein Buch mit dieser Aussage, vgl. \cite{Filieri}, außerdem \cite{Tsang}. Dort werden \glspl{mems} erwähnt. So ein \gls{mems} ist toll.

\begin{figure}[htbp]
	\centering
	\includegraphics[width=\textwidth]{./images/matlab/rollpitchyaw45-2.png}
	\caption[Extraktion der \textsc{Euler}'schen Winkel]{Extraktion der \textsc{Euler}'schen Winkel. Deutlich zu erkennen sind die Singularitäten bei Sekunden 13, 17 und 29 und 32.}
\end{figure}


\appendix							% Beginn des Anhangs
\chapter{Umprogrammieren der OpenSDA-Firmware unter Linux}
\label{chap:opensda_linux}

\lstdefinestyle{lolbash}{
    language={bash}, 
		basicstyle=\ttfamily\tiny,
		frame=single,
    moredelim=**[is][\slshape]{`}{`},
    moredelim=**[is][\color{orange}]{°}{°},
}

Wie in Abschnitt~\ref{subsec:opensda_windows8.1} erwähnt, kann das \gls{frdm-kl25z} unter Windows 8.1 nicht ordnungsgemäß
in Betrieb genommen werden, sobald die Firmware einen \gls{msd}-Modus verwendet. Dies betrifft sowohl den Programmer
der \gls{mcu}, als auch den Programmer des \gls{OpenSDA}-Chips selbst.

Im Folgenden findet sich eine Beschreibung des Flashvorganges unter Linux, nachdem das Board auch unter Windows 8.1
wieder verwendet werden kann. Dieser Vorgang sollte auch dann zum Erfolg führen, wenn das Board unter Windows 8.1
bereits fehlerhaft (d.h. mit Abbruch des Dateitransfers) geflasht wurde.

Zuerst wird das Board bei gedrückter Reset-Taste angeschlossen, wodurch es im Bootloader-Modus startet. Die Ausgabe 
von \texttt{lsusb} sollte das Gerät mit der ID \texttt{2504:0200} auflisten.

\begin{lstlisting}[style=lolbash]
user@host:~$ lsusb
Bus 002 Device 001: ID 1d6b:0002 Linux Foundation 2.0 root hub
Bus 007 Device 001: ID 1d6b:0001 Linux Foundation 1.1 root hub
Bus 006 Device 001: ID 1d6b:0001 Linux Foundation 1.1 root hub
°Bus 005 Device 008: ID 2504:0200°
Bus 005 Device 001: ID 1d6b:0001 Linux Foundation 1.1 root hub
Bus 001 Device 001: ID 1d6b:0002 Linux Foundation 2.0 root hub
Bus 004 Device 001: ID 1d6b:0001 Linux Foundation 1.1 root hub
Bus 003 Device 002: ID 0483:2016 STMicroelectronics Fingerprint Reader
Bus 003 Device 003: ID 0a5c:2110 Broadcom Corp. BCM2045B (BDC-2) [Bluetooth Controller]
Bus 003 Device 001: ID 1d6b:0001 Linux Foundation 1.1 root hub
\end{lstlisting}

Mittels \texttt{dmesg} wird nun ermittelt, als welches Device das USB-Gerät angemeldet wurde.

\begin{lstlisting}[style=lolbash]
user@host:~$ dmesg
[ 2178.532100] usb 5-2: new full-speed USB device number 10 using uhci_hcd
[ 2178.703183] usb 5-2: New USB device found, idVendor=2504, idProduct=0200
[ 2178.703193] usb 5-2: New USB device strings: Mfr=1, Product=2, SerialNumber=3
°[ 2178.703201] usb 5-2: Product: OpenSDA MSD APP°
°[ 2178.703207] usb 5-2: Manufacturer: FREESCALE SEMICONDUCTOR INC.°
°[ 2178.703213] usb 5-2: SerialNumber: 0123456789ABCDEF°
°[ 2178.706301] usb-storage 5-2:1.0: USB Mass Storage device detected°
[ 2178.706467] scsi11 : usb-storage 5-2:1.0
°[ 2179.709257] scsi 11:0:0:0: Direct-Access     FSL      FSL/PEMICRO MSD  0001 PQ: 0 ANSI: 4°
[ 2179.709869] sd 11:0:0:0: Attached scsi generic sg2 type 0
°[ 2179.718217] sd 11:0:0:0: [sdb] 1983999 512-byte logical blocks: (1.01 GB/968 MiB)°
[ 2179.721525] sd 11:0:0:0: [sdb] Write Protect is off
[ 2179.721531] sd 11:0:0:0: [sdb] Mode Sense: 00 00 00 00
[ 2179.724171] sd 11:0:0:0: [sdb] Asking for cache data failed
[ 2179.724183] sd 11:0:0:0: [sdb] Assuming drive cache: write through
[ 2179.742182] sd 11:0:0:0: [sdb] Asking for cache data failed
[ 2179.742187] sd 11:0:0:0: [sdb] Assuming drive cache: write through
[ 2179.766202]  sdb:
[ 2179.783187] sd 11:0:0:0: [sdb] Asking for cache data failed
[ 2179.783193] sd 11:0:0:0: [sdb] Assuming drive cache: write through
[ 2179.783197] sd 11:0:0:0: [sdb] Attached SCSI removable disk
\end{lstlisting}

Man kann erkennen, dass das Laufwerk als \texttt{/dev/sdb} angemeldet wurde.
Es kann nun mittels \texttt{mount -t vfat} gemountet und der Erfolg überprüft werden.

\begin{lstlisting}[style=lolbash]
user@host:~$ sudo mount -t vfat /dev/sdb /mnt
user@host:~$ ls -lisa /mnt
insgesamt 84
  1 16 drwxr-xr-x  2 root root 16384 Jan  1  1970 .
  2  4 drwxr-xr-x 23 root root  4096 Nov 21 02:12 ..
102 16 -r-xr-xr-x  1 root root   512 Aug  8  2012 FSL_WEB.HTM
100 16 -r-xr-xr-x  1 root root    68 Aug  8  2012 LASTSTAT.TXT
101 16 -r-xr-xr-x  1 root root  1536 Aug  8  2012 SDA_INFO.HTM
103 16 -r-xr-xr-x  1 root root   512 Aug  8  2012 TOOLS.HTM
\end{lstlisting}

Es ist zu beachten, dass der Mountvorgang durchaus mehrere Minuten dauern kann.

Die Datei \texttt{LASTSTAT.TXT} beinhaltet den letzten Status der Firmware, womit man
den erfolgreichen Verlauf des Mountvorganges überprüfen kann.

\begin{lstlisting}[style=lolbash]
user@host:~$ cat /mnt/LASTSTAT.TXT
°Ready.°
\end{lstlisting}

Die neue Firmware kann nun durch einen Kopierbefehl an das Gerät gesendet werden; Im Falle
der \emph{Segger J-Link}-Variante könnte dies etwa wie folgt aussehen:

\begin{lstlisting}[style=lolbash]
user@host:~$ sudo cp JLink_OpenSDA.sda /mnt
user@host:~$ cat /mnt/LASTSTAT.TXT
°Completed.°
\end{lstlisting}

Ein \texttt{Completed.} signalisiert den erfolgreichen Flash-Vorgang.

Nach dem Auswerfen des Laufwerkes mittels \texttt{umount}

\begin{lstlisting}[style=lolbash]
user@ahost:~$ sudo umount /mnt
\end{lstlisting}

und einem Neustart des Boards im regulären Modus (d.h. ohne gedrückten Reset-Taster) meldet 
sich das Board mit der neuen Firmware an:

\begin{lstlisting}[style=lolbash]
user@host:~$ lsusb
Bus 002 Device 001: ID 1d6b:0002 Linux Foundation 2.0 root hub
Bus 007 Device 001: ID 1d6b:0001 Linux Foundation 1.1 root hub
Bus 006 Device 001: ID 1d6b:0001 Linux Foundation 1.1 root hub
°Bus 005 Device 011: ID 1366:0101 SEGGER J-Link ARM°
Bus 005 Device 001: ID 1d6b:0001 Linux Foundation 1.1 root hub
Bus 001 Device 001: ID 1d6b:0002 Linux Foundation 2.0 root hub
Bus 004 Device 001: ID 1d6b:0001 Linux Foundation 1.1 root hub
Bus 003 Device 002: ID 0483:2016 STMicroelectronics Fingerprint Reader
Bus 003 Device 003: ID 0a5c:2110 Broadcom Corp. BCM2045B (BDC-2) [Bluetooth Controller]
Bus 003 Device 001: ID 1d6b:0001 Linux Foundation 1.1 root hub
\end{lstlisting}

Analog kann die Ausgabe mittels \texttt{dmesg} überprüft werden.

\begin{lstlisting}[style=lolbash]
user@host:~$ dmesg
[ 2637.380139] usb 5-2: USB disconnect, device number 10
[ 2638.312070] usb 5-2: new full-speed USB device number 11 using uhci_hcd
[ 2638.479164] usb 5-2: New USB device found, idVendor=1366, idProduct=0101
[ 2638.479176] usb 5-2: New USB device strings: Mfr=1, Product=2, SerialNumber=3
°[ 2638.479183] usb 5-2: Product: J-Link°
°[ 2638.479189] usb 5-2: Manufacturer: SEGGER°
°[ 2638.479195] usb 5-2: SerialNumber: 000621000000°
\end{lstlisting}

Wird stattdessen die \emph{P\&E Microcomputer Debug OpenSDA}-Variante geflasht, lauten die letzten
Ausgaben sinngemäß

\begin{lstlisting}[style=lolbash]
user@host:~$ lsusb
Bus 002 Device 001: ID 1d6b:0002 Linux Foundation 2.0 root hub
Bus 007 Device 001: ID 1d6b:0001 Linux Foundation 1.1 root hub
Bus 006 Device 001: ID 1d6b:0001 Linux Foundation 1.1 root hub
°Bus 005 Device 013: ID 1357:0089 P&E Microcomputer Systems°
Bus 005 Device 001: ID 1d6b:0001 Linux Foundation 1.1 root hub
Bus 001 Device 001: ID 1d6b:0002 Linux Foundation 2.0 root hub
Bus 004 Device 001: ID 1d6b:0001 Linux Foundation 1.1 root hub
Bus 003 Device 002: ID 0483:2016 STMicroelectronics Fingerprint Reader
Bus 003 Device 003: ID 0a5c:2110 Broadcom Corp. BCM2045B (BDC-2) [Bluetooth Controller]
Bus 003 Device 001: ID 1d6b:0001 Linux Foundation 1.1 root hub
\end{lstlisting}

und 

\begin{lstlisting}[style=lolbash]
sunside@aquitaine:~$ dmesg | tail
[ 3015.141218] sd 12:0:0:0: [sdb] Attached SCSI removable disk
[ 3346.412156] usb 5-2: USB disconnect, device number 12
[ 3347.768119] usb 5-2: new full-speed USB device number 13 using uhci_hcd
[ 3347.950185] usb 5-2: New USB device found, idVendor=1357, idProduct=0089
[ 3347.950195] usb 5-2: New USB device strings: Mfr=1, Product=3, SerialNumber=5
°[ 3347.950202] usb 5-2: Product: OpenSDA Hardware°
°[ 3347.950208] usb 5-2: Manufacturer: P&E Microcomputer Systems Inc.°
°[ 3347.950214] usb 5-2: SerialNumber: SDADBB27E5D°
[ 3347.953292] cdc_acm 5-2:1.0: This device cannot do calls on its own. It is not a modem.
[ 3347.953332] cdc_acm 5-2:1.0: ttyACM0: USB ACM device
\end{lstlisting}

\backmatter					% Nachspann 

%% Stichwortverzeichnis anzeigen %%%%%%%%%%%%%%%%%%%%%%%%%%%%%%%%%%%%%%
\printindex

%% Bibliographie unter Verwendung von dinnat %%%%%%%%%%%%%%%%%%%%%%%%%%
%\setbibpreamble{Präambel des Literaturverzeichnisses}		% Text vor dem Verzeichnis
\bibliographystyle{dinat}
\bibliography{./bib/quellen}	% Sie benötigen einen *.bib-Datei

\end{document}