%%%%%%%%%%%%%%%%%%%%%%%%%%%%%%%%%%%%%%%%%%%%%%%%%%%%%%%%%%%%%%%%%%%%%%%
%% Optionen zum Layout des Buchs                                     %%
%%%%%%%%%%%%%%%%%%%%%%%%%%%%%%%%%%%%%%%%%%%%%%%%%%%%%%%%%%%%%%%%%%%%%%%
\documentclass[
a4paper,							% alle weiteren Papierformat einstellbar
%landscape,						% Querformat
11pt,								  % Schriftgröße (12pt, 11pt (Standard))
%BCOR1cm,							% Bindekorrektur, bspw. 1 cm
%DIVcalc,							% führt die Satzspiegelberechnung neu aus
%											  s. scrguide 2.4
%oneside,							% einseitiges Layout
%twocolumn,						% zweispaltiger Satz
%openany,							% Kapitel können auch auf linken Seiten beginnen
%halfparskip*,				% Absatzformatierung s. scrguide 3.1
%headsepline,					% Trennline zum Seitenkopf	
%footsepline,					% Trennline zum Seitenfuß
%notitlepage,					% in-page-Titel, keine eigene Titelseite
bibliography=totoc,
chapterprefix,				% vor Kapitelüberschrift wird "Kapitel Nummer" gesetzt
appendixprefix,				% Anhang wird "Anhang" vor die Überschrift gesetzt 
%normalheadings,			% Überschriften etwas kleiner (smallheadings)
%leqno,								% Nummerierung von Gleichungen links
fleqn,								% Ausgabe von Gleichungen linksbündig
%draft								  % überlangen Zeilen in Ausgabe gekennzeichnet
]
{scrbook}

%\pagestyle{empty}		% keine Kopf und Fußzeile (k. Seitenzahl)
%\pagestyle{headings}	% lebender Kolumnentitel  

%% Deutsche Anpassungen %%%%%%%%%%%%%%%%%%%%%%%%%%%%%%%%%%%%%
\usepackage[utf8x]{inputenc}
\usepackage[T1]{fontenc}
\usepackage[ngerman]{babel}
\usepackage{lmodern}

%% Falls die automatische Worttrennung in Wörtern mit Umlauten
%% nicht funktionieren sollte oder der Text pixelig aussieht:
%% ==> Installieren Sie die cm-super Fonts (z.B. mit dem mikTeX Package Manager).
%% Eine nicht ganz vollwertige Alternative ist die Verwendung dieses Pakets:
%\usepackage{ae, aeguill}

%% Stichwortverzeichnis %%%%%%%%%%%%%%%%%%%%%%%%%%%%%%%%%%%%%
\usepackage{makeidx}

%% Packages für Grafiken & Abbildungen %%%%%%%%%%%%%%%%%%%%%%
\usepackage{graphicx}
\usepackage{subcaption} %%Moderne Alternative zu subfig

%% Beachten Sie:
%% Die Einbindung einer Grafik erfolgt mit \includegraphics{Dateiname}
%% bzw. über den Dialog im Einfügen-Menü.
%% 
%% Im Modus "LaTeX => PDF" können Sie u.a. folgende Grafikformate verwenden:
%%   .jpg  .png  .pdf  .mps
%% 
%% In den Modi "LaTeX => DVI", "LaTeX => PS" und "LaTeX => PS => PDF"
%% können Sie u.a. folgende Grafikformate verwenden:
%%   .eps  .ps  .bmp  .pict  .pntg

%% Extended UTF-8 support  %%%%%%%%%%%%%%%%%%%%%%%%%%%%%%%%%%%%%%%%%%%%
\usepackage{ucs}

%% Tabellen  %%%%%%%%%%%%%%%%%%%%%%%%%%%%%%%%%%%%%%%%%%%%%%%%%%%%%%%%%%
%% http://webdocs.cs.ualberta.ca/~c603/latex/tabularx.pdf
\usepackage{tabularx}

%% Protokolle etc. als Bitfeld  %%%%%%%%%%%%%%%%%%%%%%%%%%%%%%%%%%%%%%%
%% http://texdoc.net/texmf-dist/doc/latex/bytefield/bytefield.pdf
\usepackage{bytefield}

%% Bibliographiestil %%%%%%%%%%%%%%%%%%%%%%%%%%%%%%%%%%%%%%%%%%%%%%%%%%
%% ftp://ftp.tex.ac.uk/tex-archive/macros/latex/contrib/natbib/natbib.pdf
\usepackage{natbib}

%% Lorem Ipsum %%%%%%%%%%%%%%%%%%%%%%%%%%%%%%%%%%%%%%%%%%%%%%%%%%%%%%%%
%% http://ctan.org/pkg/lipsum
\usepackage{lipsum}

%% Hyperlinks %%%%%%%%%%%%%%%%%%%%%%%%%%%%%%%%%%%%%%%%%%%%%%%%%%%%%%%%%
%% http://www.tug.org/applications/hyperref/ftp/doc/manual.pdf
\usepackage[backref,bookmarks=true,pdfpagelabels,plainpages=false]{hyperref}

\hypersetup{colorlinks=true}

%% Glossaries  %%%%%%%%%%%%%%%%%%%%%%%%%%%%%%%%%%%%%%%%%%%%%%%%%%%%%%%%%
%% http://ctan.org/pkg/glossaries
\usepackage[acronym]{glossaries}
\addto{\captionsngerman}{
  \renewcommand{\acronymname}{Abkürzungen}
}

\addto{\captionsngerman}{
  \renewcommand{\glossaryname}{Glossar}
}

\newglossaryentry{Cortex}{name = {Cortex}, description={32bit-Mikrocontrollerarchitektur von ARM, Inc}}
\newglossaryentry{cortex-m0}{name = {Cortex-M0}, description={\gls{mcu} der M0-Serie der ARM \gls{Cortex}-Architektur}}
\newglossaryentry{cortex-m0+}{name = {Cortex-M0+}, description={\gls{mcu} der M0+-Serie der ARM \gls{Cortex}-Architektur}}
\newglossaryentry{CoreSight}{name = {CoreSight}, description={Debug-Schnittstelle in \gls{Cortex}-\glspl{mcu}}}

\newglossaryentry{Kinetis}{name = {Kinetis}, description={32bit-Mikrocontrollertyp von Freescale Semiconductor}}
\newglossaryentry{kl25z}{name = {KL25Z}, description={Mikrocontroller der \gls{Kinetis}-L-Serie von Freescale auf Basis des ARM Cortex-M0+}}
\newglossaryentry{frdm-kl25z}{name = {FRDM-KL25Z}, description={Freescale Freedom Development Board mit KL25Z-\gls{mcu}}}

\newglossaryentry{Gimbal Lock}{name = {Gimbal Lock}, description={Kardanische Blockade eines Systemes bei der Verwendung von Euler'schen Winkeln, die bei ungünstiger Kombination von Rotationen zum Verlust eines Freiheitsgrades führt}}

\newglossaryentry{Kalman-Filter}{name = {Kalman-Filter}, description={Rekursives, lineares Filter zur Schätzung stochastischer Systemparameter, dessen Entwicklung auf Rudolf Emil Kálmán zurückgeht}}
\newglossaryentry{OpenSDA}{name = {OpenSDA}, description={Proprietäre, erweiterbare Programmierschnittstelle von Freescale}}
\newglossaryentry{OpenOCD}{name = {OpenOCD}, description={Quelloffene Programmierschnittstelle}}
\newglossaryentry{JTAG}{name = {JTAG}, description={Schnittstelle für Tests von Controllerschnittstellen und Programmierung}}
\newglossaryentry{elf}{name = {ELF}, description={Executable and Linking Format, eine ausführbare Datei}}
\newglossaryentry{mbed}{name = {mbed}, description={Plattform für die Entwicklung auf Cortex-M-MCUs}}
\newglossaryentry{i2c}{name = {I2C}, description={(auch IIC) Inter-Integrated Circuit. Ein von Philips entworfenes Bussystem für Inter-Chip-Kommunikation.}}

% http://tex.stackexchange.com/questions/8946/how-to-combine-acronym-and-glossary
%\newglossaryentry{i2cg}{
%	name=\glslink{i2c}{Application Programming Interface (\gls{i2c})},
%	description={Application Programming Interface Desc}
%}

% http://tex.stackexchange.com/questions/8946/how-to-combine-acronym-and-glossary
%\newglossaryentry{i2c}{
%	type=\acronymtype,
%	name=I2C,
%	first=Inter-Integrated Circuit (I2C),
%	see=[Glossary:]{\gls{i2cg}}, 
%	description=\glslink{i2cg}{Inter-Integrated Circuit}
%}

%\newacronym{i2c}{I2C}{Inter-Integrated Circuit \glsadd{i2cg}}
%\newacronym{i2c}{I2C}{Inter-Integrated Circuit}

\newacronym[
	\glsshortpluralkey={MEMS},
	\glslongpluralkey={mikroelektromechanische Systeme}
]{mems}{MEMS}{mikroelektromechanisches System}

\newacronym{marg}{MARG}{Magnetic, Angular Rate and Gravitational}

\newacronym[
	\glsshortpluralkey={IMUs},
	\glslongpluralkey={Inertial Measurement Units}
]{imu}{IMU}{Inertial Measurement Unit}

\newacronym{hci}{HCI}{Human-Computer Interaction}
\newacronym{hid}{HID}{Human Interface Device}
\newacronym{cmsis}{CMSIS}{Cortex Microcontroller Software Interface Standard}
\newacronym{cmsis-dap}{CMSIS-DAP}{\gls{cmsis} Debug Access Port}
\newacronym[
	\glsshortpluralkey={MCUs},
	\glslongpluralkey={Microcontroller Units}
]{mcu}{MCU}{Microcontroller Unit}
\newacronym{ocd}{OCD}{On-Chip-Debugger}
\newacronym{msd}{MSD}{Mass Storage Device}
\newacronym{usb}{USB}{Universal Serial Bus}
\newacronym{gdb}{GDB}{GNU Debugger}
\newacronym{swd}{SWD}{Serial Wire Debug}

\newacronym[
	\glsshortpluralkey={GPIOs},
	\glslongpluralkey={General-Purpose I/Os}
]{gpio}{GPIO}{General-Purpose I/O}


%% PDF-spezifische Einstellungen %%%%%%%%%%%%%%%%%%%%%%%%%%%%%%%%%%%%%%
\hypersetup{
						pdfpagemode=UseOutlines
						}

\makeatletter
\AtBeginDocument{
	\let\oldand\and\def\and{and }
	\hypersetup{
							pdftitle = {\@title},
							pdfauthor = {\@author},
							pdfsubject = {\@subject},
							pdfkeywords = {BHT, ARM, Cortex-M0, Kalman-Filter, MARG, IMU, Sensor Fusion, Signal Processing}
						}
	\let\and\oldand
}
\makeatother

%% Dokumenten-Titel und verwandtes %%%%%%%%%%%%%%%%%%%%%%%%%%%%%%%%%%%%%
\subject{Fortgeschrittene ARM-Microcontroller-Programmierung}
\title{MARG Sensor Fusion mittels Kalman-Filter auf ARM Cortex-M0}
\author{Markus Mayer
				\and
				Julian Dombrow}
\publishers{Beuth-Hochschule für Technik, Berlin}
\date{\today}

%% Stichwortverzeichnis anfordern %%%%%%%%%%%%%%%%%%%%%%%%%%%%%%%%%%%%%%
\makeindex

%% Glossar anfordern %%%%%%%%%%%%%%%%%%%%%%%%%%%%%%%%%%%%%%%%%%%%%%%%%%%
\makeglossaries

\begin{document}

%% Angaben zur Standardformatierung des Titels %%%%%%%%%%%%%%%%%%%%%%%%
%\titlehead{Titelkopf}
%\thanks{Fußnote}					% entspr. \footnote im Fließtext

%% Rückseite der Titelseite %%%%%%%%%%%%%%%%%%%%%%%%%%%%%%%%%%%%%%%%%%%
\uppertitleback{Markus Mayer, B.Eng.\\Matr.-Nr. 798481\\\url{widemeadows@gmail.com}}
%\lowertitleback{Titelrückseitenfuß}

%% Widmungsseite %%%%%%%%%%%%%%%%%%%%%%%%%%%%%%%%%%%%%%%%%%%%%%%%%%%%%%
\dedication{Widmung}

\pagenumbering{Alph}
\maketitle 						% Titelei wird erzeugt
\thispagestyle{empty}


%% Der Text %%%%%%%%%%%%%%%%%%%%%%%%%%%%%%%%%%%%%%%%%%%%%%%%%%%%%%%%%%%
\frontmatter					% Vorspann (z.B. römische Seitenzahlen)
\chapter{Einleitung}

Ein häufiges Problem in der mobilen Robotik und eine zunehmende Notwendigkeit in der \gls{hci} --- etwa bei der Entwicklung 
von modernen Controllern für Unterhaltungssysteme --- ist die robuste Erkennung der Orientierung eines
beweglichen Systems, sei es einer autonomen Plattform (z.B. Roboter, Quadrocopter, ...) oder eines Eingabegerätes
(z.B. Nintendo Wiimote).

Während hochpräzise Inertialsensoren (\gls{imu}) wie Gyroskope in ihrer ursprünglichen Bauart rein mechanische Systeme sind, halten seit
einigen Jahren zunehmend \glspl{mems} Einzug in die Sensortechnik und stehen dabei in unterschiedlichen Genauigkeits-
und Kostengraden sowohl für Low-End, als auch High-End-Anwendungen zur Verfügung.

Hierbei werden grundlegend zwei Typen von Sensoren unterschieden: Inertialsensoren (\glspl{imu}), welche auf der Messung der Trägheit
des Systems beruhen --- klassische Vertreter sind der Accelerometer zur Messung von Beschleunigungen, 
sowie der Gyrosensor zur Messung von Drehraten --- als auch magnetische Sensoren, welche die Auswirkungen eines Magnetfeldes auf das 
System messen. Werden solche drei Sensoren in einem System kombiniert, spricht man von einem sogenannten \gls{marg}-Sensorsystem.

Dieses Projekt beschäftigt sich mit der Fusionierung eines \gls{marg}-Sensorsystems zur robusten Orientierungserkennung. Ziel ist
es, ein Maß für die Ausrichtung des Systemes im Raum zu ermitteln, welches

\begin{itemize}
	\item unanfällig gegenüber Messrauschen der Sensoren,
	\item stabil gegenüber externen Beschleunigungen und
	\item frei von Singulatitäten (\gls{Gimbal Lock}) ist.
\end{itemize}

Hierbei wird ein reguläres \gls{Kalman-Filter} zum Einsatz auf einem \gls{cortex-m0} entwickelt, welches adaptiv auf vorliegende
Messwerte reagiert und für den Echtzeiteinsatz geeignet ist. Die Implementierung erfolgt hierbei auf einem \gls{kl25z},
einem Mikrocontroller der Kinetis-Serie von Freescale auf Basis des ARM Cortex-M0+.

\begin{figure}[htbp]
		\centering
	\begin{subfigure}[b]{\textwidth}
		\centering
		\includegraphics[width=0.8\textwidth]{./images/board.jpg}
		\caption[FRDM-KL25Z mit externen Sensoren]{FRDM-KL25Z mit externen Sensoren: MPU6050 (blau) und HMC5883L (rot)}
		\label{fig:board}
	\end{subfigure}

	\begin{subfigure}[b]{\textwidth}
		\centering
		\includegraphics[width=0.8\textwidth]{./images/board-mit-logan.jpg}
		\caption[FRDM-KL25Z mit angeschlossenem Logic-Analyzer]{FRDM-KL25Z mit angeschlossenem Logic-Analyzer (rechts). Im Hintergrund links der verwendete UART-zu-USB-Transceiver.}
		\label{fig:board-logan}
	\end{subfigure}
	
	\caption{Prototyp auf dem Breadboard}
	\label{fig:prototype}
\end{figure}


\mainmatter						% Hauptteil
%\setpartpreamble[Position][Breite]{Text}
\setpartpreamble[u, r][6cm]{Dies ist ein Beispiel für eine Part-Präambel.}
\part[]{Hauptteil}
\chapter{Gliederung}

In den report- und book-Klassen steht, im Vergleich zu den article-Klassen als zusätzliche Gliederungseinheit \verb#\chapter[Kurzform]{Langform}# zur Verfügung. 

Kapitel beginnen in der Regel in Büchern auf einer ungeraden, d.\,h. rechten Seite. Will man fortlaufenden Textsatz erreichen und also den Beginn auch auf linken Seiten zulassen, verwendet man die Option \verb#openany# gleich in der Dokumenten-Präambel. Hier finden sich auch andere Optionen zur Regelung der Überschriftengröße und deren Beschriftung.
\chapter{Präambeln}

Durch den Befehl \verb#\setpartpreamble[Position][Breite]{Präambel}# wird zusammen mit der Angabe des Teils (part) zudem der angegebene Text gesetzt. Dies kann z.\,B. eine kurze Inhaltsangabe sein. Ein Beipiel ist unter Hauptteil zu sehen.  Die Präambel wird in eine Box gesetzt, deren Position und Breite angegeben werden kann. Unterbleibt dies, wird sie unterhalb der Überschriften im normalen Blocksatz über den gesamten Textbereich gesetzt.

Eine ähnliche Funktion findet sich auch für Kapitel (chapter). Die Anweisung lautet hier entsprechend \verb#\setchapterpreamble[Position][Breite]{Präambel}#. 

Für ein einleitendes Zitat, ein sog. Diktum bietet das KOMA-Script die Anweisung \verb#\dictum[Urheber]{Spruch}#. Sie wird in der Regel in eine \verb#\setchapterpreamble# oder \verb#\setpartpreamble# gesetzt. Ein Beispiel soll folgen:
\setchapterpreamble[u]{%
\dictum[Luhmann]{Die Klassiker sind Klassiker, weil sie Klassiker sind \dots}}
\chapter{Diktum}

Übrigens wird ohne weitere Angaben ein Drittel der aktuellen Textbreite verwendet. Wie fast alles bei der Verwendung von \LaTeX , kann dies natürlich angepasst werden. Wie das geht und auch alles andere zur Verwendung von Präambeln steht im scrguide 3.\,6.\,2.

Exemplarischer Verweis auf ein Buch mit dieser Aussage, vgl. \cite{Filieri}, außerdem \cite{Tsang}. Dort werden \glspl{mems} erwähnt. So ein \gls{mems} ist toll.

\appendix							% Beginn des Anhangs
\chapter{Schluss}

Für den Schluss ist zu überlegen, wie man den Anhang formatiert haben möchte: Das KOMA-Script kennt den Befehl \verb#\backmatter#. Hierdurch wird die Nummerierung der Gliederungseinheiten im Text und im Inhaltsverzeichnis unterdrückt. Erwartet man die übliche Beschriftung mit "`Anhang A"' bzw. "`A."' so verwendet man den Befehl \verb#\appendix# und verzichte auf \verb#\backmatter# oder setze es zu einem späteren Punkt ein.

Viel Spaß! Für Rückfragen, die diese Vorlage betreffen, stehe ich Ihnen gern in der Mailingliste von TXC zur Verfügung. Ansonsten sind die Dokumente \texttt{lshort}, \texttt{l2tabu}, die \texttt{FAQ der Newsgroup de.text.tex} und natürlich der \texttt{scrguide} immer sehr hilfreich.


\backmatter					% Nachspann 

%% Stichwortverzeichnis anzeigen %%%%%%%%%%%%%%%%%%%%%%%%%%%%%%%%%%%%%%
\printindex

%% Bibliographie unter Verwendung von dinnat %%%%%%%%%%%%%%%%%%%%%%%%%%
%\setbibpreamble{Präambel des Literaturverzeichnisses}		% Text vor dem Verzeichnis
\bibliographystyle{dinat}
\bibliography{./bib/quellen}	% Sie benötigen einen *.bib-Datei

\end{document}