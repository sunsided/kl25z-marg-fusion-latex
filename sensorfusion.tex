%%%%%%%%%%%%%%%%%%%%%%%%%%%%%%%%%%%%%%%%%%%%%%%%%%%%%%%%%%%%%%%%%%%%%%%
%% Optionen zum Layout des Buchs                                     %%
%%%%%%%%%%%%%%%%%%%%%%%%%%%%%%%%%%%%%%%%%%%%%%%%%%%%%%%%%%%%%%%%%%%%%%%
\documentclass[
a4paper,							% alle weiteren Papierformat einstellbar
%landscape,						% Querformat
11pt,								  % Schriftgröße (12pt, 11pt (Standard))
%BCOR1cm,							% Bindekorrektur, bspw. 1 cm
%DIVcalc,							% führt die Satzspiegelberechnung neu aus
%											  s. scrguide 2.4
%oneside,							% einseitiges Layout
%twocolumn,						% zweispaltiger Satz
%openany,							% Kapitel können auch auf linken Seiten beginnen
%halfparskip*,				% Absatzformatierung s. scrguide 3.1
%headsepline,					% Trennline zum Seitenkopf	
%footsepline,					% Trennline zum Seitenfuß
%notitlepage,					% in-page-Titel, keine eigene Titelseite
bibliography=totoc,
chapterprefix,				% vor Kapitelüberschrift wird "Kapitel Nummer" gesetzt
appendixprefix,				% Anhang wird "Anhang" vor die Überschrift gesetzt 
%normalheadings,			% Überschriften etwas kleiner (smallheadings)
%leqno,								% Nummerierung von Gleichungen links
fleqn,								% Ausgabe von Gleichungen linksbündig
%draft								  % überlangen Zeilen in Ausgabe gekennzeichnet
]
{scrbook}

%\pagestyle{empty}		% keine Kopf und Fußzeile (k. Seitenzahl)
%\pagestyle{headings}	% lebender Kolumnentitel  

\usepackage[titletoc,title]{appendix}

%% Deutsche Anpassungen %%%%%%%%%%%%%%%%%%%%%%%%%%%%%%%%%%%%%
\usepackage[utf8x]{inputenc}
\usepackage[T1]{fontenc}
\usepackage[ngerman]{babel}
\usepackage{lmodern}

%\PrerenderUnicode{ü}
%\PrerenderUnicode{Ü}
%\PrerenderUnicode{ä}
%\PrerenderUnicode{Ä}
%\PrerenderUnicode{ö}
%\PrerenderUnicode{Ö}
%\PrerenderUnicode{ß}

%% Falls die automatische Worttrennung in Wörtern mit Umlauten
%% nicht funktionieren sollte oder der Text pixelig aussieht:
%% ==> Installieren Sie die cm-super Fonts (z.B. mit dem mikTeX Package Manager).
%% Eine nicht ganz vollwertige Alternative ist die Verwendung dieses Pakets:
%\usepackage{ae, aeguill}

%% Stichwortverzeichnis %%%%%%%%%%%%%%%%%%%%%%%%%%%%%%%%%%%%%
\usepackage{makeidx}

%% Packages für Grafiken & Abbildungen %%%%%%%%%%%%%%%%%%%%%%
\usepackage{graphicx}
\usepackage{caption}
\usepackage{subcaption} %%Moderne Alternative zu subfig

%% Beachten Sie:
%% Die Einbindung einer Grafik erfolgt mit \includegraphics{Dateiname}
%% bzw. über den Dialog im Einfügen-Menü.
%% 
%% Im Modus "LaTeX => PDF" können Sie u.a. folgende Grafikformate verwenden:
%%   .jpg  .png  .pdf  .mps
%% 
%% In den Modi "LaTeX => DVI", "LaTeX => PS" und "LaTeX => PS => PDF"
%% können Sie u.a. folgende Grafikformate verwenden:
%%   .eps  .ps  .bmp  .pict  .pntg


%% Mathe %%%%%%%%%%%%%%%%%%%%%%%%%%%%%%%%%%%%%
\usepackage{amsmath}
\usepackage{amsfonts}

%% Todo %%%%%%%%%%%%%%%%%%%%%%%%%%%%%%%%%%%%%
\usepackage{todonotes}

%% Farbanpassungen %%%%%%%%%%%%%%%%%%%%%%%%%%%%%%%%%%%%%%%%%%%%%%%%%%%%
\usepackage{xcolor}
\definecolor{unigruen}{rgb}{0,0.5961,0.6314}
\definecolor{unirot}{rgb}{0.373,0.0941,0.1176}

%% Seitenränder %%%%%%%%%%%%%%%%%%%%%%%%%%%%%%%%%%%%%%%%%%%%%%%%%%%%%%%
\usepackage[left=40mm, right=30mm, bottom=30mm]{geometry}

%% Kopf- und Fußzeilen %%%%%%%%%%%%%%%%%%%%%%%%%%%%%%%%%%%%%%%%%%%%%%%%
\usepackage{scrpage2}
\clearscrheadfoot %% löscht standard einstellungen
\ohead{\pagemark} %kopf rechts Seitennummer
%%\ihead{\thechapter\ \thesection\ \sectionmark} %Kopf links
\ihead{\leftmark}
\automark{section} 
\setheadsepline{0.2pt} %% horizontale linie im Kopf
\setfootsepline{0.2pt} %% horizontale linie im Fuß

%% Extended UTF-8 support  %%%%%%%%%%%%%%%%%%%%%%%%%%%%%%%%%%%%%%%%%%%%
\usepackage{ucs}

%% Tabellen  %%%%%%%%%%%%%%%%%%%%%%%%%%%%%%%%%%%%%%%%%%%%%%%%%%%%%%%%%%
\usepackage{booktabs}
%\usepackage{tabularx}

%% Protokolle etc. als Bitfeld  %%%%%%%%%%%%%%%%%%%%%%%%%%%%%%%%%%%%%%%
%% http://texdoc.net/texmf-dist/doc/latex/bytefield/bytefield.pdf
\usepackage{bytefield}


\newcommand{\colorbitbox}[3]{%
\rlap{\bitbox{#2}{\color{#1}\rule{\width}{\height}}}%
\bitbox{#2}{#3}}

\definecolor{lightcyan}{rgb}{0.84,1,1}
\definecolor{lightgreen}{rgb}{0.64,1,0.71}
\definecolor{lightred}{rgb}{1,0.7,0.71}

%% Bibliographiestil %%%%%%%%%%%%%%%%%%%%%%%%%%%%%%%%%%%%%%%%%%%%%%%%%%
%% ftp://ftp.tex.ac.uk/tex-archive/macros/latex/contrib/natbib/natbib.pdf
\usepackage{natbib}

%% Lorem Ipsum %%%%%%%%%%%%%%%%%%%%%%%%%%%%%%%%%%%%%%%%%%%%%%%%%%%%%%%%
%% http://ctan.org/pkg/lipsum
\usepackage{lipsum}

%% Absatz-Einstellungen %%%%%%%%%%%%%%%%%%%%%%%%%%%%%%%%%%%%%%%%%%%%%%%
\usepackage{parskip}

%% Code-Listings %%%%%%%%%%%%%%%%%%%%%%%%%%%%%%%%%%%%%%%%%%%%%%%%%%%%%%
\usepackage[final]{listings}

%% Hyperlinks %%%%%%%%%%%%%%%%%%%%%%%%%%%%%%%%%%%%%%%%%%%%%%%%%%%%%%%%%
%% http://www.tug.org/applications/hyperref/ftp/doc/manual.pdf
\usepackage[backref,bookmarks=true,pdfpagelabels,plainpages=false]{hyperref}

\hypersetup{colorlinks=true}

%% Glossaries  %%%%%%%%%%%%%%%%%%%%%%%%%%%%%%%%%%%%%%%%%%%%%%%%%%%%%%%%%
%% http://ctan.org/pkg/glossaries

%\usepackage[nonumberlist,acronym,toc,section]{glossaries}
\usepackage[acronym]{glossaries}

%% Stichwortverzeichnis anfordern %%%%%%%%%%%%%%%%%%%%%%%%%%%%%%%%%%%%%%
\makeindex

%% Glossar anfordern %%%%%%%%%%%%%%%%%%%%%%%%%%%%%%%%%%%%%%%%%%%%%%%%%%%
\makeglossaries

% Glossar einbinden
\addto{\captionsngerman}{
  \renewcommand{\acronymname}{Abkürzungen}
}

\addto{\captionsngerman}{
  \renewcommand{\glossaryname}{Glossar}
}

\newglossaryentry{Cortex}{name = {Cortex}, description={32bit-Mikrocontrollerarchitektur von ARM, Inc}}
\newglossaryentry{cortex-m0}{name = {Cortex-M0}, description={Mikroprozessor der M0-Serie der ARM \gls{Cortex}-Architektur}}
\newglossaryentry{cortex-m0+}{name = {Cortex-M0+}, description={Mikroprozessor der M0+-Serie der ARM \gls{Cortex}-Architektur}}
\newglossaryentry{CoreSight}{name = {CoreSight}, description={Debug-Schnittstelle in \gls{Cortex}-\glspl{mcu}}}

\newglossaryentry{Kinetis}{name = {Kinetis}, description={32bit-Mikrocontrollertyp von Freescale Semiconductor}}
\newglossaryentry{kl25z}{name = {KL25Z}, description={Mikrocontroller der \gls{Kinetis}-L-Serie von Freescale auf Basis des ARM Cortex-M0+}}
\newglossaryentry{frdm-kl25z}{name = {FRDM-KL25Z}, description={Freescale Freedom Development Board mit KL25Z-\gls{mcu}}}

\newglossaryentry{Gimbal Lock}{name = {Gimbal Lock}, description={Kardanische Blockade eines Systemes bei der Verwendung von Euler'schen Winkeln, die bei ungünstiger Kombination von Rotationen zum Verlust eines Freiheitsgrades führt}}

\newglossaryentry{Kalman-Filter}{name = {Kalman-Filter}, description={Rekursives, lineares Filter zur Schätzung stochastischer Systemparameter, dessen Entwicklung auf Rudolf Emil Kálmán zurückgeht}}
\newglossaryentry{OpenSDA}{name = {OpenSDA}, description={Proprietäre, erweiterbare Programmierschnittstelle von Freescale}}
\newglossaryentry{OpenOCD}{name = {OpenOCD}, description={Quelloffene Programmierschnittstelle}}
\newglossaryentry{JTAG}{name = {JTAG}, description={Schnittstelle für Tests von Controllerschnittstellen und Programmierung}}
\newglossaryentry{elf}{name = {ELF}, description={Executable and Linking Format, eine ausführbare Datei}}
\newglossaryentry{mbed}{name = {mbed}, description={Plattform für die Entwicklung auf Cortex-M-MCUs}}
\newglossaryentry{Thumb2}{name = {Thumb2}, description={Befehlssatz von \glslink{Cortex}{ARM Cortex-M}-Prozessoren}}

\newglossaryentry{Quaternion}{name = {Quaternion}, description={Hamilton-Zahl $\mathbb{H}$ im 4-dimensionalen komplexer Raum, die zur singularitätsfreien Beschreibung von Orientierungen verwendet werden kann}}

\newglossaryentry{hard iron}{name = {Hard-Iron-Effekt}, description={Lineare Verzerrung im Magnetfeld durch Einwirkung "`harter"' ferromagnetischer Metalle, vgl. Soft-Iron-Effekt}}
\newglossaryentry{soft iron}{name = {Soft-Iron-Effekt}, description={Nichtlineare Verzerrung im Magnetfeld durch Einwirkung "`weicher"' ferromagnetischer Metalle, vgl. Hard-Iron-Effekt}}

\newglossaryentry{systick}{name = {SysTick}, description={System Tick Interrupt: Konfigurierbarer Timerinterrupt des Cortex-Cores zur Zeitmessung, welcher die Realisierungen von Echtzeitanwendungen unterstützen soll.}}

\newglossaryentry{bit banding}{name = {Bit-Banding}, description={Mapping von Speicheradresse auf einzelne Bits einer anderen Adresse}}

\newglossaryentry{tilt compensation}{name = {Tilt Compensation}, description={Korrektur der dreidimensionalen Neigung des Magnetometer-Messvektors mittels zusätzlicher Ausrichtungsdaten zur Bestimmung der magnetischen Nordrichtung in der zweidimensionalen Ebene}}


\newglossaryentry{Komplementaerfilter}{name = {Komplementärfilter}, description={Wichtungsfilter zwischen Beschleunigungs- und Drehratensensor zur Korrektur der spezifischen Drift- und Rauschmerkmale}}

\newglossaryentry{triad}{name = {TRIAD}, description={Algorithmus zur Orientierungsdeterminierung aus drei Vektorbeobachtungen}}

\newglossaryentry{q16}{name = {Q16}, description={32bit-Zahlenformat für Festkommazahlen mit jeweils 16 bit für Vor"- und Nachkommaanteil}}

\newglossaryentry{mpug}{name = {MPU}, description={Motion Processing Unit, eine durch Kopplung externer Sensoren erweiterbare Sensorserie von Invensense zur Verarbeitung und Fusion von Inertialdaten.}}

\newglossaryentry{i2cg}{name = {I\textsuperscript{2}C}, description={(auch IIC) Inter-Integrated Circuit. Ein von Philips entworfenes Bussystem für Inter-Chip-Kommunikation}}

\newglossaryentry{imug}{name = {IMU}, description={Inertial Measurement Unit, dt. Inertialmesseinheit; Ein Sensorsystem bestehend aus mehreren kombinierten Inertialsensoren}}

\newacronym{i2c}{I\textsuperscript{2}C}{Inter-Integrated Circuit\glsadd{i2cg}}
\newacronym{mpu}{MPU}{Motion Processing Unit\glsadd{mpug}}

\newacronym{spi}{SPI}{Serial Peripheral Interface}

\newacronym{dmp}{DMP}{Digital Motion Processor}
\newacronym{mac}{MAC}{Multiply and Accumulate}

%\newacronym{i2c}{I2C}{Inter-Integrated Circuit}

\newacronym[
	\glsshortpluralkey={MEMS},
	\glslongpluralkey={mikroelektromechanische Systeme}
]{mems}{MEMS}{mikroelektromechanisches System}

\newacronym{marg}{MARG}{Magnetic, Angular Rate and Gravitational}

\newacronym[
	\glsshortpluralkey={IMUs},
	\glslongpluralkey={Inertial Measurement Units}
]{imu}{IMU}{Inertial Measurement Unit\glsadd{imug}}

\newacronym{hci}{HCI}{Human-Computer Interaction}
\newacronym{hid}{HID}{Human Interface Device}
\newacronym{cmsis}{CMSIS}{Cortex Microcontroller Software Interface Standard}
\newacronym{cmsis-dap}{CMSIS-DAP}{\gls{cmsis} Debug Access Port}
\newacronym[
	\glsshortpluralkey={MCUs},
	\glslongpluralkey={Microcontroller Units}
]{mcu}{MCU}{Microcontroller Unit}
\newacronym{ocd}{OCD}{On-Chip-Debugger}
\newacronym{msd}{MSD}{Mass Storage Device}
\newacronym{usb}{USB}{Universal Serial Bus}
\newacronym{gdb}{GDB}{GNU Debugger}
\newacronym{dcm}{DCM}{Direction Cosine Matrix}
\newacronym{swd}{SWD}{Serial Wire Debug}
\newacronym{request}{REQUEST}{Recursive Quaternion Estimator}
\newacronym{quest}{QUEST}{Quaternion Estimator}

\newacronym{uart}{UART}{Universal Asynchronous Receiver Transmitter}

\newacronym[
	\glsshortpluralkey={GPIOs},
	\glslongpluralkey={General-Purpose I/Os}
]{gpio}{GPIO}{General-Purpose I/O}


\newacronym{soh}{SOH}{Start of Header}
\newacronym{eot}{EOT}{End of Transmission}
\newacronym{esc}{ESC}{Escape}

\newacronym{p2pp}{P2PP}{Point-to-Point Protocol}

\newacronym{wnu}{WNU}{West-North-Up}
\newacronym{enu}{ENU}{East-North-Up}
\newacronym{ecef}{ECEF}{Earth-Centered, Earth-Fixed}
\newacronym{fpu}{FPU}{Floating-Point Unit}

\newacronym{bme}{BME}{Bit Manipulation Engine}
\newacronym{bfi}{\texttt{BFI}}{Bit Field Insert}
\newacronym{lac}{\texttt{LAC}}{Load-and-Clear}
\newacronym{las}{\texttt{LAS}}{Load-and-Set}
\newacronym{ubfx}{\texttt{UBFX}}{Usigned Bit Field Extract}
\newacronym{pll}{PLL}{Phase-Locked Loop}

%% PDF-spezifische Einstellungen %%%%%%%%%%%%%%%%%%%%%%%%%%%%%%%%%%%%%%
\hypersetup{
						pdfpagemode=UseOutlines
						}

\makeatletter
\AtBeginDocument{
	\let\oldand\and\def\and{and }
	\hypersetup{
							pdftitle = {\@title},
							pdfauthor = {\@author},
							pdfsubject = {\@subject},
							pdfkeywords = {BHT, ARM, Cortex-M0, Kalman-Filter, MARG, IMU, Sensor Fusion, Signal Processing}
						}
	\let\and\oldand
}
\makeatother

%% Dokumenten-Titel und verwandtes %%%%%%%%%%%%%%%%%%%%%%%%%%%%%%%%%%%%%
\subject{Fortgeschrittene ARM-Microcontroller-Programmierung}
\title{Orientierungserkennung durch MARG-Sensorfusion mittels eines Kalman-Filters auf dem Freescale Kinetis KL25Z}
\author{Markus Mayer
				\and
				Julian Dombrow}
\publishers{Beuth-Hochschule für Technik, Berlin}
\date{\today}

\definecolor{mygray}{rgb}{0.5,0.5,0.5}

% Code-Stile
\lstdefinestyle{lolbash}{
    language={bash}, 
		basicstyle=\ttfamily\tiny,
		frame=single,
    moredelim=**[is][\slshape]{`}{`},
    moredelim=**[is][\color{orange}]{°}{°}
}

\lstdefinestyle{lolc}{
    language={C}, 
		basicstyle=\ttfamily\tiny,
		frame=single,
    moredelim=**[is][\slshape]{`}{`},
    moredelim=**[is][\color{orange}]{°}{°},
		showstringspaces=false,
		keywordstyle=\bfseries\color{green!40!black},
		commentstyle=\itshape\color{purple!40!black},
		identifierstyle=\color{black},
		stringstyle=\color{blue},
		escapeinside={\%*}{*)},
		breakatwhitespace=true,
		breaklines=true,
		morekeywords={uint8_t, uint16_t, uint_fast8_t, uint_least8_t, uint_fast16_t, uint_least8_t,
									mma8451q_confreg_t, mma8451q_acc_t,
									mpu6050_confreg_t},
		numbers=left,
		numbersep=5pt,
		numberstyle=\tiny\color{mygray}
}

\lstdefinestyle{lolcsharp}{
    language={Java}, 
		basicstyle=\ttfamily\tiny,
		frame=single,
		keywordstyle=\bfseries\color{green!40!black},
		commentstyle=\itshape\color{purple!40!black},
		identifierstyle=\color{black},
		stringstyle=\color{blue},
		morekeywords={readonly, const, event},
		breakatwhitespace=true,
		breaklines=true,
		numbers=left,
		numbersep=5pt,
		numberstyle=\tiny\color{mygray},
		escapeinside={\%*}{*)}
}

\lstdefinestyle{lolmatlab}{
    language={Matlab}, 
		basicstyle=\ttfamily\tiny,
		frame=single,
		keywordstyle=\bfseries\color{green!40!black},
		commentstyle=\itshape\color{purple!40!black},
		identifierstyle=\color{black},
		stringstyle=\color{blue},
		morekeywords={readonly, const, event},
		breakatwhitespace=true,
		breaklines=true,
		numbers=left,
		numbersep=5pt,
		numberstyle=\tiny\color{mygray},
		escapeinside={\%*}{*)}
}

\begin{document}

\cleardoublepage

\pagenumbering{gobble} % Seitenzahl entfernen! fix für pdflatex "identifier ... has been already used"'
\clearpage
\setlength{\voffset}{-15mm}
%\setlength{\hoffset}{10mm}
\thispagestyle{empty}

\makeatletter
\begin{titlepage}

\begin{flushright}
\includegraphics[height=2cm]{./images/Beuth_Logo_basis.pdf}
\end{flushright}

\textcolor{unigruen}{
	{\hrule height 1.5pt }
}

\begin{centering}
\vspace{2cm}
\textsc{\huge{Beuth Hochschule \\ für Technik Berlin}}\\
\vspace{2cm}
\textsc{\large{Fachbereich VI \\ Technische Informatik \\ Master of Embedded Systems}}\\
\vspace{0.6cm}
\textsc{\large{Fach: \\ Fortgeschrittene ARM-Microcontroller-Programmierung}}\\
\textsc{\large{WS 2013/14}}\\
\vspace{0.6cm}
\textsc{\large{Projekt: \\ \@title}}\\
\end{centering}

\vspace{2cm}
\begin{table}[h]
\begin{center}
\textsc{
\begin{tabular}{l l}
	\@author
\end{tabular}
}
\end{center}
\end{table}

\vfill

\begin{center}
\textsc{\large{Berlin, den \today}}
\end{center}

\end{titlepage}
\makeatother

\cleardoublepage
%% Angaben zur Standardformatierung des Titels %%%%%%%%%%%%%%%%%%%%%%%%
%\titlehead{Titelkopf}
%\thanks{Fußnote}					% entspr. \footnote im Fließtext

%% Rückseite der Titelseite %%%%%%%%%%%%%%%%%%%%%%%%%%%%%%%%%%%%%%%%%%%
\uppertitleback{Markus Mayer, B.Eng.\\Matr.-Nr. 798481\\\url{widemeadows@gmail.com}}
%\lowertitleback{Titelrückseitenfuß}

%% Widmungsseite %%%%%%%%%%%%%%%%%%%%%%%%%%%%%%%%%%%%%%%%%%%%%%%%%%%%%%
\dedication{Widmung}

\pagenumbering{Alph}
\maketitle 						% Titelei wird erzeugt
\thispagestyle{empty}

%% Erzeugung von Verzeichnissen %%%%%%%%%%%%%%%%%%%%%%%%%%%%%%%%%%%%%%%
\tableofcontents			% Inhaltsverzeichnis
\thispagestyle{empty}

\listoffigures				% Abbildungsverzeichnis
\thispagestyle{empty}

\lstlistoflistings
\thispagestyle{empty}

\listoftables				% Tabellenverzeichnis
\thispagestyle{empty}

\printglossaries		% Glossar
\thispagestyle{empty}

%\printglossary[type=symbolslist,title=Symbolverzeichnis]
%\thispagestyle{empty}

%% Der Text %%%%%%%%%%%%%%%%%%%%%%%%%%%%%%%%%%%%%%%%%%%%%%%%%%%%%%%%%%%
\frontmatter					% Vorspann (z.B. römische Seitenzahlen)
\chapter{Einleitung}

Ein häufiges Problem in der mobilen Robotik und eine zunehmende Notwendigkeit in der \gls{hci} --- etwa bei der Entwicklung 
von modernen Controllern für Unterhaltungssysteme --- ist die robuste Erkennung der Orientierung eines
beweglichen Systems, sei es einer autonomen Plattform (z.B. Roboter, Quadrocopter, ...) oder eines Eingabegerätes
(z.B. Nintendo Wiimote).

Während hochpräzise Inertialsensoren (\gls{imu}) wie Gyroskope in ihrer ursprünglichen Bauart rein mechanische Systeme sind, halten seit
einigen Jahren zunehmend \glspl{mems} Einzug in die Sensortechnik und stehen dabei in unterschiedlichen Genauigkeits-
und Kostengraden sowohl für Low-End, als auch High-End-Anwendungen zur Verfügung.

Hierbei werden grundlegend zwei Typen von Sensoren unterschieden: Inertialsensoren (\glspl{imu}), welche auf der Messung der Trägheit
des Systems beruhen --- klassische Vertreter sind der Accelerometer zur Messung von Beschleunigungen, 
sowie der Gyrosensor zur Messung von Drehraten --- als auch magnetische Sensoren, welche die Auswirkungen eines Magnetfeldes auf das 
System messen. Werden solche drei Sensoren in einem System kombiniert, spricht man von einem sogenannten \gls{marg}-Sensorsystem.

Dieses Projekt beschäftigt sich mit der Fusionierung eines \gls{marg}-Sensorsystems zur robusten Orientierungserkennung. Ziel ist
es, ein Maß für die Ausrichtung des Systemes im Raum zu ermitteln, welches

\begin{itemize}
	\item unanfällig gegenüber Messrauschen der Sensoren,
	\item stabil gegenüber externen Beschleunigungen und
	\item frei von Singulatitäten (\gls{Gimbal Lock}) ist.
\end{itemize}

Hierbei wird ein reguläres \gls{Kalman-Filter} zum Einsatz auf einem \gls{cortex-m0} entwickelt, welches adaptiv auf vorliegende
Messwerte reagiert und für den Echtzeiteinsatz geeignet ist. Die Implementierung erfolgt hierbei auf einem \gls{kl25z},
einem Mikrocontroller der Kinetis-Serie von Freescale auf Basis des ARM Cortex-M0+.

\begin{figure}[htbp]
		\centering
	\begin{subfigure}[b]{\textwidth}
		\centering
		\includegraphics[width=0.8\textwidth]{./images/board.jpg}
		\caption[FRDM-KL25Z mit externen Sensoren]{FRDM-KL25Z mit externen Sensoren: MPU6050 (blau) und HMC5883L (rot)}
		\label{fig:board}
	\end{subfigure}

	\begin{subfigure}[b]{\textwidth}
		\centering
		\includegraphics[width=0.8\textwidth]{./images/board-mit-logan.jpg}
		\caption[FRDM-KL25Z mit angeschlossenem Logic-Analyzer]{FRDM-KL25Z mit angeschlossenem Logic-Analyzer (rechts). Im Hintergrund links der verwendete UART-zu-USB-Transceiver.}
		\label{fig:board-logan}
	\end{subfigure}
	
	\caption{Prototyp auf dem Breadboard}
	\label{fig:prototype}
\end{figure}


\mainmatter						% Hauptteil
%\setpartpreamble[Position][Breite]{Text}
\setpartpreamble[u, r][6cm]{Dies ist ein Beispiel für eine Part-Präambel.}
\part[]{Entwicklungsumgebung}
\chapter{Entwicklungsumgebung}

\section{Freescale Kinetis KL25Z}

Auf jedem Freescale Freedom-Entwicklungsboard wie dem \gls{frdm-kl25z} ist  außerdem ein \gls{ocd} verbaut, welcher zusätzlich zum Debugging die 
Fähigkeit besitzt, die \gls{mcu} zu programmieren. Freescale nennt diese Einheit \gls{OpenSDA} (ganz im Sinne der "`Freedom"'-Benamung der Boards), 
die proprietärer Natur ist, jedoch die zusätzliche Beschaffung eines externen Programmers erspart.

\section{Freescale OpenSDA}

Der \gls{OpenSDA}-Chip auf dem  kommuniziert dabei mit der im \gls{Cortex}-Kern integrierten \gls{CoreSight}-Einheit über das \gls{JTAG}-Protokoll.

\begin{figure}[htbp]
		\centering
		\includegraphics{./images/opensda-block-diagram.png}
    \caption[OpenSDA-Blockdiagramm]{OpenSDA-Blockdiagramm}
		Quelle: \texttt{http://mcuoneclipse.com/2012/09/20/opensda-on-the-freedom-kl25z-board/}
    \label{fig:cmsis-dap}
\end{figure}

Eine Gegenüberstellung der alternativen SDA-Firmwares ohne Anspruch auf Vollständigkeit kann Tabelle \ref{tab:opensda-options} entnommen werden.

\begin{table}
\begin{tabular}{llll} 
\toprule
Gegenüberstellung von OpenSDA-Firmwares\\  
\midrule 
Name & Art & Breakpoints & UART \\ 
\midrule 
OpenSDA & proprietär & hardware & ja\\
CMSIS-DAP & open source & hardware & nein \\
Segger J-Link & proprietär & unbegrenzt & nein \\ 
\bottomrule
\end{tabular}
\caption{Gegenüberstellung von OpenSDA-Firmwares}
\label{tab:opensda-options}
\end{table}

\subsection{CMSIS-DAP}

Eine alternative (und quelloffene) Firmware für den \gls{OpenSDA}-Chip ist die \gls{cmsis-dap}-Firmware.

--> Verwendung mit \gls{OpenOCD}

\begin{figure}[htbp]
		\centering
		\includegraphics[width=0.8\textwidth]{./images/cmsis_dap_interface.png}
    \caption[CMSIS-DAP]{CMSIS-DAP}
		Quelle: \texttt{http://nimblemachines.com/cmsis-dap/}
    \label{fig:cmsis-dap}
\end{figure}


\subsection{Segger J-Link}

\lipsum[2]


\section{Freescale CodeWarrior}

\lipsum[3]

\section{SysProgs VisualGDB}

\lipsum[4]
\chapter{Sensoren}

Alle in diesem Projekt verwendeten Sensoren verfügen über eine \gls{i2c}-Schnittstelle, über welche die Kommunikation abgewickelt wurde. Einige Module, wie die MPU6050 (siehe Abschnitt~\ref{sec:mpu6050})
verfügen darüberhinaus über Interrupt-Leitungen, welche das Vorliegen neuer Messwerte signalisieren.

\section{MMA8451Q - Accelerometer}

Der Freescale MMA8451Q ist der auf dem Freedom-Board verbaute Beschleunigungssensor. Er bietet eine Auflösung von X bei einer maximalen Samplingrate von Y. 

\subsection{Treiber}

\section{MPU6050 - Accelerometer und Gyrosensor}
\label{sec:mpu6050}

Die Invensense MPU6050 ist eine \gls{imu}, die einen Beschleunigungs- und einen Drehratensensor vereinigt. Diese \gls{mpu} umfasst sowohl einen Beschleunigungs-, als auch einen Drehratensensor und bietet die Möglichkeit, über einen als slave betriebenen, zusätzlichen \gls{i2c}-Bus weitere Sensoren zu verarbeiten, wodurch Aufwand vom Hauptcontroller abfällt. Der \gls{mpu} zugeschaltet ist ein sog. \gls{dmp}, welcher die internen Daten, sowie die über den slave-Bus bezogenen externen Daten fusionieren kann. Während die Ergebnisregister des \gls{dmp} frei zugänglich sind, sind die Inhalte dieser Felder nicht dokumentiert. Der Zugriff auf sie erfolgt stattdessen --- sofern erwünscht --- über eine proprietäre MotionsApps-Firmware\footnote{\url{http://www.invensense.com/developers/forum/viewtopic.php?f=3&t=142}}.

\subsection{Treiber}

\subsection{Lessons Learned}

Bei der Kommunikation mit der MPU6050 trat das Problem auf, dass bei aktiviertem Interrupt-Signal und reduzierter Samplingrate nach einem Kaltstart keinerlei nennenswerte Verzögerung (d.h. nur wenige Millisekunden \todo{Verifizieren}) zwischen aufienander folgenden Interruptsignalen festzustellen war. 
Dies führte zu der Problematik, dass die durch den Interrupt-Handler freigeschaltete Routine zum Beziehen der Sensordaten über \gls{i2c} und die anschließende Verarbeitung direkt durch einen erneuten Interrupt unterbrochen wurde. Da die serielle Ausgabe der Werte nur mit deutlich geringerer Taktung laufen kann, führte dies zu einem Engpass am vorgeschalteten Ringpuffer, wodurch die Verarbeitung deutlich inperformant wurde.

Wurde das System jedoch durch einen Reset (d.h. durch einen Warmstart) neu initialisiert, verlief die Kommunikation wie gewünscht. Durch einen weiteren Kaltstart konnte der Zustand erneut herbeigeführt werden.

\todo{Bild mit Freifeuer vom Logan}

Dieses Problem war erst dadurch zu beheben, dass zu Beginn der Konfiguration der \gls{imu} die interne Takteinheit deaktiviert wurde, um sie dann im Zuge der folgenden Konfiguration erneut auf den gewünschten Betriebsmodus zu stellen.

\todo{Bild ohne Freifeuer vom Logan}

\section{HMC5883L - Magnetometer}

Der HMC5883L von Honeywell liefert bei einer Auflösung von 12 Bit (inkl. Vorzeichen) bei einem Wertebereich von $\pm$ 8 Gauss, wodurch er für den Einsatz in Präsenz starker lokaler Magnetfelder geeignet ist.

\subsection{Treiber}

\subsection{Lessons Learned}

Bei der Auswertung der Sensordaten des HMC5883L traten zwei kleine bis mittlere Komplikationen auf. Eines dieser Probleme lag in der Ansteuerung des Sensors, 
das andere dagegen in der Interpretation der Messwerte begründet.

\subsubsection{Reihenfolge der Messwerte}

Während in sämtlichen anderen Sensoren die Messwerte in üblicher $X$,$Y$,$Z$-Reihenfolge vorlagen, speichert der HMC5883L die Messwerte in der Reihenfolge $X$,$Z$,$Y$. Dies ist aus dem Datenblatt (\citealp{hmc5883l}) nur
dann ersichtlich, wenn die im Datenblatt vorliegende --- sehr kurze --- Registerliste (S. 11) sehr genau beachtet wird. Im Gegensatz zu der dort aufgeführten Reihenfolge, verweisen sämtlichen
anderen Stellen des Datenblattes (insbesondere im Abschnitt Data Output Registers, S. 15) auf die reguläre Anordnung.

\subsubsection{Interpretation der Messergebnisse}

\begin{figure}[htbp]
		\centering
		\begin{subfigure}[b]{0.5\textwidth}
			\centering
			\includegraphics[width=\textwidth]{./images/earth_magnetic_field_poles_shutterstock.jpg}
			\caption{Populistische Darstellung \\ Quelle: Shutterstock}
		\end{subfigure}%
		~
		\begin{subfigure}[b]{0.5\textwidth}
			\centering
			\includegraphics[width=\textwidth]{./images/Dipole_field_wikibooks.jpg}
			\caption{Vereinfachte realistische Darstellung \\ Quelle: Wikibooks}
		\end{subfigure}%
		\caption[Darstellungen des Erdmagnetfeldes]{Darstellungen des Erdmagnetfeldes.\\Links: Die Feldlinien treffen sich an den Polen.\\Rechts: Die Feldlinien durchdringen den Erdmantel.}
\end{figure}
\chapter{Sonstiger Code}

\section{Serielle Übertragung}

\subsection{Streamingmodi}
\label{subsec:streamingmodi}

Die Firmware ist in der Lage, Sensor- und Fusionsdaten in verschiedenen Modi über die serielle Schnittstelle auszugeben. Zur Aktivierung dieser Modi siehe Abschnitt~\ref{subsubsec:transmission_host_to_client}.

\subsubsection{Sensorrohdaten (Modus 0)}

Im Rohdatenmodus (Prefix \emph{0\textsubscript{dec}}) werden die ungefilterten Sensordaten des MPU6050-Accelerometers, sowie des HMC5883L-Magnetometers übertragen. Zur Unterstützung der Verarbeitung
und um Unabhängigkeit von der Konfiguration zu erreichen, werden die Daten konvertiert in das \gls{q16}-Format übertragen.

\begin{bytefield}[bitwidth=1.2em]{25}
	\bitheader{0-24} \\
	\colorbitbox{lightgreen}{1}{0} & 
	\bitbox{4}{A\textsubscript{x}} & \bitbox{4}{A\textsubscript{y}} & \bitbox{4}{A\textsubscript{z}} &
	\bitbox{4}{M\textsubscript{x}} & \bitbox{4}{M\textsubscript{y}} & \bitbox{4}{M\textsubscript{z}}% \\
	%\bitbox[t]{1}{} & \bitbox[t]{6}{$\underbrace{\hspace{6.6em}}_{\text{\normalsize Accelerometer}}$} & \bitbox[t]{6}{$\underbrace{\hspace{6.6em}}_{\text{\normalsize Magnetometer}}$}
\end{bytefield}

Dieser Modus wird auch aktiv, wenn kein gültiger Modusbefehl erkannt wurde.

\subsubsection{Roll-Pitch-Yaw (Modus 42)}

In diesem Modus (Prefix \emph{42\textsubscript{dec}}) werden die aus der \gls{dcm} extrahierten \textsc{Euler}'schen Winkel\todo{howto - Verweis auf Fusion!} (Tait-Bryan $X\text{--}Y\text{--}Z$) übertragen.
Die Winkel liegen im \gls{q16}-Format vor.

\begin{bytefield}[bitwidth=1.2em]{13}
	\bitheader{0-12} \\
	\colorbitbox{lightgreen}{1}{42} & 
	\bitbox{4}{Roll $\theta$} & \bitbox{4}{Pitch $\phi$} & \bitbox{4}{Yaw $\psi$}
\end{bytefield}

\subsubsection{Orientierungs-Quaternion (Modus 43)}

In diesem Modus (Prefix \emph{43\textsubscript{dec}}) wird zur effizienten Übertragung der singularitätsfreien Orientierung anstelle der vollständigen \gls{dcm} ein
Orientierungs-Quaternion übertragen\todo{howto - Verweis auf Fusion!}, wobei die Komponenten im \gls{q16}-Format vorliegen.

\begin{bytefield}[bitwidth=1.2em]{17}
	\bitheader{0-16} \\
	\colorbitbox{lightgreen}{1}{43} & 
	\bitbox{4}{W} & \bitbox{4}{X} & \bitbox{4}{Y} & \bitbox{4}{Z}
\end{bytefield}

\subsubsection{Orientierungs-Quaternion und Roll-Pitch-Yaw (Modus 44)}

Dieser Modus (Prefix \emph{44\textsubscript{dec}}) ist eine Kombination aus den Modi 42 und 43 und kann verwendet werden, um die eigenständige Berechnung der \textsc{Euler}'schen Winkel
auf dem Hostrechner zu ersparen.

\begin{bytefield}[bitwidth=1.2em]{29}
	\bitheader{0-28} \\
	\colorbitbox{lightgreen}{1}{44} & 
	\bitbox{4}{W} & \bitbox{4}{X} & \bitbox{4}{Y} & \bitbox{4}{Z} &
	\bitbox{4}{Roll $\theta$} & \bitbox{4}{Pitch $\phi$} & \bitbox{4}{Yaw $\psi$}
\end{bytefield}

\subsection{Übertragungsprotokoll}

\subsubsection{Übertragung Host zu Firmware}
\label{subsubsec:transmission_host_to_client}

Da die Firmware keine komplexen Befehle, sondern lediglich Moduswechsel kennt, bestehen Kommandos vom Host aus lediglich einem Byte, welches dem zu aktivierenden Modus entspricht. 
Diese Modi sind in Abschnitt~\ref{subsec:streamingmodi} (ab Seite~\pageref{subsec:streamingmodi}) definiert. Der vom Host zu sendende Wert entspricht dabei dem jeweiligen Prefix.

\subsubsection{Übertragung Firmware zu Host}

Hinsichtlich der seriellen Übertragung von der Firmware an den Hostrechner per \gls{uart} wurde ein Kommunikationsprotokoll (genannt \gls{p2pp}) entworfen,
um die Nachsynchronisierbarkeit der Daten zu gewährleisten.

Den eigentlichen Daten vorgeschaltet ist eine frei definierbare Präambel. In der umgesetzten Implementierung besteht diese aus den Bytes 
\texttt{218}, \texttt{122} (\texttt{0xDA7A}, Hexspeak für "`DATA"'). Die Präambel wird vom Dekoder selbst ignoriert, kann jedoch verwendet werden, 
um Zeilenumbrüche oder Markierungen in einem Terminalprogramm zu erzwingen.

\begin{bytefield}[bitwidth=4em]{2}

	\bitheader{0-1} \\
	
	\begin{rightwordgroup}{Präambel}
		\bitbox{1}{\texttt{0xDA}} & \bitbox{1}{\texttt{0x7A}}
	\end{rightwordgroup} 
	
\end{bytefield}

Das Übertragungsprotokoll ist definiert durch drei Oktette: Den Startmarker \emph{S} = \texttt{0x01} (\gls{soh}), 
der Länge der Nutzdaten \emph{L}, sowie dem Endmarker \emph{E} = \texttt{0x04} (\gls{eot}).

Das eigentliche Datenpaket kann dabei variabler Länge (bis einschließlich 255 Byte) sein, wobei den Daten ein
benutzerdefiniertes Prefix \emph{P} vorgeschaltet sein kann, um das Codieren von Pakettypen zu erleichtern.
Da die Länge des Prefixes mit im Längenfeld \emph{L} kodiert ist --- es handelt sich nur um eine logische
Trennung, die nicht protokollspezifisch ist --- reduziert sich die maximale (virtuelle) Anzahl der Daten"-ok"-tette entsprechend.
Einen Verwendungszweck der benutzerdefinierten Prefixe stellen die in Abschnitt~\ref{subsec:streamingmodi} definierten 8 bit 
breiten Moduscodes dar.

\begin{bytefield}[bitwidth=1.1em]{32}

	\bitheader{0-31} \\
	
	%\begin{rightwordgroup}{Header}
		\colorbitbox{lightcyan}{1}{S} & \colorbitbox{lightcyan}{1}{L} & \bitbox{1}{P} &	\bitbox[rt]{29}{} \\
	% \end{rightwordgroup} \\
	
	\wordbox[lr]{1}{Nutzdaten} \\
	\skippedwords \\
	\wordbox[lr]{1}{} \\
	\bitbox[lrb]{31}{} & \colorbitbox{lightcyan}{1}{E}

\end{bytefield}

Die entsprechende Sendefunktion im \gls{p2pp}-Encoder ist wie folgt implementiert:

\begin{lstlisting}[style=lolc]
void P2PPE_TransmissionPrefixed(
	register const uint8_t*const prefix, 
	register uint8_t prefixCount, 
	register const uint8_t*const data, 
	register uint8_t dataCount, 
	register void (*sendHandler)(uint8_t dataByte))
{
	/* send the preamble */
	for (int i=0; i<DEFAULT_PREAMBLE_LENGTH; ++i)
	{
		sendHandler(default_preamble[i]); /* 0xDA, 0x7A */
	}
	
	/* send the header */
	sendHandler(SOH);
	sendHandler(dataCount + prefixCount);
	
	/* send prefix */
	for (int i=0; i<prefixCount; ++i)
	{
		register uint8_t byte = prefix[i];
		encodeAndSend(byte, sendHandler);
	}
	
	/* send data */
	for (int i=0; i<dataCount; ++i)
	{
		register uint8_t byte = data[i];	
		encodeAndSend(byte, sendHandler);
	}
	
	/* send data frame end */
	sendHandler(EOT);
}
\end{lstlisting}

Da das Paketende durch den \gls{eot}-Marker gesetzt ist, darf dieser Wert (sowie das Zeichen \gls{esc}) in den
Nutzdaten nicht vorkommen. Tritt ein solcher Wert auf, wird das wird \gls{esc} (\texttt{0x1B}) eingefügt
und der ursprüngliche Inhalt mit dem Wert \texttt{0x42} ver\textsc{xor}t übertragen.

\begin{lstlisting}[style=lolc]
static inline void encodeAndSend(register uint8_t byte, register void (*sendHandler)(uint8_t dataByte))
{
	if ( EOT == byte || ESC == byte)
	{
		sendHandler(ESC);
		byte ^= ESC_XOR; /* ESC_XOR = 0x42 */
	}
	
	sendHandler(byte);	
}
\end{lstlisting}

\subsection{P2PP-Decoder}
\label{subsec:p2pp_decoder}

Decoder für das \gls{p2pp}-Protokoll wurden sowohl in MATLAB, als auch in C\# geschrieben. Beispielimplementierungen in MATLAB können in 
Listings~\ref{lst:matlab-p2ppdecoder-setup} und \ref{lst:matlab-p2ppdecoder} gefunden werden, eine Implementierung für C\# findet sich in Listing~\ref{lst:csharp-p2ppdecoder}.

Als wesentlich für den performanten Empfang von Daten über die serielle Schnittstelle in MATLAB stellten sich zwei Dinge heraus: Der Dekoder darf\footnote{Bis einschließlich MATLAB R2013b}
nicht als Klasse implementiert sein. Dies bringt extreme Performanceeinbußen mit sich bringt, was der internen Handhabung von Klassen in MATLAB geschuldet ist.

Zum anderen sollte die (üblicherweise empfohlene) Verwendung des \texttt{serial}-Objektes zum Lesen mittels \texttt{fread} vermieden werden, da diese Funktion für
Parameter jenes Typs inperformant implementiert ist\footnote{Dies kann in MATLAB eingesehen werden.}. Um die (oft überflüssige, aber aufwendige) Konstruktion von Fehlerstrings und wiederkehrende
Java-Calls zum Beziehen des Sockets zu vermeiden, schlagen wir den in Listing~\ref{lst:matlab-serial-conf} und \ref{lst:matlab-serial-grab} dargestellten Nutzungsweg vor.

Zuerst wird das Serial-Objekt wie gehabt mittels \texttt{serial(...)} bezogen. In Listing~\ref{lst:matlab-serial-conf} wird exemplarisch der Port \texttt{COM3} mit 115,2 kbaud und 
der üblichen Einstellung "`8-N-1"' verwendet. Ein Interner Puffer von 1024 Byte wird eingerichtet und der Lesemodus auf \emph{continuous} geschaltet, um unnötiges Polling (Modus \emph{manual}) zu vermeiden.
Anschließend wird der Socket geöffnet.

\begin{lstlisting}[style=lolmatlab,caption={Konfiguration des Serial-Ports in MATLAB},label=lst:matlab-serial-conf]
s = serial('COM3', ...
        'FlowControl', 'none', ...
        'BaudRate', 115200, ...
        'DataBits', 8, ...
        'Parity', 'none', ...
        'StopBits', 1, ...
        'TimeOut', 1, ...
        'InputBufferSize', 1024, ...
        'ReadAsyncMode', 'continuous', ...
        'Terminator', 0, ...
        'BytesAvailableFcnCount', 12, ...
        'BytesAvailableFcnMode', 'byte' ...
        );

fopen(s);
\end{lstlisting}

Anstelle nun direkt mittels \texttt{fread(s)} Daten anzufordern, wird einmalig mittels \texttt{igetfield} eine Referenz auf das native Java-Objekt bezogen. Dieses
kann dann mittels einer überladenen Version von \texttt{fread} ausgewertet werden (vgl. Listing~\ref{lst:matlab-serial-grab}).

\begin{lstlisting}[style=lolmatlab,caption={Verwendung des Serial-Ports in MATLAB},label=lst:matlab-serial-grab]
% zu Beginn der Funktion, nach fopen(s)
sjobject = igetfield(s, 'jobject');

% in der Empfangsschleife
bulkSize = 80; % max. 80 Byte pro Lesevorgang
rawout = fread(sjobject, bulkSize, 0, 0); % ..., 0, 0 steht fuer "unsigned integer, 8 bit"
bytes = typecast(rawout(1), 'uint8');     % dies hat jedoch keinerlei Einfluss, daher manueller cast

% schliessen wie ueblich
fclose(s);
delete(s);
\end{lstlisting}

\begin{lstlisting}[style=lolmatlab,caption={Setup des \gls{p2pp}-Decoders in MATLAB},label=lst:matlab-p2ppdecoder-setup]
function prepareProtocolDecode
        
    % Definitions
    global SOH EOT ESC ESC_XOR DA TA
    SOH     = uint8(1);
    EOT     = uint8(4);
    ESC     = uint8(27);
    ESC_XOR = uint8(66);
    DA      = uint8(218);
    TA      = uint8(122);

    global state
    state = 0;
    
    % Decoding variables
    global dataLength data dataBytesRead escapeDetected dataReady lastByte
    lastByte = NaN;
    dataLength = 0;
    data = [];
    dataBytesRead = 0;
    escapeDetected = false;
    dataReady = false;
    
    % Data counters
    global sensorDataCount RPY
    RPY = 1;
    sensorDataCount = zeros(3,1);
end
\end{lstlisting}

\begin{lstlisting}[style=lolmatlab,caption={\gls{p2pp}-Decoder in MATLAB},label=lst:matlab-p2ppdecoder]
function [availableData] = protocolDecode(byte)
    global SOH EOT ESC ESC_XOR DA TA
    global state
    global dataLength data dataReady dataBytesRead escapeDetected lastByte

    availableData = 0;
    
    % Switch states
    switch state
        % Await preamble
        case 0
            if byte == DA
                % ignored
            elseif (byte == TA) && (lastByte == DA)
                state = 1; % preamble detected
            else
                % error state
            end

            lastByte = byte;

        % Await SOH
        case 1
            if byte == SOH
                state = 2;
            else
                % error state
            end

        % Read length
        case 2
            dataLength = byte;
            dataBytesRead = 0;
            data = zeros(dataLength, 1, 'uint8');
            state = 3;

        % Read data bytes
        case 3
            if byte == ESC
                escapeDetected = true;
            else
                if escapeDetected
                    byte = bitxor(byte, ESC_XOR);
                    escapeDetected = false;
                end

                dataBytesRead = dataBytesRead + 1;
                data(dataBytesRead) = byte;

                if dataBytesRead == dataLength
                    state = 4;
                end
            end

        % Await EOT
        case 4
            if byte == EOT
                state = 0;
                dataReady = true;
                availableData = dataBytesRead;
            else
                % error state
            end
    end
end
\end{lstlisting}

\begin{lstlisting}[style=lolcsharp,caption={\gls{p2pp}-Decoder in C\#},label=lst:csharp-p2ppdecoder]
class ProtocolDecoder
{
	private const int SOH = 1;
	private const int EOT = 4;
	private const int ESC = 27;
	private const int ESC_XOR = 66;
	private const int DA = 218;
	private const int TA = 122;
	
	private int dataLength = 0;
	private int dataBytesRead = 0;
	private byte[] data;

	private int lastdatum = 0;
	private bool escapeDetected = false;
	private int state = 0;

	public event EventHandler<DataEvent> DataReady;

	/// <summary>
	/// Decodes the specified data btye.
	/// </summary>
	/// <param name="datum">The data byte.</param>
	public void Decode(int datum)
	{
		switch (state)
		{
			case 0: // Await preamble
			{
				if (datum == DA) { /* ignored */ }
				else if ((datum == TA) && (lastdatum == DA))
				{
					// preamble detected
					state = 1;
					break;
				}
				else { /* error case */ }

				// Remember datum for protocol decoding
				lastdatum = datum;
				break;
			}

			case 1: // Await SOH
			{
				if (datum == SOH)
				{
					state = 2; break;
				}
				
				state = 0; // error case
				break;
			}

			case 2: // Read length
			{
				dataLength = datum;
				dataBytesRead = 0;
				data = new byte[dataLength];
				state = 3;
				break;
			}

			case 3: // Read datum
			{
				if (datum == ESC)
				{
					escapeDetected = true;
				}
				else
				{
					// decode escaped byte
					if (escapeDetected)
					{
						datum = datum ^ ESC_XOR;
						escapeDetected = false;
					}

					// store data byte and increment counter
					data[dataBytesRead++] = (byte)datum;

					// if all bytes are read, wait for EOT
					if (dataBytesRead == dataLength)
					{
						state = 4;
					}
				}
				break;
			}
			case 4: // await EOT
			{
				if (datum == EOT)
				{
					state = 0;

					// raise event
					var handler = DataReady;
					if (handler != null)
					{
						handler(this, new DataEvent(data));
					}
					
					break;
				}
				
				state = 0; // error case
				break;
			}
		}
	}
}
\end{lstlisting}
\setchapterpreamble[u]{%
\dictum[Luhmann]{Die Klassiker sind Klassiker, weil sie Klassiker sind \dots}}
\chapter{Fusionsalgorithmus}

Exemplarischer Verweis auf ein Buch mit dieser Aussage, vgl. \cite{Filieri}, außerdem \cite{Tsang}. Dort werden \glspl{mems} erwähnt. So ein \gls{mems} ist toll.

\begin{figure}[htbp]
	\centering
	\includegraphics[width=\textwidth]{./images/matlab/rollpitchyaw45-2.png}
	\caption[Extraktion der \textsc{Euler}'schen Winkel]{Extraktion der \textsc{Euler}'schen Winkel. Deutlich zu erkennen sind die Singularitäten bei Sekunden 13, 17 und 29 und 32.}
\end{figure}


\appendix							% Beginn des Anhangs

\begin{appendices}
\chapter{Umprogrammieren der OpenSDA-Firmware unter Linux}
\label{chap:opensda_linux}

\lstdefinestyle{lolbash}{
    language={bash}, 
		basicstyle=\ttfamily\tiny,
		frame=single,
    moredelim=**[is][\slshape]{`}{`},
    moredelim=**[is][\color{orange}]{°}{°},
}

Wie in Abschnitt~\ref{subsec:opensda_windows8.1} erwähnt, kann das \gls{frdm-kl25z} unter Windows 8.1 nicht ordnungsgemäß
in Betrieb genommen werden, sobald die Firmware einen \gls{msd}-Modus verwendet. Dies betrifft sowohl den Programmer
der \gls{mcu}, als auch den Programmer des \gls{OpenSDA}-Chips selbst.

Im Folgenden findet sich eine Beschreibung des Flashvorganges unter Linux, nachdem das Board auch unter Windows 8.1
wieder verwendet werden kann. Dieser Vorgang sollte auch dann zum Erfolg führen, wenn das Board unter Windows 8.1
bereits fehlerhaft (d.h. mit Abbruch des Dateitransfers) geflasht wurde.

Zuerst wird das Board bei gedrückter Reset-Taste angeschlossen, wodurch es im Bootloader-Modus startet. Die Ausgabe 
von \texttt{lsusb} sollte das Gerät mit der ID \texttt{2504:0200} auflisten.

\begin{lstlisting}[style=lolbash]
user@host:~$ lsusb
Bus 002 Device 001: ID 1d6b:0002 Linux Foundation 2.0 root hub
Bus 007 Device 001: ID 1d6b:0001 Linux Foundation 1.1 root hub
Bus 006 Device 001: ID 1d6b:0001 Linux Foundation 1.1 root hub
°Bus 005 Device 008: ID 2504:0200°
Bus 005 Device 001: ID 1d6b:0001 Linux Foundation 1.1 root hub
Bus 001 Device 001: ID 1d6b:0002 Linux Foundation 2.0 root hub
Bus 004 Device 001: ID 1d6b:0001 Linux Foundation 1.1 root hub
Bus 003 Device 002: ID 0483:2016 STMicroelectronics Fingerprint Reader
Bus 003 Device 003: ID 0a5c:2110 Broadcom Corp. BCM2045B (BDC-2) [Bluetooth Controller]
Bus 003 Device 001: ID 1d6b:0001 Linux Foundation 1.1 root hub
\end{lstlisting}

Mittels \texttt{dmesg} wird nun ermittelt, als welches Device das USB-Gerät angemeldet wurde.

\begin{lstlisting}[style=lolbash]
user@host:~$ dmesg
[ 2178.532100] usb 5-2: new full-speed USB device number 10 using uhci_hcd
[ 2178.703183] usb 5-2: New USB device found, idVendor=2504, idProduct=0200
[ 2178.703193] usb 5-2: New USB device strings: Mfr=1, Product=2, SerialNumber=3
°[ 2178.703201] usb 5-2: Product: OpenSDA MSD APP°
°[ 2178.703207] usb 5-2: Manufacturer: FREESCALE SEMICONDUCTOR INC.°
°[ 2178.703213] usb 5-2: SerialNumber: 0123456789ABCDEF°
°[ 2178.706301] usb-storage 5-2:1.0: USB Mass Storage device detected°
[ 2178.706467] scsi11 : usb-storage 5-2:1.0
°[ 2179.709257] scsi 11:0:0:0: Direct-Access     FSL      FSL/PEMICRO MSD  0001 PQ: 0 ANSI: 4°
[ 2179.709869] sd 11:0:0:0: Attached scsi generic sg2 type 0
°[ 2179.718217] sd 11:0:0:0: [sdb] 1983999 512-byte logical blocks: (1.01 GB/968 MiB)°
[ 2179.721525] sd 11:0:0:0: [sdb] Write Protect is off
[ 2179.721531] sd 11:0:0:0: [sdb] Mode Sense: 00 00 00 00
[ 2179.724171] sd 11:0:0:0: [sdb] Asking for cache data failed
[ 2179.724183] sd 11:0:0:0: [sdb] Assuming drive cache: write through
[ 2179.742182] sd 11:0:0:0: [sdb] Asking for cache data failed
[ 2179.742187] sd 11:0:0:0: [sdb] Assuming drive cache: write through
[ 2179.766202]  sdb:
[ 2179.783187] sd 11:0:0:0: [sdb] Asking for cache data failed
[ 2179.783193] sd 11:0:0:0: [sdb] Assuming drive cache: write through
[ 2179.783197] sd 11:0:0:0: [sdb] Attached SCSI removable disk
\end{lstlisting}

Man kann erkennen, dass das Laufwerk als \texttt{/dev/sdb} angemeldet wurde.
Es kann nun mittels \texttt{mount -t vfat} gemountet und der Erfolg überprüft werden.

\begin{lstlisting}[style=lolbash]
user@host:~$ sudo mount -t vfat /dev/sdb /mnt
user@host:~$ ls -lisa /mnt
insgesamt 84
  1 16 drwxr-xr-x  2 root root 16384 Jan  1  1970 .
  2  4 drwxr-xr-x 23 root root  4096 Nov 21 02:12 ..
102 16 -r-xr-xr-x  1 root root   512 Aug  8  2012 FSL_WEB.HTM
100 16 -r-xr-xr-x  1 root root    68 Aug  8  2012 LASTSTAT.TXT
101 16 -r-xr-xr-x  1 root root  1536 Aug  8  2012 SDA_INFO.HTM
103 16 -r-xr-xr-x  1 root root   512 Aug  8  2012 TOOLS.HTM
\end{lstlisting}

Es ist zu beachten, dass der Mountvorgang durchaus mehrere Minuten dauern kann.

Die Datei \texttt{LASTSTAT.TXT} beinhaltet den letzten Status der Firmware, womit man
den erfolgreichen Verlauf des Mountvorganges überprüfen kann.

\begin{lstlisting}[style=lolbash]
user@host:~$ cat /mnt/LASTSTAT.TXT
°Ready.°
\end{lstlisting}

Die neue Firmware kann nun durch einen Kopierbefehl an das Gerät gesendet werden; Im Falle
der \emph{Segger J-Link}-Variante könnte dies etwa wie folgt aussehen:

\begin{lstlisting}[style=lolbash]
user@host:~$ sudo cp JLink_OpenSDA.sda /mnt
user@host:~$ cat /mnt/LASTSTAT.TXT
°Completed.°
\end{lstlisting}

Ein \texttt{Completed.} signalisiert den erfolgreichen Flash-Vorgang.

Nach dem Auswerfen des Laufwerkes mittels \texttt{umount}

\begin{lstlisting}[style=lolbash]
user@ahost:~$ sudo umount /mnt
\end{lstlisting}

und einem Neustart des Boards im regulären Modus (d.h. ohne gedrückten Reset-Taster) meldet 
sich das Board mit der neuen Firmware an:

\begin{lstlisting}[style=lolbash]
user@host:~$ lsusb
Bus 002 Device 001: ID 1d6b:0002 Linux Foundation 2.0 root hub
Bus 007 Device 001: ID 1d6b:0001 Linux Foundation 1.1 root hub
Bus 006 Device 001: ID 1d6b:0001 Linux Foundation 1.1 root hub
°Bus 005 Device 011: ID 1366:0101 SEGGER J-Link ARM°
Bus 005 Device 001: ID 1d6b:0001 Linux Foundation 1.1 root hub
Bus 001 Device 001: ID 1d6b:0002 Linux Foundation 2.0 root hub
Bus 004 Device 001: ID 1d6b:0001 Linux Foundation 1.1 root hub
Bus 003 Device 002: ID 0483:2016 STMicroelectronics Fingerprint Reader
Bus 003 Device 003: ID 0a5c:2110 Broadcom Corp. BCM2045B (BDC-2) [Bluetooth Controller]
Bus 003 Device 001: ID 1d6b:0001 Linux Foundation 1.1 root hub
\end{lstlisting}

Analog kann die Ausgabe mittels \texttt{dmesg} überprüft werden.

\begin{lstlisting}[style=lolbash]
user@host:~$ dmesg
[ 2637.380139] usb 5-2: USB disconnect, device number 10
[ 2638.312070] usb 5-2: new full-speed USB device number 11 using uhci_hcd
[ 2638.479164] usb 5-2: New USB device found, idVendor=1366, idProduct=0101
[ 2638.479176] usb 5-2: New USB device strings: Mfr=1, Product=2, SerialNumber=3
°[ 2638.479183] usb 5-2: Product: J-Link°
°[ 2638.479189] usb 5-2: Manufacturer: SEGGER°
°[ 2638.479195] usb 5-2: SerialNumber: 000621000000°
\end{lstlisting}

Wird stattdessen die \emph{P\&E Microcomputer Debug OpenSDA}-Variante geflasht, lauten die letzten
Ausgaben sinngemäß

\begin{lstlisting}[style=lolbash]
user@host:~$ lsusb
Bus 002 Device 001: ID 1d6b:0002 Linux Foundation 2.0 root hub
Bus 007 Device 001: ID 1d6b:0001 Linux Foundation 1.1 root hub
Bus 006 Device 001: ID 1d6b:0001 Linux Foundation 1.1 root hub
°Bus 005 Device 013: ID 1357:0089 P&E Microcomputer Systems°
Bus 005 Device 001: ID 1d6b:0001 Linux Foundation 1.1 root hub
Bus 001 Device 001: ID 1d6b:0002 Linux Foundation 2.0 root hub
Bus 004 Device 001: ID 1d6b:0001 Linux Foundation 1.1 root hub
Bus 003 Device 002: ID 0483:2016 STMicroelectronics Fingerprint Reader
Bus 003 Device 003: ID 0a5c:2110 Broadcom Corp. BCM2045B (BDC-2) [Bluetooth Controller]
Bus 003 Device 001: ID 1d6b:0001 Linux Foundation 1.1 root hub
\end{lstlisting}

und 

\begin{lstlisting}[style=lolbash]
sunside@aquitaine:~$ dmesg | tail
[ 3015.141218] sd 12:0:0:0: [sdb] Attached SCSI removable disk
[ 3346.412156] usb 5-2: USB disconnect, device number 12
[ 3347.768119] usb 5-2: new full-speed USB device number 13 using uhci_hcd
[ 3347.950185] usb 5-2: New USB device found, idVendor=1357, idProduct=0089
[ 3347.950195] usb 5-2: New USB device strings: Mfr=1, Product=3, SerialNumber=5
°[ 3347.950202] usb 5-2: Product: OpenSDA Hardware°
°[ 3347.950208] usb 5-2: Manufacturer: P&E Microcomputer Systems Inc.°
°[ 3347.950214] usb 5-2: SerialNumber: SDADBB27E5D°
[ 3347.953292] cdc_acm 5-2:1.0: This device cannot do calls on its own. It is not a modem.
[ 3347.953332] cdc_acm 5-2:1.0: ttyACM0: USB ACM device
\end{lstlisting}
\chapter{Weitere Listings}
\label{chap:listings}

\begin{lstlisting}[style=lolc,caption={Bit Manipulation Engine: \texttt{XOR}},label=lst:bme-xor]
/* ueblicher C-Code */

GPIOA_PDOR ^= 0x02; 
// 0000005E 0x.... LDR R0,??DataTable6_5 ;; 0x400ff000 
// 00000060 0x6800 LDR R0,[R0, #+0] 
// 00000062 0x2102 MOVS R1,#+2 
// 00000064 0x4041 EORS R1,R1,R0 
// 00000066 0x.... LDR R0,??DataTable6_5 ;; 0x400ff000 
// 00000068 0x6001 STR R1,[R0, #+0] 

/* Verwendung der BME */
#define BME_XOR_ADDR(ADDR) (*(volatile uint32_t *)(((uint32_t)ADDR) | (3<<26))) 

BME_XOR_ADDR(&GPIOA_PDOR) = 0x02; 
// 00000014 0x.... LDR R0,??DataTable6_6 ;; 0x4c0ff000 
// 00000016 0x2102 MOVS R1,#+2 
// 00000018 0x6001 STR R1,[R0, #+0] 
\end{lstlisting}

\begin{lstlisting}[style=lolmatlab,caption={Setup des \gls{p2pp}-Decoders in MATLAB},label=lst:matlab-p2ppdecoder-setup]
function prepareProtocolDecode
        
    % Definitions
    global SOH EOT ESC ESC_XOR DA TA
    SOH     = uint8(1);
    EOT     = uint8(4);
    ESC     = uint8(27);
    ESC_XOR = uint8(66);
    DA      = uint8(218);
    TA      = uint8(122);

    global state
    state = 0;
    
    % Decoding variables
    global dataLength data dataBytesRead escapeDetected dataReady lastByte
    lastByte = NaN;
    dataLength = 0;
    data = [];
    dataBytesRead = 0;
    escapeDetected = false;
    dataReady = false;
    
    % Data counters
    global sensorDataCount RPY
    RPY = 1;
    sensorDataCount = zeros(3,1);
end
\end{lstlisting}

\begin{lstlisting}[style=lolmatlab,caption={\gls{p2pp}-Decoder in MATLAB},label=lst:matlab-p2ppdecoder]
function [availableData] = protocolDecode(byte)
    global SOH EOT ESC ESC_XOR DA TA
    global state
    global dataLength data dataReady dataBytesRead escapeDetected lastByte

    availableData = 0;
    
    % Switch states
    switch state
        % Await preamble
        case 0
            if byte == DA
                % ignored
            elseif (byte == TA) && (lastByte == DA)
                state = 1; % preamble detected
            else
                % error state
            end

            lastByte = byte;

        % Await SOH
        case 1
            if byte == SOH
                state = 2;
            else
                % error state
            end

        % Read length
        case 2
            dataLength = byte;
            dataBytesRead = 0;
            data = zeros(dataLength, 1, 'uint8');
            state = 3;

        % Read data bytes
        case 3
            if byte == ESC
                escapeDetected = true;
            else
                if escapeDetected
                    byte = bitxor(byte, ESC_XOR);
                    escapeDetected = false;
                end

                dataBytesRead = dataBytesRead + 1;
                data(dataBytesRead) = byte;

                if dataBytesRead == dataLength
                    state = 4;
                end
            end

        % Await EOT
        case 4
            if byte == EOT
                state = 0;
                dataReady = true;
                availableData = dataBytesRead;
            else
                % error state
            end
    end
end
\end{lstlisting}

\begin{lstlisting}[style=lolcsharp,caption={\gls{p2pp}-Decoder in C\#},label=lst:csharp-p2ppdecoder]
class ProtocolDecoder
{
	private const int SOH = 1;
	private const int EOT = 4;
	private const int ESC = 27;
	private const int ESC_XOR = 66;
	private const int DA = 218;
	private const int TA = 122;
	
	private int dataLength = 0;
	private int dataBytesRead = 0;
	private byte[] data;

	private int lastdatum = 0;
	private bool escapeDetected = false;
	private int state = 0;

	public event EventHandler<DataEvent> DataReady;

	/// <summary>
	/// Decodes the specified data btye.
	/// </summary>
	/// <param name="datum">The data byte.</param>
	public void Decode(int datum)
	{
		switch (state)
		{
			case 0: // Await preamble
			{
				if (datum == DA) { /* ignored */ }
				else if ((datum == TA) && (lastdatum == DA))
				{
					// preamble detected
					state = 1;
					break;
				}
				else { /* error case */ }

				// Remember datum for protocol decoding
				lastdatum = datum;
				break;
			}

			case 1: // Await SOH
			{
				if (datum == SOH)
				{
					state = 2; break;
				}
				
				state = 0; // error case
				break;
			}

			case 2: // Read length
			{
				dataLength = datum;
				dataBytesRead = 0;
				data = new byte[dataLength];
				state = 3;
				break;
			}

			case 3: // Read datum
			{
				if (datum == ESC)
				{
					escapeDetected = true;
				}
				else
				{
					// decode escaped byte
					if (escapeDetected)
					{
						datum = datum ^ ESC_XOR;
						escapeDetected = false;
					}

					// store data byte and increment counter
					data[dataBytesRead++] = (byte)datum;

					// if all bytes are read, wait for EOT
					if (dataBytesRead == dataLength)
					{
						state = 4;
					}
				}
				break;
			}
			case 4: // await EOT
			{
				if (datum == EOT)
				{
					state = 0;

					// raise event
					var handler = DataReady;
					if (handler != null)
					{
						handler(this, new DataEvent(data));
					}
					
					break;
				}
				
				state = 0; // error case
				break;
			}
		}
	}
}
\end{lstlisting}

\begin{lstlisting}[style=lolc,caption={Gravitations-Beispiel mittels kalman-clib},label=lst:kalman-clib]
/*!
* The formulas used are:
* s = s + v*T + g*0.5*T^2
* v = v + g*T
* g = g
*
* The time constant is set to T = 1s.
* The initial estimation of the gravity constant is set to 6 m/s^2.
*/

#include <assert.h>
#include "kalman_example_gravity.h"

// create the filter structure
#define KALMAN_NAME gravity
#define KALMAN_NUM_STATES 3
#define KALMAN_NUM_INPUTS 0
#include "kalman_factory_filter.h"

// create the measurement structure
#define KALMAN_MEASUREMENT_NAME position
#define KALMAN_NUM_MEASUREMENTS 1
#include "kalman_factory_measurement.h"

// clean up
#include "kalman_factory_cleanup.h"

/*!
* \brief Initializes the gravity Kalman filter
*/
static void kalman_gravity_init()
{
    /* initialize the filter structures */
    kalman_t *kf = kalman_filter_gravity_init();
    kalman_measurement_t *kfm = kalman_filter_gravity_measurement_position_init();

    /* set initial state */
    matrix_t *x = kalman_get_state_vector(kf);
    x->data[0] = 0; // s_i
    x->data[1] = 0; // v_i
    x->data[2] = 6; // g_i

    /* set state transition */
    matrix_t *A = kalman_get_state_transition(kf);
    
    /* set time constant */
    const matrix_data_t T = 1;

    /* transition of x to s */
    matrix_set(A, 0, 0, 1);   // 1
    matrix_set(A, 0, 1, T);   // T
    matrix_set(A, 0, 2, (matrix_data_t)0.5*T*T); // 0.5 * T^2
    
    /* transition of x to v */
    matrix_set(A, 1, 0, 0);   // 0
    matrix_set(A, 1, 1, 1);   // 1
    matrix_set(A, 1, 2, T);   // T

    /* transition of x to g */
    matrix_set(A, 2, 0, 0);   // 0
    matrix_set(A, 2, 1, 0);   // 0
    matrix_set(A, 2, 2, 1);   // 1

    /* set covariance */
    matrix_t *P = kalman_get_system_covariance(kf);

    matrix_set_symmetric(P, 0, 0, (matrix_data_t)0.1);   // var(s)
    matrix_set_symmetric(P, 0, 1, 0);   // cov(s,v)
    matrix_set_symmetric(P, 0, 2, 0);   // cov(s,g)

    matrix_set_symmetric(P, 1, 1, 1);   // var(v)
    matrix_set_symmetric(P, 1, 2, 0);   // cov(v,g)

    matrix_set_symmetric(P, 2, 2, 1);   // var(g)

    /* set measurement transformation */
    matrix_t *H = kalman_get_measurement_transformation(kfm);

    matrix_set(H, 0, 0, 1);     // z = 1*s 
    matrix_set(H, 0, 1, 0);     //   + 0*v
    matrix_set(H, 0, 2, 0);     //   + 0*g

    /* set process noise */
    matrix_t *R = kalman_get_process_noise(kfm);

    matrix_set(R, 0, 0, (matrix_data_t)0.5);     // var(s)
}

// define measurements.
//
// MATLAB source
// -------------
// s = s + v*T + g*0.5*T^2; 
// v = v + g*T;
#define MEAS_COUNT (15)
static matrix_data_t real_distance[MEAS_COUNT] = {
    (matrix_data_t)0,
    (matrix_data_t)4.905,
    /* ... snip ... */
    (matrix_data_t)828.94,
    (matrix_data_t)961.38 };

// define measurement noise with variance 0.5
//
// MATLAB source
// -------------
// noise = 0.5^2*randn(15,1);
static matrix_data_t measurement_error[MEAS_COUNT] = {
    (matrix_data_t)0.13442,
    /* ... snip ... */
    (matrix_data_t)-0.015764,
    (matrix_data_t)0.17869 };

/*!
* \brief Runs the gravity Kalman filter.
*/
void kalman_gravity_demo()
{
    // initialize the filter
    kalman_gravity_init();

    // fetch structures
    kalman_t *kf = &kalman_filter_gravity;
    kalman_measurement_t *kfm = &kalman_filter_gravity_measurement_position;

    matrix_t *x = kalman_get_state_vector(kf);
    matrix_t *z = kalman_get_measurement_vector(kfm);
    
    // filter!
    for (int i = 0; i < MEAS_COUNT; ++i)
    {
        // prediction.
        kalman_predict(kf);

        // measure ...
        matrix_data_t measurement = real_distance[i] + measurement_error[i];
        matrix_set(z, 0, 0, measurement);

        // update
        kalman_correct(kf, kfm);
    }

    // fetch estimated g
    matrix_data_t g_estimated = x->data[2];
    assert(g_estimated > 9 && g_estimated < 10);
}
\end{lstlisting}

\begin{lstlisting}[style=lolc,caption={Gravitations-Beispiel mittels libfixkalman},label=lst:libfixkalman]
/*!
* The formulas used are:
* s = s + v*T + g*0.5*T^2
* v = v + g*T
* g = g
*
* The time constant is set to T = 1s.
* The initial estimation of the gravity constant is set to 6 m/s^2.
*/

// no control inputs given, so _uc type used
kalman16_uc_t kf;

kalman16_observation_t kfm;

#define matrix_set(matrix, row, column, value) \
    matrix->data[row][column] = value

#define matrix_set_symmetric(matrix, row, column, value) \
    matrix->data[row][column] = value; \
    matrix->data[column][row] = value

/*!
* \brief Initializes the gravity Kalman filter
*/
static void kalman_gravity_init()
{
    /* initialize the filter structures */
    kalman_filter_initialize_uc(&kf, KALMAN_NUM_STATES);
    kalman_observation_initialize(&kfm, KALMAN_NUM_STATES, KALMAN_NUM_MEASUREMENTS);

    /* set initial state */
    mf16 *x = kalman_get_state_vector_uc(&kf);
    x->data[0][0] = 0; // s_i
    x->data[1][0] = 0; // v_i
    x->data[2][0] = fix16_from_float(6); // g_i

    /* set state transition */
    mf16 *A = kalman_get_state_transition_uc(&kf);
    
    /* set time constant */
    const fix16_t T = fix16_one;
    const fix16_t Tsquare = fix16_sq(T);

    /* helper */
    const fix16_t fix16_half = fix16_from_float(0.5);

    /* transition of x to s */
    matrix_set(A, 0, 0, fix16_one);   // 1
    matrix_set(A, 0, 1, T);   // T
    matrix_set(A, 0, 2, fix16_mul(fix16_half, Tsquare)); // 0.5 * T^2
    
    /* transition of x to v */
    matrix_set(A, 1, 0, 0);   // 0
    matrix_set(A, 1, 1, fix16_one);   // 1
    matrix_set(A, 1, 2, T);   // T

    /* transition of x to g */
    matrix_set(A, 2, 0, 0);   // 0
    matrix_set(A, 2, 1, 0);   // 0
    matrix_set(A, 2, 2, fix16_one);   // 1

    /* set covariance */
    mf16 *P = kalman_get_system_covariance_uc(&kf);

    matrix_set_symmetric(P, 0, 0, fix16_half);   // var(s)
    matrix_set_symmetric(P, 0, 1, 0);   // cov(s,v)
    matrix_set_symmetric(P, 0, 2, 0);   // cov(s,g)

    matrix_set_symmetric(P, 1, 1, fix16_one);   // var(v)
    matrix_set_symmetric(P, 1, 2, 0);   // cov(v,g)

    matrix_set_symmetric(P, 2, 2, fix16_one);   // var(g)

    /* set system process noise */
    mf16 *Q = kalman_get_system_process_noise_uc(&kf);
    mf16_fill(Q, F16(0.0001));

    /* set measurement transformation */
    mf16 *H = kalman_get_observation_transformation(&kfm);

    matrix_set(H, 0, 0, fix16_one);     // z = 1*s 
    matrix_set(H, 0, 1, 0);     //   + 0*v
    matrix_set(H, 0, 2, 0);     //   + 0*g

    /* set process noise */
    mf16 *R = kalman_get_observation_process_noise(&kfm);

    matrix_set(R, 0, 0, fix16_half);     // var(s)
}

// define measurements.
//
// MATLAB source
// -------------
// s = s + v*T + g*0.5*T^2; 
// v = v + g*T;
#define MEAS_COUNT (15)
static fix16_t real_distance[MEAS_COUNT] = {
    F16(4.905),
    /* ... snip ... */
    F16(961.38),
    F16(1103.6) };

// define measurement noise with variance 0.5
//
// MATLAB source
// -------------
// noise = 0.5^2*randn(15,1);
static fix16_t measurement_error[MEAS_COUNT] = {
    F16(0.13442),
    /* ... snip ... */
    F16(-0.015764),
    F16(0.17869) };

/*!
* \brief Runs the gravity Kalman filter.
*/
void kalman_gravity_demo()
{
    // initialize the filter
    kalman_gravity_init();

    mf16 *x = kalman_get_state_vector_uc(&kf);
    mf16 *z = kalman_get_observation_vector(&kfm);
    
    // filter!
    uint_fast16_t i;
    for (i = 0; i < MEAS_COUNT; ++i)
    {
        // prediction.
        kalman_predict_uc(&kf);

        // measure ...
        fix16_t measurement = fix16_add(real_distance[i], measurement_error[i]);
        matrix_set(z, 0, 0, measurement);

        // update
        kalman_correct_uc(&kf, &kfm);
    }

    // fetch estimated g
    const fix16_t g_estimated = x->data[2][0];
    const float value = fix16_to_float(g_estimated);
    assert(value > 9.7 && value < 10);
}
\end{lstlisting}

\begin{lstlisting}[style=lolc,caption={Fusion: \textsc{Euler}'sche Winkel aus \gls{dcm}},label=lst:libfixmath_dcm_euler]
void sensor_dcm2rpy(
	const mf16 *RESTRICT const dcm, 
	fix16_t *RESTRICT const roll, 
	fix16_t *RESTRICT const pitch, 
	fix16_t *RESTRICT const yaw)
{
    *pitch = -fix16_asin(dcm->data[0][2]);
    *roll  = fix16_atan2(dcm->data[1][2], dcm->data[2][2]);
    *yaw   = fix16_atan2(dcm->data[0][1], dcm->data[0][0]);
}
\end{lstlisting}

\begin{lstlisting}[style=lolc,caption={Fusion: Ermittlung des Quaternions aus den fusionierten Datem},label=lst:quaternion_from_kalman]
HOT NONNULL LEAF
static void fetch_quaternion_opt2(register qf16 *RESTRICT const quat)
{
	const register mf16 *const x2 = kalman_get_state_vector_uc(&kf_orientation);
	const register mf16 *const x3 = kalman_get_state_vector_uc(&kf_attitude);

	// m00 = R(1, 1);    m01 = R(1, 2);    m02 = R(1, 3);
	// m10 = R(2, 1);    m11 = R(2, 2);    m12 = R(2, 3);
	// m20 = R(3, 1);    m21 = R(3, 2);    m22 = R(3, 3);

	const fix16_t m10 = x2->data[0][0];
	const fix16_t m11 = x2->data[1][0];
	const fix16_t m12 = x2->data[2][0];

	const fix16_t m20 = -x3->data[0][0];
	const fix16_t m21 = -x3->data[1][0];
	const fix16_t m22 = -x3->data[2][0];

	// calculate cross product for C1
	// m0 = cross([m10 m11 m12], [m20 m21 m22])
	// -->
	//      m00 = m11*m22 - m12*m21
	//      m01 = m12*m20 - m10*m22
	//      m02 = m10*m21 - m11*m20
	fix16_t m00 = fix16_sub(fix16_mul(m11, m22), fix16_mul(m12, m21));
	fix16_t m01 = fix16_sub(fix16_mul(m12, m20), fix16_mul(m10, m22));
	fix16_t m02 = fix16_sub(fix16_mul(m10, m21), fix16_mul(m11, m20));

	// normalize C1 
	const register fix16_t norm = norm3(m00, m01, m02);
	m00 = fix16_div(m00, norm);
	m01 = fix16_div(m01, norm);
	m02 = fix16_div(m02, norm);

	// "Angel" code
	// http://www.euclideanspace.com/maths/geometry/rotations/conversions/matrixToQuaternion/

	fix16_t qw, qx, qy, qz;

	// check the matrice's trace
	const register fix16_t trace = fix16_add(m00, fix16_add(m11, m22));
	if (trace > 0)
	{
		/*
		s = 0.5 / sqrt(trace + 1.0);
		qw = 0.25 / s;
		qx = ( R(3,2) - R(2,3) ) * s;
		qy = ( R(1,3) - R(3,1) ) * s;
		qz = ( R(2,1) - R(1,2) ) * s;
		*/

		const fix16_t s = fix16_div(F16(0.5), fix16_sqrt(fix16_add(F16(1.0), trace)));

		qw = fix16_div(F16(0.25), s);
		qx = fix16_mul(fix16_sub(m21, m12), s);
		qy = fix16_mul(fix16_sub(m02, m20), s);
		qz = fix16_mul(fix16_sub(m10, m01), s);
	}
	else
	{
		if (m00 > m11 && m00 > m22)
		{
			/*
			s = 2.0 * sqrt( 1.0 + R(1,1) - R(2,2) - R(3,3));
			qw = (R(3,2) - R(2,3) ) / s;
			qx = 0.25 * s;
			qy = (R(1,2) + R(2,1) ) / s;
			qz = (R(1,3) + R(3,1) ) / s;
			*/
			const fix16_t s = fix16_mul(F16(2), fix16_sqrt(fix16_add(F16(1), fix16_sub(m00, fix16_add(m11, m22)))));

			qw = fix16_div(fix16_sub(m21, m12), s);
			qx = fix16_mul(F16(0.25), s);
			qy = fix16_div(fix16_add(m01, m10), s);
			qz = fix16_div(fix16_add(m02, m20), s);
		}
		else if (m11 > m22)
		{
			/*
			s = 2.0 * sqrt( 1.0 + R(2,2) - R(1,1) - R(3,3));
			qw = (R(1,3) - R(3,1) ) / s;
			qx = (R(1,2) + R(2,1) ) / s;
			qy = 0.25 * s;
			qz = (R(2,3) + R(3,2) ) / s;
			*/
			const fix16_t s = fix16_mul(F16(2), fix16_sqrt(fix16_add(F16(1), fix16_sub(m11, fix16_add(m00, m22)))));

			qw = fix16_div(fix16_sub(m02, m20), s);
			qx = fix16_div(fix16_add(m01, m10), s);
			qy = fix16_mul(F16(0.25), s);
			qz = fix16_div(fix16_add(m12, m21), s);
		}
		else
		{
			/*
			s = 2.0 * sqrt( 1.0 + R(3,3) - R(1,1) - R(2,2) );
			qw = (R(2,1) - R(1,2) ) / s;
			qx = (R(1,3) + R(3,1) ) / s;
			qy = (R(2,3) + R(3,2) ) / s;
			qz = 0.25 * s;
			*/
			const fix16_t s = fix16_mul(F16(2), fix16_sqrt(fix16_add(F16(1), fix16_sub(m22, fix16_add(m00, m11)))));

			qw = fix16_div(fix16_sub(m10, m01), s);
			qx = fix16_div(fix16_add(m02, m20), s);
			qy = fix16_div(fix16_add(m12, m21), s);
			qz = fix16_mul(F16(0.25), s);
		}
	}

	// compose quaternion
	quat->a = qw;
	quat->b = qx;
	quat->c = qy;
	quat->d = qz;

	// normalizify
	qf16_normalize(quat, quat);
}
\end{lstlisting}

\begin{lstlisting}[style=lolc,caption={Fusion: Berechnung des Vorwärtsvektors aus Magnetometer und Accelerometer},label=lst:magnetometer_project]
HOT LEAF NONNULL
STATIC_INLINE void magnetometer_project(
	fix16_t *RESTRICT const mx, 
	fix16_t *RESTRICT const my, 
	fix16_t *RESTRICT const mz)
{
	const mf16 *const x = kalman_get_state_vector_uc(&kf_attitude);

	register const fix16_t acc_x = x->data[0][0];
	register const fix16_t acc_y = x->data[1][0];
	register const fix16_t acc_z = x->data[2][0];

	/************************************************************************/
	/* Instead of tilt corrected magnetometer, use TRIAD algorithm          */
	/************************************************************************/

	// calculate cross product for C1
	// m = cross([m_magnetometer.x m_magnetometer.y m_magnetometer.z], [m_accelerometer.x m_accelerometer.y m_accelerometer.z])
	// -->
	//      mx = m_magnetometer.y*m_accelerometer.z - m_magnetometer.z*m_accelerometer.y
	//      my = m_magnetometer.z*m_accelerometer.x - m_magnetometer.x*m_accelerometer.z
	//      mz = m_magnetometer.x*m_accelerometer.y - m_magnetometer.y*m_accelerometer.x
	*mx = fix16_sub(fix16_mul(m_magnetometer.y, acc_z), fix16_mul(m_magnetometer.z, acc_y));
	*my = fix16_sub(fix16_mul(m_magnetometer.z, acc_x), fix16_mul(m_magnetometer.x, acc_z));
	*mz = fix16_sub(fix16_mul(m_magnetometer.x, acc_y), fix16_mul(m_magnetometer.y, acc_x));

	// normalize C1 
	const register fix16_t norm = norm3(*mx, *my, *mz);
	*mx = fix16_div(*mx, norm);
	*my = fix16_div(*my, norm);
	*mz = fix16_div(*mz, norm);
}
\end{lstlisting}

\begin{lstlisting}[style=lolc,caption={Fusion: Korrektur des \emph{orientation}-Filters mittels Magnetometer},label=lst:magnetometer_correct]
HOT
static void fusion_update_orientation(register const fix16_t deltaT)
{
	/************************************************************************/
	/* Calculate metrics required for update                                */
	/************************************************************************/
	fix16_t mx, my, mz;
	magnetometer_project(&mx, &my, &mz);
	
	/************************************************************************/
	/* Prepare noise                                                        */
	/************************************************************************/

	tune_measurement_noise(&kfm_magneto);
	{
		mf16 *const R = &kfm_magneto.R;

		// anyway, overwrite covariance of projection
		matrix_set(R, 0, 0, fix16_mul(initial_r_projection, alpha1));
		matrix_set(R, 1, 1, fix16_mul(initial_r_projection, alpha1));
		matrix_set(R, 2, 2, fix16_mul(initial_r_projection, alpha1));
	}

	/************************************************************************/
	/* Prepare measurement                                                  */
	/************************************************************************/
	{
		mf16 *const z = &kfm_magneto.z;

		matrix_set(z, 0, 0, mx);
		matrix_set(z, 1, 0, my);
		matrix_set(z, 2, 0, mz);

		matrix_set(z, 3, 0, kf_attitude.x.data[3][0]);
		matrix_set(z, 4, 0, kf_attitude.x.data[4][0]);
		matrix_set(z, 5, 0, kf_attitude.x.data[5][0]);
	}

	/************************************************************************/
	/* Perform Kalman update                                                */
	/************************************************************************/

	kalman_correct_uc(&kf_orientation, &kfm_magneto);

	/************************************************************************/
	/* Re-orthogonalize and update state matrix                             */
	/************************************************************************/

	fusion_sanitize_state(&kf_orientation);
}
\end{lstlisting}
\end{appendices}

\backmatter					% Nachspann 

%% Stichwortverzeichnis anzeigen %%%%%%%%%%%%%%%%%%%%%%%%%%%%%%%%%%%%%%
\printindex

%% Bibliographie unter Verwendung von dinnat %%%%%%%%%%%%%%%%%%%%%%%%%%
%\setbibpreamble{Präambel des Literaturverzeichnisses}		% Text vor dem Verzeichnis
\bibliographystyle{dinat}
\bibliography{./bib/quellen}	% Sie benötigen einen *.bib-Datei

\end{document}